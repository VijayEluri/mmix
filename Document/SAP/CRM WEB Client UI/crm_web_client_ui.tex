%2010/11/05 The programming experience in CRM WEB Clint UI is frustrating.
%The expectation is that a book about it is availabe for reference.
%This Book is for ourself, at the same time, hope to save some effort for you.

\topglue .4in % This makes an inch of blank space (1in=2.54cm).
\centerline{\bf CRM WEB Client UI Programming}
\smallskip % This puts a little extra space after the title line.
\centerline{\it Eddie Wu}

\beginsection {Introduction.}\par 

\beginsection {Introduction.}\par 

\settabs
\+\indent&Working Experience: \quad & Table description \cr
\+&Table Name:                & Table description \cr
\+&CRMD\_ADMIN\_H                     & Male \cr
\+&TSOCM\_CR\_CONTEXT                     & Bachelor \cr
\+&CMCR\_OBJA     & $6$ years in J2EE Development ($11$ years total)\cr
\+&Mobile Phone:                & 13661788694\cr
\+&Email:                       & ueddieu@yahoo.com.cn\cr
%\+&Home Address		        & Room 502, Building 12, Nong 85, Xiangquan Road, Shanghai\cr
%\+&Postal Code		            & 200333\cr

\beginsection{Generic Interaction Layer Model.}\par

The model itself is described in the table .

Two important things we care about:
a. How the model is initialized. b. how do we navigate through the relation.

\beginsection{UI Framework}\par
Context Node class have suffix CN. sometimes with extra two bit of numbers. it is 
actually the wrapper class of the Model, getter/setter method will delegate to its
model.
besideds the relation between Model Node, there are also relations between 
context Node.

\beginsection{Runtime Lifecycle}\par
i.e. In what context will will our code be triggered?

\bye