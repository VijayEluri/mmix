%3/26/2009 1:10PM
%I just try to document something. it could be about myself, my son, or my wife.
%Meanwhile I want to practice the knowledge I just learned about typesetting.


\magnification\magstephalf
\parskip3pt
\baselineskip14pt
\def\AW{Addison\kern.1em--Wesley}
\font\sc=cmcsc10 %use lower case as (Monthly)


\centerline{\bf Biography on Self-study of Computer Science}
\bigskip
\centerline{Eddie Wu, Shanghai}
\bigskip
\centerline{March 28, 2009}
\bigskip
{
\baselineskip12pt

{\narrower\smallskip\noindent
{\bf Abstract.} I started to self-study computer science from the autumn of 
1999, till now, almost ten years has passed. During this period, I have
learned a lot of subjects on computer sciene, including 
Programming Language%(Java, C++, C)
, Operating System \& Kernel, Oracle Database, Algorithm \& Algorithm Analysis,
 Network Security, Computer Architecure, Regular Expression, etc. I try to wrote 
 down this kind of experience, just for records and for fun.
\smallskip}

}



%\medskip\noindent
\beginsection 1. Programming Language.

{\bf Java.} Java is the first programming language I meet during my self-study. I still 
remember the first Java book I bought is the Chinese Translation of ``Java How 
to Program''. especially I remember the confusion I have gotten when I see the 
self-reference structure in Java, I thought it is kind of recursive definition, 
because at that time I do not know that at 
the low level, reference is just a pointer, a word of memory, it does not have 
any type information. 

Before that, I already I have gotten some ideas about programming from college's 
Visuan Basic course, so it not a problem for me to understand $a=a+1$.

At that time, I am interested on playing GO, which is a ancient board game 
oriented from Chins and is popular in China, Japan, and Korea. The very first 
non-trivial program I have ever developered is a program used to play Go.

Basically, I need to translate the rule of GO into the Java language, I also 
need to consider the data structure to represent the board and status of Board.

I was a wonderful experience for me, I become familiar with the process of 
solving problem with computer.

\smallskip\noindent
{\bf C.} I start to learn C when I decide to take the National Programmer Exam.
I reference book I used is ``C How to program''. The most imressive small 
program I have written during the learing is a program to find the Horse 
Transversal in ChessBoard with Backtracking Algorithm.

Another impression I have is that Pointer in C is actually easy to understand compared to 
other technology or concept. 

After i start to work on j2EE platform, I still spend some time to read an old book called
``Secret in C'' during my leisure time. the most impressive thing i leared is the 
difference between Pointer and Array, they are different, but so closely related 
that they can be exchanged in most of the case.

\smallskip\noindent
{\bf C++.} After I have master the Java programming language, I start to have
a look on the C++ programming language. since I do not have opportunity to 
use it in my daily work, I focus on the theory and concept of C++ although I did
verify the concept with small C++ programs. The impressive books includes 
``More Effective C++'', ``C++ Design and Envolution'', ``Accelarated C++''.

Another reason to learn C++ is to understand the performance difference 
between Java and C++. But finally I realized that it is hard to compare 
the performance of two programming language even though you can have the 
same domain context and same algorithm.

%\medskip\noindent
\beginsection 2. Operating System \& Kernel.

My first touch with Operating System is during my preparation of Graduate Entry
Exam at the end of 2003. The textbook we use is Minix version 2 developed by 
Tanebum.%although I fail to pass the exam 

Later I also read the book ``The Design and Implementation of Unix'', which is
 based on Unix V6.

Starting from the autumn of 2006, I become interested in Linux Kernel, I have
read the following books.
 ``Linux Kernel Scenario Analysis'', which cover 2.4.0
 ``Understanding the Linux Kernel'',
 ``Understanding the Virturl Memory'',
 ``The Design and Implementation of Linux Kernel'', Robert love.
 
I like to know the details such how the synchronization are implemented in 
low level. Actually  this kind of knownledge help me understand java concurrency
programming better.

I also tried to build kernel by myself, the book ``Linux Kernel In a Nutsheel''
helps. In order to understand the build process, i learned GNU Make, but I have
 not gone too far in that direction till now.



%\medskip\noindent
\beginsection 3. Oracle Database.

During my training period of my first IT job at Zhongxin, I have the oppotunity 
to install 
Oracle 9i database and play with it for a while. Becuase of this previous 
experience, when I start my second IT job
at BLEUM, I am able to take the oppotunity of database maintenance.

Starting from April of 2006, I start to pursue Oracle Certified Professional on 
Database Administrator. and I got the OCP at July. During this period, I have 
systematically learned the ``Oracle 9i Concept'' and ``Oracle Programming---%
Expert on one one'', and ``Effective Oracle by Design''. I am grateful that I 
have the oppotunity to apply my Oracle Knowldge and Experience in Performance
tuning of the J2EE WEB applicaion developed by our team.

The most helpful knowledge about Oracle is the internal mechanism of how all 
kinds of features are implemented. For example, How is the SQL statement get
executed. without such kind of knowledge, you can not tell the difference
 between a good SQL statement and a bad one, because they may be equivalent 
 from logical point of view.
 Another tips is about practice and testing. Oracle provides lots of tools to 
 help you get the trace information about low level details. using them to 
 collect information will help you make the right decision.
 
  

%\medskip\noindent
\beginsection 4. Network Security.

I have security concern in my mind when I first time started to surf the 
internet, majorly because the rumor about virus and cracker. So I bought a book
about the security.

Later when I start to develop WEB applicaiton on J2EE Platform, I had the 
oppotunity to look deeper into the security mechanism. I master the concept like
symmetric encryption. computational infeasibility. In order to understand them,
I also need to know more mathematics, especially the number theory.

%\medskip\noindent
\beginsection 5. Computer Architecure.%\extra \par will make the \noindent fail.

From June of 2007, after I have finished a few important project in tight 
schedule. I got some time to read the MMIX---The RISC Computer for the new 
Millennium which I borrowed from Shanghai Library. It is the first time for me 
to have a close look at the computer 
architecture of a mordern RISC computer. Honestly I like to such kind of 
hardware details because I believed in solid fundation. I also bought volume 1 
of TAOCP({\it The Art of Computer Programming\/}) but I did not have enough 
time to finish the reading because I start to prepare for PMP training and Exam 
from March, 2008. 

After I pass the PMP Exam on September 2008, I reread TAOCP Volume~1, Fascicle~1
and I was even inspired to implemnent a simulator for it. Knuth already provided 
a perfect simulator for it, I just want to practice my Java programming skill by
reimplementing it in Java. Till now, I have finished all the non-floating 
instruction. In order to implement the complex mechanism such as ``Register Stack''
I have also borrowed ``MMIXWare --- '' and read it carefully to get all the details.
 
But I have not implemented floating point instruction. Although Java announce 
it is IEEE-754 compatible, it is actually not easy to implement MMIX's point 
instruction with Java. I have read Knuth's C code of MMIX simulator, I know how 
the floating point instruction can be simulated with Integer Arithmatic. Maybe I 
could have converted the C code to Java, but it sounds  boring to me, so the Java
Implementation of floating point MMIX instruction is still pending. 

Late, I also read ``Computer Architecure'' and ``Hardware Software Interface'' to 
know more about the pipeling, cache, multi-issue.

I also read the book like ``See MIPS Run'', which 

These book mentioned how hardware provides support to morder OS concepts, such 
memory mapping (Virtual Memory).

\beginsection 6. Algorithm and Algorithm Analysis.%\par

Actually I have spend many effort in learing Algorithm, the first Algorithm 
book I have read from cover to cover is ``Algorithm, Data Structure, and 
Applicaion---Described in C++''. It is a good text book for me as a 
self-studier, some algorithm, such as LZ compresion algorithm, is so beautiful
and impressive that I can still remember them today.

I become interested in Donald E. Knuth's ``The Art of Computer Programming'' 
since I first time heard of it on 2002. but I never have the time to go deep into
 it until June, 2007.
 
From May 2007, after I come back from San Francisco, one team mates recommends 
the MIT course ``Algorithm Introduction'' to me, which can be downloaded from
MIT freely. so I can take the course offline and practice my English skill. 

I also bought the Fascicle 2,~3,~4 of Volume~4 and downloaded Pre-Fascicle 0a, 
0b, 0c, 1a,and 1b. all of them are about ``Combinatorial Searching''.
During the reading of TAOCP, I have written my program to verify my 
understanding of the algorithm in the book. It provides so many ideas for you 
to explore, there are many more ideas in the exercises. 

%\medskip\noident
\beginsection 7. Regular Expression.

Regular expression is a handy powerful tool. I have the time to go deep into it
on May 2008. I learn it as one way to relax my mind when I am prepareing for 
PMP exam. I majorly refer to ``Master Regular Expression'', I practiced most 
of examples described in the book. The most important thing is that I not only 
start to use it in my daily work, but also help my teammated use it in their 
work. I also realized that most of tool actually have good build in
support of regular expression, they are just there waiting for us to use.

I also have the pleasure to find that my team mates improve their efficiency by 
 using regular expression in their daily work.

%\medskip\noindent
%{\bf Other}
\beginsection{8. Summary}\par

I have start to learn all kinds of different things. they are interesting and I
enjoy the course of learing. but till now I still did not reveal the way to 
take effect of those kinds of broad knowledge.   
 
 \bye
