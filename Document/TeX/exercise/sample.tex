  \hsize=29pc
\vsize=42pc
\footline={\tenrm Chapter 1 Inroduction\quad \dotfill \quad Page \folio}

%\hrulefill{}
\topglue 1in

\centerline{\bf A Bold, Centered Title}
\smallskip
\rightline{ \it Eddie Wu}

\beginsection 1. Plain \TeX nology

Subsequent paragraph {\it are} indented.\footnote*{The amount of intentation 
can be changed b y changing a parameter} (See?) The computer breaks a paragraph's
text into lines in a interesting way---set reference~[1]---and h%
yphenates words automatically when necessary.

\midinsert
\narrower\narrower
\noindent \llap{``}If there hadn't been room for this material on the present page,
it would have been inserted on the next one.''
\endinsert

\midinsert
\narrower
\noindent \llap{o }If there hadn't been room for this material on the present page,
it would have been inserted on the next one.
\endinsert

\midinsert
%\narrower
\noindent \item{o }If there hadn't been room for this material on the present page,
it would have been inserted on the next one.
\endinsert

\midinsert
\narrower
\noindent \llap{$<$good$>$}If there hadn't been room for this material on the present page,
it would have been inserted on the next one.
\endinsert

\midinsert
%\narrower
\noindent \item{$<$good$>$}If there hadn't been room for this material on the present page,
it would have been inserted on the next one.
\endinsert


\proclaim Theorem T. The typesetting of $math$ is discussed in chapters 16--19, and 
math symbols are summarized in Appendix~F.

\beginsection 2. Bibliography

\frenchspacing
\item{[1]} D.~E. Knuth and M.~F. Plass, ``Breaking paragraphs into lines,''{\sl
 Softw. pract. exp. \bf11} (1981). 1119--1184.
\bye
