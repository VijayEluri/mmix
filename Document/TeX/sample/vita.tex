% This page defines the format for Vitas

\font\tentt=cmtt10
\font\sltt=cmsltt10
\let\mainfont=\tenrm

\font\tenssb=cmssbx10
\font\twelvess=cmss10 at 12pt
\font\eightrm=cmr8
\font\eightsl=cmsl8
\def\sc{\eightrm}

\font\rus=lhwnr10
\font\grk=ygr10

\def\TeX{T\hbox{\hskip-.1667em\lower.424ex\hbox{E}\hskip-.125em X}}
\font\logo=manfnt % font used for the METAFONT logo
\def\MF{{\logo META}\-{\logo FONT}\spacefactor1000\relax}
\def\slMF{{\logo 89:;}\-{\logo <=>:}} % slanted version
\let\,=\thinspace
\chardef\_=`\_

\def\og#1{\leavevmode\smash{\vtop{% crude approximation of Polish ogonek
  \baselineskip0pt\lineskip0pt\lineskiplimit0pt
  \ialign{##\crcr\relax#1\cr
    \hidewidth\kern.2em
    \dimen0=.0040ex \multiply\dimen0\fontdimen1\font
    \kern-.0156\dimen0`\hidewidth\cr}}}}

\def\p #1.{\vskip 3pt plus 2pt minus 1pt
 \noindent\hbox to 20pt{\hss\bf#1. }\hangindent 20pt\ignorespaces}

\outer\def\finishpage{\par\vfill\eject}

\def\\{\par\hangindent 20pt\noindent\ignorespaces}

\outer\def\sectionbegin #1. #2{\vskip 0pt plus 400pt\penalty-100
 \vskip 12pt plus -394pt minus 4pt
 \noindent\tenssb\hbox to 0pt{\hss#1. }#2
 \par\penalty10000\vskip 3pt plus 2pt minus 1pt\rm}

\def\pat#1 {\par\noindent
 \hbox to 40pt{\hss #1\hskip 5pt}\hangindent 40pt\ignorespaces}

\def\yskip {\vskip 3pt plus 2pt minus 1pt}
\def\yyskip {\penalty-100\vskip 6pt plus 4pt minus 2pt}
\def\xskip {\hskip 7pt plus 3pt minus 4pt}

\def\star{\hbox to 0pt{\hss *}}

\hyphenation{Pas-a-dena}

\newdimen\unicodeptsize \unicodeptsize=10pt

\begingroup
\catcode`@=11 % we will define two private macro names

\gdef\Uni#1:#2:#3:#4:#5<#6>% ems:cols:rows:-hoff:rows+voff<hexbitmap>
 {\leavevmode \hbox to#1\unicodeptsize
    {\special{" 0 0 moveto currentpoint translate
                \unic@deptsize \unic@deptsize scale #2 #3 true
                [24 0 0 -24 #4 #5] {<#6>} imagemask}\hss}}

{\catcode`p=12\catcode`t=12\gdef\uni@ff#1pt{#1}}
\gdef\unic@deptsize{\expandafter\uni@ff\the\unicodeptsize\space}

\endgroup

\def\today{\ifcase\month\or
  January\or February\or March\or April\or May\or June\or
  July\or August\or September\or October\or November\or December\fi
  \space\number\day, \number\year}

%%%%%

\line{\twelvess CURRICULUM VIT\kern-1.5pt\AE\hfill\eightrm\today}
\vskip 10pt

\sectionbegin 1. {Biographical and Personal Information}
\\Donald E. Knuth, born January 10, 1938, Milwaukee, Wisconsin; U. S. citizen.
\\Chinese name 
\Uni1.08:24:24:-1:20% Unicode char "9ad8
<002000001800000806ffffff00000002004003ffe00300e00300c00300c003ffc00300c02000043ffffe30000e31008c31ffcc3181cc31818c31818c31ff8c31818c30007c300018>%
\thinspace \Uni1.08:24:24:-1:20% Unicode char "5fb7
<1c038018030018030631ffff30060067860446fffe86ccce0ccccc0ccccc18cccc18fffc38c00c38001878fffc58040098030818398618b18318b00b19b0081b300c1b3ffc181ff8>%
\thinspace \Uni1.08:24:24:-1:20% Unicode char "7eb3
<0601c00e01800c018018018018218231bfff61b187433186ff3186c631860c318618334630332663b6367e341660380600300600300603b0061e3006f03006c0300600303e00300c>%
\ (pronounced G\=ao D\'en\`a or Ko Tokuno).
\\Married to Nancy Jill Carter
[\Uni1.08:24:24:-1:20% Unicode char "9ad8
<002000001800000806ffffff00000002004003ffe00300e00300c00300c003ffc00300c02000043ffffe30000e31008c31ffcc3181cc31818c31818c31ff8c31818c30007c300018>%
\thinspace \Uni1.08:24:24:-1:20% Unicode char "7cbe
<1c038018030018031818fffc9b0300db03307a7ff87c03001a0306ffffff1800003840083c7ffc3e601c7b60185b7ff8986018986018187ff81860181860181860181860f8186030>%
\thinspace \Uni1.08:24:24:-1:20% Unicode char "862d
<018180018186ffffff018180418182ffe7ffe0e7077fc3fe60c3067fdbfe60db0667ffc660180662184663ffe6635ae6633cc663ffc6633cc6607b0660d9e66318466c181e60180c>%
] (b.\ July 15, 1939), June 24, 1961.
\\Children: John Martin (b.\ July 21, 1965),
 Jennifer Sierra (b.\ December 12, 1966).

\sectionbegin 2. {Academic History}
\\Case Institute of Technology, September 1956--June 1960; B.S., summa cum
laude, June, 1960; M.S. (by special vote of the faculty), June 1960.
\\California Institute of Technology, September 1960--June 1963; Ph.D.\ in
Math\-e\-ma\-tics, June 1963.
 Thesis: ``Finite Semifields and Projective Planes.''

\sectionbegin 3. {Employment Record}
\\Consultant, Burroughs Corp., Pasadena, California, 1960--1968.
\\Assistant Professor of Mathematics, California Institute of Technology,
1963--1966.
\\Associate Professor of Mathematics, California Institute of Technology,
1966--1968.
\\Professor of Computer Science, Stanford University, 1968--.
\\Staff Mathematician, Institute for Defense Analyses---Communications Research
Division, 1968--1969.
\\Guest Professor of Mathematics, University of Oslo, 1972--1973.
\\Professor of Electrical Engineering (by courtesy),
 Stanford University, 1977--.
\\Fletcher Jones Professor of Computer Science,
 Stanford University, 1977--1989.
\\Professor of The Art of Computer Programming,
 Stanford University, 1990--1992.
\\Professor of The Art of Computer Programming, Emeritus,
 Stanford University, 1993--.
\\Visiting Professor in Computer Science, University of Oxford, 2002--2006.

\sectionbegin 4. {Professional Societies}
\\American Guild of Organists, 1965--.
\\American Mathematical Society, 1961--.
\\\indent Committee on Composition Technology, 1978--1981.
\\Association for Computing Machinery, 1959--.
\\\indent Chairman, subcommittee on ALGOL, 1963--1964.
\\\indent General technical achievement awards subcommittee, 1975--1979.
\\\indent National Lecturer, 1966--1967.
\\\indent Visiting Scientist, 1966--1967.
\\Mathematical Association of America, 1959--.
\\Society for Industrial and Applied Mathematics, 1965--.

\sectionbegin 5. {Publications}
\\(see attached list)

\sectionbegin 6. {Patents}
\pat 3422405 (with R. E. Packard)\xskip Digital computers having an indirect
 field length operation. January 14, 1969.
\yskip
\pat 3454929 (with D. P. Hynes)\xskip Computer Edit System. July 8, 1969.
\yskip
\pat 3548174 Random number generator.  December 15, 1970.
\yskip
\pat 3626167 (with L. R. Guck, L. G. Hanson)\xskip Scaling and number base
 converting method and apparatus.  December 7, 1971.
\yskip
\pat 5305118 (with Stephen N. Schiller)\xskip Methods of controlling dot size
 in digital halftoning with multi-cell threshold arrays. April 19, 1994.
 European patent 96108227.8-2202, July 17, 1996.

\sectionbegin 7. {Principal Invited Lectures Given}
\\ACM National Convention, Syracuse, 1962.
\\NATO Summer School, Denmark, 1967.
\\Britannica Scholar, Chicago, 1968.
\\International Symposium on Teaching of Programming, Newcastle-Upon-Tyne,
 1970.
\\International Congress of Mathematicians, Nice, 1970.
\\IFIP Congress, Ljubljana, 1971.
\\International Congress on Logic, Methodology, and Philosophy of Science,
Bucharest, 1971.
\\Mathematical Association of America, San Francisco, 1974.
\\The Computer Science Lecture, Carnegie-Mellon University, 1974.
\\ACM National Convention, San Diego, 1974.
\\Symposium on Computational Systems, Monterrey, Mexico, 1975.
\\Chaire Aisenstadt, Montr\'eal, 1975.
\\American Association for the Advancement of Science, Boston, 1976; see
 paper P82 below.
\\Gibbs Lecture (American Mathematical Society), Atlanta, 1978; see
 paper P91 below.
\\Gillies Lectures, University of Illinois, 1979.
\\Hitchcock Professor, University of California, 1979.   
\\Ritt Lecturer, Columbia University, 1980.  
\\International Colloquium on Automata, Languages, and Programming,
 Epidaurus, Greece, 1985; see paper Q82 below.
\\4th SIAM Conference on Discrete Mathematics, San Francisco, 1988.
\\IFIP Congress, San Francisco, 1989 (keynote address); see paper P138 below.
\\Organick Memorial Lectures, University of Utah, 1990.
\\Donegall Lecturer in Mathematics, Trinity College, Dublin, 1992.
\\International Symposium on Teaching of Programming, Newcastle-Upon-Tyne
 (25th Anniversary Year), 1992.
\\Weizmann Memorial Lectures, Weizmann Institute of Science, 1992.
\\ACM--SIAM Symposium on Discrete Algorithms, Austin, 1993.
\\ATypI Congress, San Francisco, 1994.
\\Unicode Conference, San Jos\'e, 1995.
\\Commemorative lecture for Fiftieth Anniversary of Mathematisch Centrum,
 Amsterdam, 1996.
\\Commemorative lecture for Kyoto Prize, Kyoto, 1996.
\\SIAM Annual Meeting, Stanford, 1997.
\\God and Computer Lectures, MIT, 1999.
\\Pascal Lectures, University of Waterloo, 2000.
\\Strachey Lecture, Oxford University, 2001.

\sectionbegin 8. {Editorial Boards}
\\ACM Transactions on Algorithms, 2004--.
\\Acta Informatica, 1970--1979.
\\Advances in Mathematics, 1971--1979.
\\Applied Mathematics Letters, 1987--.
\\Communications of the ACM, 1966.
\\Combinatorica, 1985--.
\\Computers and Mathematics with Applications, 1973--1979.
\\Discrete Applied Mathematics, 1979.
\\Discrete and Computational Geometry, 1986--.
\\Discrete Mathematics, 1970--1978.
\\Electronic Journal of Combinatorics, 1994--.
\\Fibonacci Quarterly, 1964--1979.
\\Historia Mathematica, 1972--1979.
\\Human-Computer Interaction, 1985--1995.
\\IEEE Transactions on Software Engineering, 1975--1979.
\\Information Processing Letters, 1970--1979.
\\Japan Journal of Industrial and Applied Mathematics, 1997--.
\\Journal of Algorithms, 1979--2004.
\\Journal of Computer and Information Sciences, 1969--1979.
\\Journal of Computer and System Sciences, 1969--.
\\Journal of Computer Science and Technology, 1989--.
\\Journal of Experimental Algorithmics, 1996--.
\\Journal of Graph Algorithms and Applications, 1996--.
\\Journal of Graph Theory, 1975--1979.
\\Journal of the ACM, 1964--1967.
\\Journal on Statistical Planning and Inference, 1975--1979.
\\Mathematica Journal, 1990--.
\\The Mathematical Intelligencer, 1978--1979.
\\Random Structures \& Algorithms, 1990--.
\\SIAM Journal on Computing, 1973--1979.
\\Software --- Practice and Experience, 1979--.
\\Structured Programming, 1989--1993; Software Concepts and Tools, 1994--2000.
\\Theory of Computing, 2004--.
\\Utilitas Mathematica, 1970--1972.
\finishpage

\sectionbegin 9. {Honors and Awards}
\\Member, Pi Delta Epsilon, 1958--.
\\Member, Tau Beta Pi, 1958--.
\\Member, Blue Key, 1959--.
\\Case Honor Key, 1959.
\\Member, Sigma Xi, 1960--.
\\Woodrow Wilson Fellow, 1960.
\\National Science Foundation Fellow, 1960.
\\Grace Murray Hopper Award (first recipient), Association for
 Computing Ma\-chin\-ery, 1971. ($\$$1000)
\\Guggenheim Fellow, 1972--1973.
\\Fellow, American Academy of Arts and Sciences (Class I, Section 5), 1973--.
\\Alan M. Turing Award, Association for Computing Machinery, 1974.  ($\$$1000)
\\Member, National Academy of Sciences (Class III, Section 33), 1975--.
\\Lester R. Ford Award, Mathematical Association of America, 1975.  ($\$$100, 
 for paper P63.)
\\California Institute of Technology Distinguished Alumni Award, 1978.
\\National Medal of Science, 1979.
\\W. Wallace McDowell Award, IEEE Computer Society, 1980.  ($\$$1000)
\\Doctor of Science, honoris causa, Case Western Reserve University, 1980. % 1
\\Distinguished Fellow, The British Computer Society, 1980.
\\Priestley Award, Dickinson College, 1981. ($\$$1000)
\\Member, National Academy of Engineering, 1981--.
\\Honorary member, IEEE, 1982--.
\\IEEE Computer Pioneer Award (charter recipient), 1982.
\\Doctor of Science, honoris causa, Luther College, 1985. % 2
\\Doctor of Science, honoris causa, Lawrence University, 1985. % 3
\\Golden Plate Award, American Academy of Achievement, 1985.
\\ACM SIGCSE Award, 1986. (\$500)
\\Doctor of Science, honoris causa, Muhlenberg College, 1986. % 4
\\Doctor of Science, honoris causa, University of Pennsylvania, 1986. % 5
\\Docteur honoris causa, University of Paris--Sud (Orsay), 1986. % 6
\\ACM Software Systems Award, 1986.
\\Doctor of Science, honoris causa, University of Rochester, 1986. % 7
\\Steele Prize for Expository Writing, American Mathematical
 Society, 1986. (\$4000)
\\Doctor of Science, honoris causa, State University of New York at Stony
 Brook, 1987. % 8
\\The New York Academy of Sciences Award, 1987. (\$5000)
\\Benjamin Franklin Medal, Franklin Institute, Philadelphia, 1988.
\\Doctor of Science, honoris causa, Valparaiso University, 1988. % 9
\\Doctor of Science, honoris causa, University of Oxford, 1988. % 10
\\Doctor of Science, honoris causa, Brown University, 1988. % 11
\\Doctor of Science, honoris causa, Grinnell College, 1989. % 12
\\J. D. Warnier Prize, 1989. (\$3000)
\\Gold Medal Award, Case Alumni Association, 1990.
\\Doctor of Science, honoris causa, Dartmouth College, 1990. % 13
\\Doctor of Science, honoris causa, Concordia University, Montr\'eal,
 1991. % 14
\\Honorory Doctor of Technology, Royal Institute of Technology,
 Stockholm, 1991. % 15
\\Associ\'e \'Etranger, l'Acad\'emie des Sciences, Paris, 1992--.
\\Poch\"etny\u\i\ Doktor [{\rus Poche0tnyi0 Doktor}],
 Saint Petersburg University, 1992. % 16 (awarded in 1992, received in 1994)
\\Doctor of Science, honoris causa, Adelphi University, 1993. % 17
\\Utenlandsk medlem, Det Norske Videnskaps-Akademi, 1993--.
\\Lester R. Ford Award, Mathematical Association of America, 1993.  (\$250, 
 for paper P137.)
\\Docteur honoris causa, University of Marne-la-Vall\'ee, 1993. % 18
\\Best New Book: Computer Science, Association of American Publishers, 1994.
 [Awarded for {\sl The Stanford GraphBase}.]
\\ACM Fellow (charter recipient), 1994.
\\Adelsk\"old Medal, Royal Swedish Academy of Sciences, 1994.
\\IEEE John von Neumann Medal, 1995. (\$10,000)
\\Harvey Prize, Technion, 1995. (\$35,000)
\\Doctor Scienti{\ae} Mathematic{\ae}, honoris causa, Masaryk University,
 Brno, 1996. % 19
\\Memorial Medal, Mathematics and Physics Faculty, Charles University, Prague,
 1996.
\\Cum Deo Award (charter recipient), Milwaukee Lutheran High School, 1996.
\\Kyoto Prize for Advanced Technology, Inamori Foundation, 1996.
 (\setbox0=\hbox{Y}\rlap{\raise.3ex\hbox to\wd0{\hss\fiverm=\hss}}\box0
 50,000,000)
\\Doctor of Science, honoris causa, Duke University, 1998. % 20
\\Doctor of Science, honoris causa, St.\ Andrews University,
 Scotland, 1998. % 21
\\Korrespondierendes Mitglied der Mathematisch-naturwissenschaftlichen Klasse,
 Bayerische Akademie der Wissenschaften, 1998--.
\\Fellow of The Computer Museum, 1998--.
\\Doctor of Letters, honoris causa, University of Waterloo, Canada, 2000. % 22
\\Minor planet ``(21656) Knuth''
 [\kern.1em{\tt http:/\kern-.1em/%
 sunkl.asu.cas.cz/\char`~asteroid/planetky/21656/eng.htm}]
 named in 2001.
\\Epitimos Didaktor [{\grk Ep'itimos Did'aktwr}], % honorary doctor
 Athens University of Economics and Business, 2001. % 23
\\Doctor of Science, honoris causa, Eberhard Karls Universit\"at T\"ubingen,
 2001. % 24
\\Doctor Philosophi\ae\ honoris causa, Universitetet i Oslo, 2002. % 25
\\Doctor honoris causa in de Wetenschappen, Universiteit Antwerpen, 2003. % 26
\\Foreign member, Royal Society of London for
 Improving Natural Knowledge, 2003.
\\Doctor of Science, honoris causa, Harvard University, 2003. % 27
% "font of digital ingenuity, icon of algorithmic invention, whose
%  artful efforts have programmed the course of a powerful modern science"(!)
\\Epitimos Didaktor [{\grk Ep'itimos Did'aktwr}], % honorary doctor
 University of Macedonia, 2003. % 28
\\Doctor of Science, honoris causa, Universit\'e de Montr\'eal, 2004. % 29
\\Honorary Fellow, Magdalen College, Oxford University, 2005--.
\\Honorary Doctor, National Academy of Sciences,
 Republic of Armenia, 2005. % not counted with the academic doctorates
\\Doktor der Wissenschaften, honoris causa, Eidgen\"ossische Technische
 Hochschule Z\"urich, 2005. % 30
\\Honorary Doctor of Letters, Concordia University Wisconsin, 2006. % 31
\\Gold Commemorable Medal, % [sic]
 State Engineering University of Armenia, 2006.
\\Gold medal from Yerevan State University, 2006.

\sectionbegin 10. {Ph.D.\ Students\rm, thesis titles, and year of graduation}
\\Wayne Theodore Wilner, ``Declarative Semantic Definition,'' 1971.
\\Clark Allan Crane, ``Linear Lists and Priority Queues as
 Balanced Binary Trees,'' 1972.
\\Isu Fang, ``FOLDS, A Declarative Formal Language Definition System,'' 1972.
\\Michael Lawrence Fredman, ``Growth Properties of a Class of
 Recursively Defined Func\-tions,'' 1972.
\\Vaughan Ronald Pratt, ``Shellsort and Sorting Networks,'' 1972.
\\Richard Lee Sites, ``Proving that Computer Programs Terminate
 Cleanly,'' 1974.
\\Gary Don Knott, ``Deletion in Binary Storage Trees,'' 1975.
\\Edwin Hallowell Satterthwaite, Jr., ``Source Language Debugging
 Tools,'' 1975.
\\Robert Sedgewick, ``Quicksort,'' 1975.
\\Leonidas Ioannis Guibas, ``The Analysis of Hashing Algorithms,'' 1976.
\\Mark Robbin Brown, ``The Analysis of a Practical and Nearly Optimal
 Priority Queue,'' 1977.
\\Richard Eric Sweet (joint supervision with Cordell Green), ``Empirical
 Estimates of Program Entropy,'' 1977.
\\John Fredrick Reiser, ``Analysis of Additive Random Number
 Generators,'' 1977.
\\Bernard Marcel Mont-Reynaud, ``Hierarchical Properties of Flows, and the
 Determination of Inner Loops,'' 1977.
\\Luis Isidoro Trabb Pardo, ``Set Representation and Set Intersection,'' 1978.
\\Lyle Harold Ramshaw, ``Formalizing the Analysis of Algorithms,'' 1979.
\\Christopher John Van Wyk, ``A Language for Typesetting Graphics,'' 1980.
\\Jeffrey Scott Vitter, ``Analysis of Coalesced Hashing,'' 1980.
\\Michael Frederick Plass, ``Optimal Pagination Techniques for
 Automatic Typesetting Systems,'' 1981.
\\Ignacio Andres Zabala Salelles, ``Interacting with Graphic Objects,'' 1982.
\\Daniel Hill Greene, ``Labelled Formal Languages and Their Uses,'' 1983.
\\Franklin Mark Liang, ``Word Hy-phen-a-tion by Com-put-er,'' 1983.
\\Andrei Zary Broder, ``Weighted Random Mappings,'' 1985.
\\John Douglas Hobby, ``Digitized Brush Trajectories,'' 1985.
\\Scott Edward Kim, ``Viewpoint: Toward a Computer for Visual Thinkers,'' 1987.
\\Pang-Chieh Chen, ``Heuristic Sampling in Backtrack Trees,'' 1989.
\\Ramsey Wadi Haddad, ``Triangularization: A Two-Processor Scheduling
 Problem,'' 1990.
\\Tom\'as Feder, ``Stable networks and product graphs,'' 1991.

\sectionbegin 11. {Published biographical data \rm($\ast$
 means photograph included)}
\\``What's that about a score card? A computer's the thing,'' {\sl Newsweek\/
 \bf53},\,1 (January 5, 1959), 63.
\\\star{\sl IEEE Transactions on Electronic Computers \bf EC-13} (1964), 478.
\\{\sl Who's Who in Computers and Data Processing}, 1971.
\\{\sl American Men and Women of Science}, beginning with 12th edition (1972).
\\\star{\sl Datamation}, vol.~21, no.~1 (January 1975), 11--12.
\\\star{\sl IEEE Transactions on Software Engineering \bf SE-1} (1975), 3.
\\{\sl Who's Who in Computer Education and Research}, 1975.
\\{\sl bit\/ \bf7} (1975), 430--433, 444--447 (Japanese);
 written by Makoto Arisawa.
\\\star{\sl Stanford Daily}, Thursday, May 27, 1976.
\\{\sl Who's Who in America}, 40th edition (1978).
\\{\sl Dictionary of International Biography \bf 15}, 1979.
\\{\sl Leaders in Electronics}, McGraw Hill.
\\\star{\sl Men of Achievement}, 1979.
\\{\sl International Who's Who in Education}, 1980.   
\\{\sl Who's Who in Technology Today}, 2nd edition (1980).
\\{\sl China Computerworld}, no.\ 14 (July 20, 1981), p.\ 15;
 no.\ 15 (August 5, 1981), p.\ 15; no.\ 16 (August 20, 1981), p.\ 15.
\\\star{\sl Campus Report}, vol.\ 16, no.\ 17 (Stanford University,
 January 25, 1984), 5--6; vol.~16, no.~18 (February 1, 1984), 5,~8;
 written by Donald Stokes.
\\\star{\sl West\/} (San Jose Mercury News, February 19, 1984), 18--23;
 written by Jan C. Shaw.
\\\star{\sl Discover}, vol.\ 9, no.\ 5 (Time Inc., September 1984), 74--76,
 78; written by Bruce Schechter.
\\\star{\sl Computer Language}, vol.\ 1, no.\ 2 (October 1984), 17--19;
 written by Jan C. Shaw.
\\\star{\sl Cleveland}, vol.\ 15, no.\ 1 (January 1986), 106--109, 140--144;
 written by William Marling.
\\\star{\sl Portraits of Success}, by Carolyn Caddes (Portola Valley:
 Tioga Press, 1986), 78--79.
\\\star{\sl Notices of the American Math.~Society}, vol.~34, no.~2 (February
 1987), 228.
\\\star{\sl Portraits in Silicon}, by Robert Slater (Cambridge, Mass.:
 MIT Press, 1987), 341--351.
\\D. \'E. Knut i ego ``fabrika knig'', by B. B. Pokhodze\u\i, in the
 Russian translation of {\sl Mathematics for the Analysis of Algorithms\/}
 (see under Books), 114--115.
\\\star{\sl Communications of the ACM\/ \bf30} (1987), 816--819; written
 by Karen~A. Frenkel.
\\\star{\sl Peninsula}, vol.\ 3, no.\ 9 (December 1988), 72--74; written
 by Sherry Posnick-Goodwin.
  Japanese translation in {\sl Kunusu Sensei no Program-Ron\/} (see
  under Books), 168--174.
\\\star{\sl dialog Wissenschaft\/} (Nixdorf Computer AG, January 1989),
 6--15; interview by Norbert Ryska and Stuart~E. Savory.
\\Eulogy: Donald E. Knuth, by G. W. Bond, in Latin with an English paraphrase,
 {\sl Bulletin of the London Mathematical Society\/ \bf21} (1989), 110--112.
\\{\sl Japan Society for Software Science and Technology\/ \bf7},\,2 (April
 1990), 73--77; written by Makoto Arisawa.
\\\star{\sl Case Alumnus}, vol.\ 67, no.\ 8 (Spring/Summer 1990), cover and
 2--7; written by William Marling.
\\\star{\sl Byte}, vol.~15, no.~9 (September 1990), 282.
\\\star{\sl Macworld}, vol.~8, no.~7 (July 1991), 207.
\\\star{\sl The Rattle of Theta Chi}, vol.~73, no.~1 (Winter 1993), 10--11.
\\{\sl Information Science Dictionary\/} (Iwanami, Tokyo, 1990), 181.
\\\star{\sl Computers in Physics\/ \bf9} (1995), 248--249; interview by
 David~I. Lewin.
\\\star{\sl IEEE Spectrum\/ \bf32},\,6 (June 1995), 40.
\\\star{\sl Out Of Their Minds\/} by Dennis Shasha and Cathy Lazere
 (New York: Copernicus, 1995), 89--101.
\\{\sl CyberTimes\/} (10 August 1996), {\tt http:/\kern-.1em
 /www.nytimes.com/web/docsroot/library/cyber/week/0810knuth\kern-1.5pt\null}
 written by Steve Ditlea.
\\\star{\sl Lingua Franca\/ \bf6},\,6 (September 1996), 11--13; by
 Rick Perlstein.
\\\star{\sl Automatisering Gids\/} (15 Maart 1996), 9.
\\\star{\sl NRC Handelsblad\/} (28 Maart 1996); written by Dirk van Delft.
\\\star{\sl De Ingenieur\/} (3 Juli 1996), 27--29; written by H. M. Nieland.
\\\star{\sl Application Development Trends\/ \bf3},\,10 (October 1996),
 17--18; written by Elizabeth~U. Harding.
\\\star{\sl Computer Abstracts\/ \bf41},\thinspace3/4 (1997), 4--6.
\\{\sl Computer Software\/ \bf14},\,1 (January 1997), 83--86, by Makoto
 Arisawa [in Japanese].
\\{\sl Shi Jie Zhu Ming Ke Xue Jia Zhuan Ji (Biographies of World Famous
 Scientists)}, Technology Scientists I (Beijing: Science Press, 1997),
 117--125, by Dong Yunmei.
\\\star{\sl Wizards and Their Wonders}, by Christopher Morgan with photographs
 by Louis Fabian Bachrach (New York: ACM Press, 1997), 118--119.
\\{\sl Contemporary Authors\/ \bf163} (1998), 244--247.
% bio in San Jose Mercury-News, 29 Nov 1998, pages 1G, 14G, by Leigh Weimers
\\\star{\sl Technology Review\/ \bf102},\,5 (September/October 1999), 66--70, 
 by Steve Ditlea.
\\\star{\sl Salon.com $>$ Technology\/} (16 September 1999), by Mark Wallace,
 {\tt http:/\kern-.1em/www.salon.com/tech/feature/\allowbreak1999/09/16/knuth}.
\par\nobreak
\\{\sl The International Who's Who}, 64th edition, 2000.
\\\star{\sl NZZ Folio\/} (February 2002), 35--40, written by Peter Haffner.
\\{\sl Science \& Spirit\/ \bf13},\thinspace4 (July--August 2002), 13--14,
 by Laura Sivitz.
\\\star{\sl Stanford\/ \bf35},\thinspace3 (May--June 2006), 64--69,
 by Kara Platoni.
\\\star see also papers Q62 and Q154 listed below.
\finishpage

%%%%%
\centerline{\twelvess Publications of Donald E.\ Knuth}
\vskip 10pt
\sectionbegin 1. {Books}
\\{\sl The Art of Computer Programming}, Vol.\ 1: {\sl Fundamental Algorithms}
 (Reading, Mass.: Addison-Wesley, 1968), $\rm xxii + 634$~pp. Second printing,
 revised, July 1969.
\\Second edition, completely revised, December 1973. Second printing, revised,
 February 1975.
\\Third edition, completely revised, May 1997, $\rm xx+650$~pp.
\\Volume 1, Fascicle 1: {\sl {\sltt MMIX:} A RISC Computer for the New
 Millennium\/} (Upper Saddle River, N.J.: Addison-Wesley, 2005),
 $\rm v+134$~pp.
\\Romanian translation, by Adrian Davidoviciu, Adrian Petrescu,
 Smaranda Dimitriu, and Paul Zamfirescu,
{\sl Tratat de programarea calculatoarelor}, V.~1: {\sl Algoritmi fundamentali}
 (Bucharest: Editura tehnic\u a, 1974), 676 pp.
\\Romanian translation of the second edition,
 {\sl Arta program\u{a}rii calculatoarelor\/}, V.~1:
 {\sl Algoritmi fundamentali\/} (Bucharest: Editura Teora),
 616~pp.
\\Romanian translation of Volume 1, Fascicle 1, by Ioan Bledea:
 {\sl {\sltt MMIX:} Un  calculator RISC pentru noul mileniu\/}
 (Bucharest: Editura Teora, 2005), $\rm ix+149$~pp. 
\\Russian translation, by Galina P. Babenko and \t Iu.~M. Ba\t \i akovski\u\i,
 edited by K. I. Babenko, and V. S. \v Starkman, {\sl Iskusstvo
 programmirovani\t \i a dl\t \i a \'EVM}, 
 T. 1: {\sl Osnovnye algoritmy\/} (Moscow: Mir, 1976), 735~pp.
\\Russian translation of the third edition under direction of
 \t Iu.~V. Kozachenko, by S. G. Trigub, \t Iu.~G.~Gor\-dienko,
 and I. V. Krasikov, edited by S. N. Trigub,
 {\sl Iskusstvo programmirovani\t \i a}, 
 T. 1: {\sl Osnovnye algo\-ritmy\/} (Moscow: Vil'iams, 2000), 720~pp.
\\Japanese translation, under direction of Takakazu Simauti, in two volumes: 
 Chapter 1, by Ken Hirose, {\sl Kihon Samp\=o / Kiso Gainen}
 (Tokyo: Saiensu-Sha, 1978), $22 + 331$~pp.;
 Chapter 2, by Nobuo Yoneda and Katsuhiko Kakehi,
  {\sl Kihon Samp\=o / J\=oh\=o K\=oz\=o}
 (Tokyo: Saiensu-Sha, 1978), $8 + 373$~pp.
\\Japanese translation of the third edition, by
 Takashi Aoki, Kazuhiko Kakehi, Ken-Ichi Suzuki, and Takahiro Nagao, supervised
 by Makoto Arisawa and Eiiti Wada (Tokyo:\ ASCII Corporation, 2004), 
 $\rm xxii+632$~pp. 
\\Japanese translation of Volume 1, Fascicle 1, by Takashi Aoki, supervised
 by Makoto Arisawa and Eiiti Wada (Tokyo:\ ASCII Corporation, 2006), 
 $\rm vii+134$~pp. 
\\Chinese translation, by Guan JiWen and Su Yunlin, {\sl Ji Suan Ji Cheng Xu
 She Ji Ji Qiao}, 1.\ Juan: {\sl Ji Ben Suan Fa\/} (Beijing: Defense Industry
 Publishing Co., 1980), $14+573$~pp.
\\Chinese translation of the third edition, by Su YunLin,
 {\sl Jisuanji Chengxu Sheji Yishu}, 1.\ Juan: {\sl Jiben Suanfa\/}
 (Beijing: National Defense Industry Press, 2002), $\rm xx+625$~pp.
\\Spanish translation, by Michel Antscherl Harlange and
 Joan Lluis i Biset, under direction of Ram\'on Puig\-janer i Trepat,
{\sl El Arte de Programar Ordenadores}, V. 1: {\sl Algoritmos Fundamentales\/}
 (Barcelona: Revert\'e, 1980), $\rm xxiii + 672$~pp.
\\Indian Student Edition, with an introduction by P.~C.~P. Bhatt
 (New Delhi: Narosa Publishing House, 1985).
\\Hungarian translation, under direction of Mikl\'os Simonovits,
 {\sl A Sz\'am\'\i t\'og\'ep-Programoz\'as M\H uv\'eszete}, V.~1:
 {\sl Alapvet\H o Algoritmusok\/}
 (Budapest: M\H uszaki K\"onyvkiad\'o, 1987), 654~pp.
\\Polish translation of the third edition, by Grzegorz Jakacki,
 {\sl Sztuka Programowania}, Tom~1: {\sl Algorytmy Podstawowe\/}
 (Warsaw:\ Wydawnictwa Naukowo-Techniczne, 2002), $\rm xxiv+679$~pp.
\\Korean translation of the third edition, by Ryu Gwang (Seoul:\ Hanbit Media,
 2006), 793~pp.
\\Czech translation of the third edition (Prague: Fragment), in preparation.
\\German translation of the third edition (Heidelberg: Springer),
 in preparation.
\yyskip
\\{\sl The Art of Computer Programming}, Vol.\ 2: {\sl Seminumerical
 Algorithms} (Reading, Mass.: Addison-Wesley, 1969), $\rm xii + 624$~pp.
 Second printing, revised, November 1971.
\\Second edition, completely revised, January 1981, $\rm xiv + 689$~pp.
\\Third edition, completely revised, September 1997, $\rm xiv+762$~pp.
\\Russian translation, by Galina P. Babenko, \'E.~G. Belaga, and L. V.
 Ma\u \i orov, edited by K. I. Babenko, 
 {\sl Iskusstvo programmirovani\t \i a dl\t \i a \'EVM},
 T. 2: {\sl Poluchislennye algoritmy\/} (Moscow: Mir, 1977), 724 pp.
\\Russian translation of the third edition under direction of
 \t Iu.~V. Kozachenko, by L. F. Kozachenko,
 V. T. Tertyshny{\u\i}, and I. V. Krasikov,
 edited by S. N. Trigub, {\sl Iskusstvo programmirovani\t \i a}, 
 T. 2: {\sl Poluchislennye algoritmy\/} (Moscow: Vil'iams, 2000), 832~pp.
\\Japanese translation, under direction of Takakazu Simauti, in two volumes:
 Chapter 3, by Masaaki Sibuya, {\sl Jun Suchi Samp\=o / Rans\=u}
 (Tokyo: Saiensu-Sha, 1982), $\rm ii + 259$~pp.;
 Chapter 4, by Keisuke Nakagawa, {\sl Jun Suchi Samp\=o / Sanjutsu Enzan\/}
 (Tokyo: Saiensu-Sha, 1986), $\rm xii + 536$~pp.
\\Japanese translation of the third edition, by
 Hiroaki Saito, Takahiro Nagao, Shogo Matsui, Takao Matsui, and
 Hitoshi Yamaushi, supervised
 by Makoto Arisawa and Eiiti Wada (Tokyo:\ ASCII Corporation, 2004), 
 $\rm xvi+725$~pp. 
\\Romanian translation, by Florian Petrescu, Ioan Georgescu,
 Rolanda Predescu, and Paul Zamfirescu,
 {\sl Tratat de programarea calculatoarelor},
 V.~2: {\sl Algoritmi seminumerici}
 (Bucharest: Editura tehnic\u a, 1983), 722 pp.
\\Romanian translation of the second edition,
 {\sl Arta program\u{a}rii calculatoarelor\/}, V.~2:
 {\sl Algoritmi seminumerici\/} (Bucharest: Editura Teora),
 644~pp.
\\Spanish translation, in preparation (Barcelona: Revert\'e).
\\Chinese translation, by Guan JiWen and Su YunLin, {\sl Ji Suan Ji Cheng Xu
 She Ji Ji Qiao}, 2.\ Juan: {\sl Ban Shu Zhi Suan Fa\/}
 (Beijing: Defense Industry Publishing Co., 1992), $10+622$~pp.
\\Chinese translation of the third edition, by Su Yunlin,
 {\sl Jisuanji Chengxu Sheji Yishu}, 2.\ Juan: {\sl Ban Shuzhi Suanfa\/}
 (Beijing: National Defense Industry Press, 2002), $\rm xii+760$~pp.
\\Hungarian translation, under direction of Mikl\'os Simonovits,
 {\sl A Sz\'am\'\i t\'og\'ep-Programoz\'as M\H uv\'eszete}, V.~2:
 {\sl Szeminumerikus Algoritmusok\/}
 (Budapest: M\H uszaki K\"onyvkiad\'o, 1987), 690~pp.
\\Polish translation of the third edition, by Adam Malinowski,
 {\sl Sztuka Programowania}, Tom~2: {\sl Algorytmy Seminumeryczne\/}
 (Warsaw:\ Wydawnictwa Naukowo-Techniczne, 2002), $\rm xviii+820$~pp.
\\German translation of Chapter~4, by R\"udiger Loos, {\sl Arithmetik\/}
 (Heidelberg:\  Springer, 2001), $\rm xiii+538$~pp.
\\Korean translation of the third edition, by Ryu Gwang (Seoul:\ Hanbit Media),
 in preparation.
\yyskip
\\{\sl The Art of Computer Programming}, Vol.\ 3:  {\sl Sorting and Searching}
 (Reading, Mass.: Addison-Wesley, 1973), $\rm xii + 722$~pp.\
 + foldout illustration.
Second printing, revised, March 1975, $\rm xii + 725$~pp.
\\Second edition, completely revised, February 1998, $\rm xiv + 780$~pp.
\def\undercomma#1{$\baselineskip100pt\vtop{\halign{\hfil##\hfil\cr
#1\cr\noalign{\vskip-100pt\vskip.3ex}$\scriptscriptstyle,$\cr}}$}
\\Romanian translation, by Rodica Boconcios, A. Davidoviciu,
 P. Dimo, Fl.~Moraru, A. Petrescu, I. Sipo\undercomma{s}, and
 Smaranda Dimitriu,
 {\sl Tratat de programarea calculatoarelor},
 V.~3: {\sl Sortare \undercomma{s}i c\u autare\/}
 (Bucharest: Editura tehnic\u a, 1976), $\rm xii + 736$~pp.
\\Romanian translation of the second edition,
 {\sl Arta program\u{a}rii calculatoarelor\/}, V.~3:
 {\sl Sortare \undercomma{s}i c\u{a}utare\/} (Bucharest: Editura Teora),
 680~pp.
\\Russian translation, by Nadezhda I. V'\t \i ukova, V. A. Galatenko, and
 A. B. Khodulev, edited by \t Iu.~M. Ba\t \i akovski\u\i\ and V. S.
 \v Starkman, {\sl Iskusstvo programmirovani\t\i a dl\t\i a \'EVM},
 T. 3:  {\sl Sortirovka i poisk\/} (Moscow:  Mir, 1978), 844~pp.
\\Russian translation of the second edition under direction of
 \t Iu.~V. Kozachenko, by V. T. Tertyshny{\u\i} and
 I.~V. Krasikov,  edited by S. N. Trigub,
 {\sl Iskusstvo  programmirovani\t \i a}, 
 T. 3: {\sl Sortirovka i poisk\/} (Moscow: Vil'iams, 2000), 824~pp.
\\Japanese translation of the second edition, by
 Yuichiro Ishii, Hiroshi Ichiji, Hiroshi Koide,
 Eiko Takaoka, Kumiko Tanaka, and Takahiro Nagao,
 supervised by Makoto Arisawa and Eiiti Wada
 (Tokyo:\ ASCII Corporation, 2006), $\rm xvi+741$~pp. 
\\Chinese translation, by Guan JiWen and Su YunLin, {\sl Ji Suan Ji Cheng Xu
 She Ji Ji Qiao}, 3.\ Juan: {\sl Pai Xu He Cha Zhao\/} (Beijing:
 Defense Industry Publishing Co., 1985), $\rm viii+645$~pp.
\\Chinese translation of the second edition, by Su Yunlin,
 {\sl Jisuanji Chengxu Sheji Yishu}, 3.\ Juan: {\sl Paixu Yu Chazhao\/}
 (Beijing: National Defense Industry Press, 2002), $\rm x+779$~pp.
\\Spanish translation, by Jaime de Argila y de Chopitea and
 Ram\'on Puigjaner Trepat, under direction of Ram\'on Puig\-janer Trepat,
 {\sl El Arte de Programar Ordenadores}, V. 3: {\sl Clasificaci\'on
 y B\'usqueda\/} (Barcelona: Revert\'e, 1980), $\rm xxiii + 672$~pp.
\\Hungarian translation, under direction of Mikl\'os Simonovits,
 {\sl A Sz\'am\'\i t\'og\'ep-Programoz\'as M\H uv\'eszete}, V.~3:
 {\sl Keres\'es \'es Rendez\'es\/}
 (Budapest: M\H uszaki K\"onyvkiad\'o, 1988), 761~pp.
\\Polish translation of the second edition, by Krzysztof Diks and Adam
 Malinowski,
 {\sl Sztuka Programowania}, Tom~3: {\sl Sortowanie i Wyszukiwanie\/}
 (Warsaw:\ Wydawnictwa Naukowo-Techniczne, 2002), $\rm xviii+838$~pp.
\\Korean translation of the second edition, by Ryu Gwang (Seoul:\ Hanbit Media),
 in preparation.
\yyskip
\\{\sl The Art of Computer Programming}, Vol.\ 4:
 {\sl Combinatorial Algorithms}
 (Reading, Mass.: Addison-Wesley), in preparation.
\\Volume 4, Fascicle 2: {\sl Generating All Tuples and Permutations\/}
 (Upper Saddle River, N.J.: Addison-Wesley, 2005), $\rm v+127$~pp.
\\Volume 4, Fascicle 3: {\sl Generating All Combinations and Partitions\/}
 (Upper Saddle River, N.J.: Addison-Wesley, 2005), $\rm v+150$~pp.
\\Volume 4, Fascicle 4: {\sl Generating All Trees; History of Combinatorial
 Generation\/} (Upper Saddle River, N.J.: Addison-Wesley, 2006),
 $\rm vi+120$~pp.
\\Romanian translation of Volume 4, Fascicle 2, by Cora Radulian:
 {\sl Generarea tuturor tuplurilor s\llap{\lower.1ex\hbox{,}}i
 permut\u{a}rilor\/} (Bucharest: Editura Teora, 2005), $\rm vii+144$~pp. 
\\Japanese translation of Volume 4, Fascicle 2, by Hiroshi Koide, supervised
 by Makoto Arisawa and Eiiti Wada (Tokyo:\ ASCII Corporation, 2006), 
 $\rm viii+129$~pp. 
\yyskip
\\{\sl MIX} (Addison-Wesley, 1971), 48 pp.
\yyskip
\\{\sl Surreal Numbers} (Reading, Mass.: Addison-Wesley, 1974), $\rm vi
 + 119$~pp.
\\Czech translation, by Helena Ne\v set\v rilov\'a, {\sl Nadre\'aln\'a
 \v c\' \i sla}, in {\sl Pokroky Matematiky, Fyziky a Astronomie \bf 23}
 (1978), 66--76, 130--139, 187--196, 246--261.
\\German translation, by Brigitte and Karl Kunisch, {\sl Insel der Zahlen\/}
 (Braunschweig: Friedr.\ Vieweg \&\ Sohn, 1979), 124 pp.
\\Japanese translation by Junji Koda, {\sl Chogen Jis Su\/}
 (Tokyo: Kaimei Sha Ltd., 1978), 179 pp. 
\\Another Japanese translation by Junji Koda, {\sl Chogen Jis Su\/},
 published in eight monthly installments
 in {\sl Basic S\=ugaku}, August 1978 through March 1979.)
\\Japanese translation by Shunsuke Matsuura, {\sl Shifuku no Cogen Jis Su\/},
 illustrated by Yusuke Saito (Tokyo: Kashiwa Shobo, 2005), 174~pp.
\\Spanish translation, by Lluc Garriga, {\sl N\'umeros Surreales\/} 
 (Barcelona: Revert\'e, 1979), 101 pp.
\\Hungarian translation, by J\'anos Vir\'agh and Zolt\'an \'Esik,
 {\sl Sz\'amok val\'oson innen \'es t\'ul\/} (Budapest: Gondolat, 1987),
 $136+\rm ii$~pp.
\\Portuguese translation, by Jorge Nuno Silva, {\sl N\'umeros Surreais\/}
 (Lisbon: Gradiva, 2002), 113~pp.
\yyskip
\\{\sl Mariages Stables et leurs relations avec d'autres probl\`emes
 combinatoires\/} (Montr\'eal: Les Presses de l'Uni\-ver\-sit\'e de Montr\'eal,
 1976), 106 pp.  \'Edition revue et corrig\'ee, 1981. Currently
 available from Les Publications CRM / Centre de Recherches Math\'ematiques,
 Universit\'e de Montr\'eal, Montr\'eal, Quebec.
\\English translation, by Martin Goldstein, {\sl Stable Marriage and its
 Relation to Other Combinatorial Problems\/} (Providence, R.~I.: American
 Mathematical Society, 1997), $\rm xiii+74$~pp. (CRM Proceedings \& Lecture
 Notes, Volume~10.)
\yyskip
\\{\sl \TeX\ and \slMF:  New Directions in Typesetting}
 (Providence, R.I.:  American Mathematical Society,
 and Bedford, Mass.:  Digital Press, 1979), $\rm xi + 45 + 201 + 105$~pp.
\yyskip
\\(with Daniel H. Greene)\xskip
 {\sl Mathematics for the Analysis of Algorithms\/}
 (Cambridge, Mass.: Birkh\"auser Boston, 1981), 107 pp.
 Second edition, 1982, 123 pp.
 Third edition, 1990, $\rm viii+132$ pp.
\\Russian translation of the second edition by B. B. Pokhodze\u\i, edited by
 Yuri~V. Mati\t\i asevich, {\sl Matematicheskie metody analiza algoritmov\/}
 (Moscow: Mir, 1987), 120~pp.
\\Japanese translation, in preparation (Kindai Kagaku Sha).
\yyskip
\\{\sl The \TeX book\/} (Reading, Mass.: Addison-Wesley, 1984), $\rm x+483$~pp.
 Second printing, revised, October 1984. Sixth printing, revised,
 January 1986; also published as {\sl Computers \& Typesetting}, Vol.~A.
 Twenty-sixth printing, 1996, contains final revisions.
\\Japanese translation, by Yoshiteru Sagiya and Nobuo Saito,
 {\sl \TeX bukku, konpyuuta ni yoru sohan sisutemu\/}
 (Tokyo: ASCII Corporation, 1989), $\rm xix+657$~pp.
\\Russian translation, by M. V. Lisina, edited by S. V. Klimenko and
 S. N. Sokolov, {\sl Vse pro \TeX\/}
 (Protvino, Moscow: AO RD\TeX, 1993), $\rm xvi+575$~pp.
% AO means Inc; RD\TeX stands for Russian Database \TeX nology
\\Russian translation, by L. F. Kozachenko, edited by Yu.~V. Kozachenko,
 {\sl Vse pro \TeX\/} (Moscow: Vil'iams, 2003), 549~pp.
\\Polish translation, by Piotr Bolek, W{\l}odzimierz Bzyl, and Adam Dawidziuk,
 {\sl \TeX: Przewodnik u\.zytkownika\/} (Warsaw:\ Wydawnictwa
 Naukowo-Techniczne, 2005), $\rm xviii+541$~pp.
\\French translation, by Jean-C\^ome Charpentier, {\sl Le \TeX book:
 Composition informatique\/} (Paris:\ Vuibert Informatique, 2003), 576~pp.
\\Polish translation (Warsaw:\ Wydawnictwa
 Naukowo-Techniczne), in preparation.
\yyskip
\\{\sl Computers \& Typesetting}, Vol.~B: {\sl \TeX:\ The Program\/}
 (Reading, Mass.: Addison-Wesley, 1986), $\rm xvi + 594$~pp.
 Fifth printing, $\rm xviii + 600$~pp., 1994.
\yyskip
\\{\sl The \slMF\kern1ptbook\/} (Reading, Mass.: Addison-Wesley, 1986),
 $\rm xii+361$~pp.; also published as {\sl Computers \& Typesetting}, Vol.~C.
\\Japanese translation, by Yoshiteru Sagiya, {\slMF\ \sl bukku}
 (Tokyo: ASCII Corporation, 1994), $\rm xvi+451$~pp.
\\Russian translation, by L. F. Kozachenko, edited by Yu.~V. Kozachenko,
 {\sl Vse pro \TeX\/} (Moscow: Vil'iams, 2003), 549~pp.
\\Russian translation, {\sl Vse pro \slMF\/} (Moscow: Vil'iams),
 in preparation.
\yyskip
\\{\sl Computers \& Typesetting}, Vol.~D: {\sl \slMF:\ The Program\/}
 (Reading, Mass.: Addison-Wesley, 1986), $\rm xvi+560$~pp.
 Third printing, $\rm xviii+566$ pp., 1991.
\yyskip
\\{\sl Computers \& Typesetting}, Vol.~E: {\sl Computer Modern Typefaces\/}
 (Reading, Mass.: Addison-Wesley, 1986), $\rm xvi+588$~pp.
\yyskip
\\(with Ronald L. Graham and Oren Patashnik)\xskip {\sl Concrete Mathematics\/}
 (Reading, Mass.: Addison-Wesley, 1989), $\rm xiii+625$~pp.
% Second printing, revised, January 1989.
% Third printing, revised, May 1989.
% Fourth printing, revised, January 1990.
% Fifth printing, revised, July 1990.
% Sixth printing, revised, October 1990.
% Seventh printing, revised, December 1991.
% Eighth printing, revised, October 1992.
 Second Edition, January 1994, $\rm xiii+657$~pp.
% Second printing, revised, February 1995.
\\Russian translation by A. B. Khodulev and B. B. Pokhodze\u\i,
 {\sl Konkretna\t\i a  matematika\/} (Moscow: Mir, 1999), 704~pp.
\\Chinese translation by Lai FeiPei, {\sl Ju Ti Shu Xue\/}
 (Taipei: Dong Hua Publishing Co., 1990), $\rm xv+731$~pp.
\\Chinese translation by Chen YanWen, {\sl Ju Ti Shu Xue\/} (Taipei: Ru Lin
 Publishing Co., 1991), $\rm xii+695$~pp.
\\Chinese translation by Zhuang Xingu, {\sl Ju Ti Shu Xue\/} (Xi'an:
  Xi An Dian Zhi Ke Ji Da Xue Chu Ban She, 1992), $\rm xii+539$~pp.
\\Chinese translation (Beijing: China Machine Press), in preparation.
\\Italian translation edited by Giovanni Monegato, {\sl Matematica Discreta\/}
 (Milan: Editore Ulrico Hoepli, 1992), $\rm xviii+607$ pp.
\\Japanese translation, by Makoto Arisawa, Michiaki Yasumura, Tatsuya Hagino,
 and Kiyoshi Ishihata, {\sl Kompyuta no S\=ugaku\/}
 (Tokyo: Kyoritsu-Shuppan, 1993), $\rm xvi+606$ pp.
\\Portuguese translation, by Val\'eria de Magalh\~aes Iorio,
 {\sl Matem\'atica Concreta\/}  (Rio de Janeiro: Livros T\'ecnicos e
 Cient{\'\i}ficos Editora, 1995), $\rm xii+477$ pp.
\\Polish translation, by Piotr Chrz\og{a}stowski, A.~Czumaj, L.~G\og{a}sieniec, and
 M.~Raczunas, {\sl Matematyka Kon\-kretna\/}
 (Warszawa: Polskie Wydawnictwa Nauk\-owe, 1996), 718 pp.
\\Hungarian translation, by S\'andor Fridli, J\'anos Gonda, Attila Kov\'acs,
 L\'aszl\'o Lakatos, and Csab\'an\'e L\'ang, {\sl Konkr\'et Matematika\/}
 (M\H{u}szaki K\"onyvkiad\'o, 1998), $\rm xvi+647$~pp.
\\Spanish translation (Addison-Wesley Spain and Universidad Autonoma de
 Madrid), in preparation.
\\French translation, by Alain Denise, {\sl Math\'ematiques concr\`etes\/}
(Paris:\ International Thompson Publishing, 1998; later acquired by
  Vuibert Informatique), $\rm xiv+688$~pp.
\\Croatian translation (Zagreb: Golden Marketing), in preparation.

\yyskip
\\(with Tracy L. Larrabee and Paul M. Roberts)\xskip {\sl Mathematical
 Writing\/} (Washington, D.C.: Mathematical Association of America, 1989),
 $\rm ii+115$~pp.
\\Japanese translation, by Makoto Arisawa, with additional illustrations
 and notes by the translator, {\sl Kunusu Sensei no Dokyumento Sampo\/}
 (Tokyo: Kyoritsu-Shuppan, 1989), $\rm x+194$~pp.
\yyskip
\\{\sl 3:16 Bible Texts Illuminated\/} (Madison, Wisconsin: A-R Editions,
 1990), $\rm iii+268$~pp.
 Second printing, revised, January 1992.
\yyskip
\\{\sl Kunusu Sensei no Program-Ron\/} [Professor Knuth's Programming
 Discipline], anthology edited by Makoto Arisawa (Tokyo: Kyoritsu-Shuppan,
 1991), $\rm v+199$~pp.
\yyskip
\\{\sl Literate Programming\/} (Stanford, California:
 Center for the Study of Language and Information, 1992), $\rm xvi+368$~pp.
 (CSLI Lecture Notes, no.~27.)
\\Japanese translation, by Makoto Arisawa, {\sl Bungeiteki Programming\/}
 (Tokyo: ASCII Corporation, 1994), 463~pp.
\yyskip
\\{\sl Axioms and Hulls\/} (Heidelberg: Springer-Verlag, 1992),
 $\rm ix+109$~pp. (Lecture Notes in Computer Science, no.~606.)
%\\Chinese translation, by Su Yunlin (Jiang Xi Science and Technology Press),
% in preparation.
\yyskip
\\{\sl The Stanford GraphBase\/}: A Platform for Combinatorial Computing.
 (New York: ACM Press, 1993), $\rm viii+576$~pp.
\yyskip
\\(with Silvio Levy)\xskip {\sl The CWEB System of Structured Documentation}.
 (Reading, Massachusetts: Addison-Wesley, 1993), $\rm iv+227$~pp.
\\Version 3.6, with hypertext support (Reading, Massachusetts: Addison-Wesley,
 2001), $\rm ii+237$~pp.
\yyskip
\\{\sl Selected Papers on Computer Science\/} (Stanford, California:
 Center for the Study of Language and Information, 1996), $\rm xii+274$~pp.
 (CSLI Lecture Notes, no.~59.) Co-published with Cambridge University Press.
\yyskip
\\{\sl Digital Typography\/} (Stanford, California:
 Center for the Study of Language and Information, 1999), $\rm xvi+685$~pp.
 (CSLI Lecture Notes, no.~78.)
\\Russian translation by Roman Kuznets, Olga Makhovaia, Nikolai Tretiakov,
 and Yurii Tiumentsev, edited by Irina Makhovaia,
 {\sl Comp'{\t\i}uterna{\t\i}a Tipografi{\t\i}a\/} (Moscow: Mir, 2003), 669~pp.
\yyskip
\\{\sl MMIXware\/} (Heidelberg: Springer-Verlag, 1999),
 $\rm vii+550$~pp. (Lecture Notes in Computer Science, no.~1750.)\par
\yyskip
\\Japanese translation, by Takizawa Toru, (SiB\thinspace access Co., 2002),
 $\rm viii+550$~pp.
\yyskip
\\{\sl Selected Papers on Analysis of Algorithms\/} (Stanford, California:
 Center for the Study of Language and Information, 2000), $\rm xvi+621$~pp.
 (CSLI Lecture Notes, no.~102.) Printings made after 2006 have
 $\rm xvi+622$~pp., because the index has gotten longer.
\yyskip
\\{\sl Things a Computer Scientist Rarely Talks About\/} (Stanford, California:
 Center for the Study of Language and Information, 2001), $\rm xi+257$~pp.
 (CSLI Lecture Notes, no.~136.)
\\Japanese translation by Tooru Takizawa, Yuko Makino, and Noboru Tomizawa,
 {\sl Computer kagakusha ga mettan\={\i} Katarana\={\i} Koto\/}
 (Tokyo: SiBaccess Co., Ltd., 2003), $\rm x+260$~pp.
\yyskip
\\{\sl Selected Papers on Computer Languages\/} (Stanford, California:
 Center for the Study of Language and Information, 2003), $\rm xvi+594$~pp.
\yyskip
\\{\sl Selected Papers on Discrete Mathematics\/} (Stanford, California:
 Center for the Study of Language and Information, 2003), $\rm xvi+812$~pp.
\yyskip
\\{\sl Selected Papers on Design of Algorithms\/} (Stanford, California:
 Center for the Study of Language and Information), in preparation.
\yyskip
\\{\sl Selected Papers on Fun and Games\/} (Stanford, California:
 Center for the Study of Language and Information), in preparation.

\vskip 20pt
\sectionbegin 2. {Videos and Audiotapes}
\\{\sl Problem Solving with Donald Knuth}. {\sl The Stanford Video Journal},
 Vol.~1 (Stanford, California: Stanford Instructional Television Network,
 1985). Twenty 75-minute tapes.
\\``The Literate Mathematician,'' in BBC Radio~5's series {\sl Maths
 Miscellany}, produced by John Jaworski and Giselle Corbett
 (first broadcast February 21, 1993), 30~min. Cassettes available from
 BBC OUPC, Walton Hall, Milton Keynes, MK7 6BH, England.
\\``Computer Musings: The associative law, or The anatomy of
 rotations in binary trees,'' {\sl The Distinguished Lecture Series\/ \bf7}
 (Stanford, CA: University Video Communications, 1993), 68~min.
\\``Graph drawing from a user's perspective,'' opening lecture at the
 conference on Graph Drawing at MSRI, Berkeley, 18 September 1996.
 {\tt http:/\kern-.1em/www.msri.org/publications/ln/msri/1996/graph/knuth/1/}
\\``Some research problems for combinatorialists,'' lecture
 at MSRI, Berkeley, 27 January 1997.
 {\tt http:/\kern-.1em/www.\allowbreak msri.org/publications/ln/msri/1997/%
 non\char`\_workshop/knuth/1/}
\\{\sl Donald Knuth: God and Computers}, seven public lectures delivered at MIT
 in the fall of 1999. {\tt http:/\kern-.1em
 /\allowbreak technetcast.ddj.com/tnc\_program.html?program\_id=50}
\\{\sl Donald Knuth: MMIX, A RISC Computer for the New Millennium}, a
 lecture to the Boston chapter of ACM on 15 December 1999.
 {\tt http:/\kern-.1em/technetcast.ddj.com/\allowbreak
 tnc\_play\_stream.html?stream\_id=199}
\\{\sl All Questions Answered}, talk at Oslo Instute for Informatics,
 30 August 2002.
 {\tt http:/\kern-.1em/www.ifi.uio.no/\allowbreak foredrag/knuth-2002.html}
\\{\sl Bottom-Up Education; or, The Search for Intelligent Life
 in the University}, keynote speech at ITiCSE 2003, Thessaloniki, 2 July 2003.
 {\tt http:/\kern-.1em/iticse2003.uom.gr/}.
\\``Donald Knuth: Founding Artist of Computer Science,'' profile/interview
 on National Public Radio by David Kestenbaum,
 {\sl Morning Edition\/} (14 March 2005).
 {\tt http:/\kern-.1em/www.npr.org/templates/story/story.php?\allowbreak
 storyId=4532247}.

\finishpage

%%%%%
\sectionbegin 3. {Papers \rm($\ast$ means written by coauthor)}
\p  P1.  The potrzebie system of weights and measures.  {\sl MAD Magazine\/
 \bf 33} (June 1957), 36--37.  (Illustrated by Wallace Wood.)  Reprinted 
 in {\sl Like, MAD\/} (New York:  Signet Pocket Books No.~S1838, 1960), 
 139--145. Page 36 reprinted in {\sl Completely MAD\/} by Maria Reidelbach
 (Boston, Mass.: Little, Brown, 1991), 191.
\p P2.  RUNCIBLE --- Algebraic translation on a limited computer.
 {\sl Commun\-i\-ca\-tions of the ACM\/ \bf 2},\,11 (November 1959), 18--21.  
 Reprinted with amendments as Chapter~21 of {\sl Selected Papers on
 Computer Languages\/} (see under Books).
\p P3.  An imaginary number system.  {\sl Communications of the
 ACM\/ \bf 3} (April 1960), 245--247.  Errata, {\sl Communications of the 
 ACM\/ \bf 4} (August 1961), 355.
 Reprinted as Chapter~18 of {\sl Selected Papers on
 Discrete Mathematics\/} (see under Books).
\p *P4.  (with R. C. Bose, I. M. Chakravarti)\xskip  On methods of constructing
 sets of mutually orthogonal latin squares using a computer.  Part I:
 {\sl Technometrics\/ \bf 2} (1960), 507--516.  Part II:  {\sl Technometrics\/
 \bf 3} (1961), 111--117.  
\p P5.  Minimizing drum latency time.  {\sl Journal of the 
 ACM\/ \bf 8} (April 1961), 119--150.  
\p P6.  (with Jack N. Merner)\xskip  ALGOL 60 {\it Confidential\/}.
 {\sl Commun\-i\-ca\-tions of the ACM\/ \bf 4} (June 1961), 268--272.  
 Reprinted as Chapter~4 of {\sl Selected Papers on
 Computer Languages\/} (see under Books).
\p P7.  (with G. A. Bachelor, J. R. H. Dempster, J. Speroni)\xskip  SMALGOL-61.
 {\sl Commun\-i\-ca\-tions of the ACM\/ \bf 4} (November 1961), 499--502.  
 Reprinted as Chapter~5 of {\sl Selected Papers on
 Computer Languages\/} (see under Books).
\p P8.  Euler's constant to 1271 places.
 {\sl Math\-e\-ma\-tics of Compu\-ta\-tion\/ \bf 16} (1962), 275--281.  
\p P9.  Evaluation of polynomials by computer. 
 {\sl Commun\-i\-ca\-tions of the ACM\/ \bf 5} (December 1962), 595--599.  
\p P10.  A history of writing compilers.  {\sl Computers and Automation\/
 \bf 11},\,12 (December 1962), 8--18.  Reprinted in {\sl Compiler Techniques},
 Bary W. Pollack, ed., (Prince\-ton:  Auerbach, 1972), 38--56.  % Bary: sic
 Reprinted as Chapter~20 of {\sl Selected Papers on
 Computer Languages\/} (see under Books).
\p P11.  Computer-drawn flowcharts.  {\sl Commun\-i\-ca\-tions of the
 ACM\/ \bf 6} (September 1963), 555--563.
 Reprinted as Chapter~22 of {\sl Selected Papers on
 Computer Languages\/} (see under Books).
\p P12.  Length of strings for a merge sort.  {\sl Commun\-i\-ca\-tions of the
 ACM\/ \bf 6} (November 1963), 685--688.  
 Reprinted with additional material as Chapter~14 of {\sl Selected Papers
 on Analysis of Algorithms\/} (see under Books).
\p P13.  Transcendental numbers based on the
 Fibonacci sequence.  {\sl Fibonacci Quarterly\/ \bf 2} (1964), 43--44. 
\p P14.  Billiard balls in an equilateral triangle.  {\sl Recreational
 Mathematics Magazine\/ \bf 14} (January--February 1964), 20--23. 
\p P15.  (with L.~L. Bumgarner, D.~E. Hamilton, P.~Z. Ingerman, M.~P. Lietzke,
 J.~N. Merner, D.~T. Ross)\xskip  A proposal for 
 input-output conventions in ALGOL 60.  {\sl Commun\-i\-ca\-tions of
 the ACM\/ \bf 7} (May 1964), 273--283.  Russian translation by M. I. Ageev
 in {\sl Sovremennoe Programmirovanie\/ \bf1} (Moscow: Soviet Radio, 1966),
 73--107.
 Reprinted as Chapter~7 of {\sl Selected Papers on
 Computer Languages\/} (see under Books).
\p P16.  (with J. L. McNeley)\xskip  SOL --- A symbolic language for
 general-purpose systems simulation. 
 {\sl IEEE Transactions on Electronic Computers\/
 \bf EC-13} (1964), 401--408.
 Reprinted as Chapter~9 of {\sl Selected Papers on
 Computer Languages\/} (see under Books).
\p P17.  (with J. L. McNeley)\xskip  A formal definition of SOL\null.
 {\sl IEEE Transactions on Electronic Computers\/ \bf EC-13} (1964), 409--414.
 Reprinted as Chapter~10 of {\sl Selected Papers on
 Computer Languages\/} (see under Books).
\p P18.  Representing numbers using only one 4.  {\sl Mathematics Magazine\/
 \bf 37} (1964), 308--310.  
\p P19.  (with Marshall Hall, Jr.)\xskip  Combinatorial analysis
 and computers.  {\sl American Mathematical Monthly\/ \bf72}, part~2,
 {\sl Computers and Computing}, Slaught Memorial Papers No.~10 (February 1965),
 21--28.
 Reprinted as Chapter~1 of {\sl Selected Papers on
 Discrete Mathematics\/} (see under Books).
\p P20.  (with J. D. Alanen)\xskip  Tables of finite fields. 
 {\sl Sankhy\=a}, series A, {\bf 26} (1964), 305--328.  
 Reprinted as Chapter~19 of {\sl Selected Papers on
 Discrete Mathematics\/} (see under Books).
\p P21.  Finite semifields and projective planes.  {\sl Journal of Algebra\/
 \bf 2} (1965), 182--217. Reprinted in {\sl Neofields and Combinatorial
 Designs}, ed.~by D. Frank Hsu, {\sl Advances in Discrete Mathematics and
 Computer Science\/ \bf 1} (Nonantum, Massachusetts: Hadronic Press, 1984),
 57--92.
 Reprinted as Chapter~20 of {\sl Selected Papers on
 Discrete Mathematics\/} (see under Books).
\p P22.  A class of projective planes.  {\sl Transactions of the American
 Mathematical Society\/ \bf 115} (1965), 541--549.  
 Reprinted as Chapter~21 of {\sl Selected Papers on
 Discrete Mathematics\/} (see under Books).
\p P23.  On the translation of languages from left to right.
 {\sl Information and Control\/ \bf 8} (1965), 607--639.
 Russian translation by A. A. Muchnik in
 {\sl \t Iazyki i Avtomaty}, ed.\ by A. N. Maslov and \'E. D. Stotski\u\i\
 (Moscow: Mir, 1975), 9--42.
 Reprinted in {\sl Great Papers in Computer Science}, edited by Phillip
 Laplante (St.~Paul, Minnesota: West Publishing, 1996), 150--173.
 Reprinted as Chapter~15 of {\sl Selected Papers on
 Computer Languages\/} (see under Books).
\p P24.  Construction of a random sequence.  {\sl BIT\/ \bf 5} (1965),
 246--250.  
 Reprinted as Chapter~17 of {\sl Selected Papers on
 Discrete Mathematics\/} (see under Books).
\p P25.  An almost linear recurrence.  {\sl Fibonacci Quarterly\/ \bf 4}
 (1966), 117--128.  
 Reprinted as Chapter~37 of {\sl Selected Papers on
 Discrete Mathematics\/} (see under Books).
\p P26.  Oriented subtrees of an arc digraph.  {\sl Journal of
 Combinatorial Theory\/ \bf 3} (1967), 309--314. 
 Reprinted as Chapter~12 of {\sl Selected Papers on
 Discrete Mathematics\/} (see under Books).
\p P27.  (with Thomas J. Buckholtz)\xskip  Computation of tangent, Euler, and
 Bernoulli numbers.
 {\sl Mathematics of Computation\/ \bf 21} (1967), 663--688.
\p P28.  (with Richard H. Bigelow)\xskip  Programming languages for automata.
 {\sl Journal of the ACM\/ \bf 14} (October 1967), 615--635.  
 Reprinted as Chapter~12 of {\sl Selected Papers on
 Computer Languages\/} (see under Books).
\p P29.  The remaining trouble spots in ALGOL 60.
 {\sl Commun\-i\-ca\-tions of the ACM\/ \bf 10} (October 1967), 611--618.  
 Reprinted in E. Horowitz, {\sl Programming Languages: A Grand Tour\/}
 (Computer Science Press, 1982), 61--68.
 Reprinted as Chapter~8 of {\sl Selected Papers on
 Computer Languages\/} (see under Books).
\p P30.  A characterization of parenthesis languages.
  {\sl Information and Control\/ \bf 11} (1967), 269--289.  
 Reprinted as Chapter~13 of {\sl Selected Papers on
 Computer Languages\/} (see under Books).
\p P31.  Very magic squares.  {\sl American Mathematical
 Monthly\/ \bf 75} (March 1968), 260--264.  
\p P32.  Semantics of context-free languages.  {\sl Mathematical
 Systems Theory\/ \bf 2} (1968), 127--145.  Errata, 
 {\sl Mathematical Systems Theory\/ \bf 5} (1971), 95--96.  
 Reprinted as Chapter~17 of {\sl Selected Papers on
 Computer Languages\/} (see under Books).
\p P33.  Another enumeration of trees.  {\sl Canadian Journal
 of Mathematics\/ \bf 20} (1968), 1077--1086.  
 Reprinted as Chapter~13 of {\sl Selected Papers on
 Discrete Mathematics\/} (see under Books).
\p P34.  (with Peter B. Bendix)\xskip  Simple word problems in
 universal algebras.  In {\sl Compu\-ta\-tional Problems in Abstract
 Algebra}, J. Leech, ed.\ (Oxford:  Pergamon, 1970), 263--297.
 Reprinted in {\sl Automation of Reasoning}, edited by J\"org~H.
 Siekmann and Graham Wrightson, {\bf2} (Springer, 1983), 342--376.
\p P35.  The Gamov-Stern elevator problem.  {\sl Journal of Recreational 
 Mathematics\/ \bf 2} (1969), 131--137.  
\p P36.  Notes on central groupoids.  {\sl Journal of Combinatorial
 Theory\/ \bf 8} (1970), 376--390.  
 Reprinted as Chapter~22 of {\sl Selected Papers on
 Discrete Mathematics\/} (see under Books).
\p P37.  (with Chandler Davis)\xskip  Number representations and dragon
 curves.  {\sl Journal of Recrea\-tional Mathematics\/ \bf 3} (1970),
 66--81, 133--149.  
\p P38.  Permutations, matrices, and generalized Young tableaux. 
 {\sl Pacific Journal of Mathematics\/ \bf 34} (1970), 709--727.
 Reprinted as Chapter~31 of {\sl Selected Papers on
 Discrete Mathematics\/} (see under Books).
\p P39.  A note on solid partitions.  {\sl Mathematics of Computation\/
 \bf 24} (1970), 955--961.  
 Reprinted as Chapter~33 of {\sl Selected Papers on
 Discrete Mathematics\/} (see under Books).
\p P40.  Von Neumann's first computer program.  {\sl Computing
 Surveys\/ \bf 2} (December 1970), 247--260.  Reprinted in {\sl Papers
 of John von Neumann on Computing and Computer Theory}, ed.~by
 William Aspray and Arthur Burks (Cambridge, Mass.: MIT Press, 1987), 83--96.
 Reprinted with corrections as Chapter~12 of {\sl Selected Papers on
 Computer Science\/} (see under Books).
\p P41.  Optimum binary search trees.  {\sl Acta Informatica\/ \bf 1}
 (1971), 14--25.  Corrigenda, p.\ 270.
\p P42.  (with R. W. Floyd)\xskip Notes on avoiding `go to' statements.
 {\sl Information Processing Letters\/ \bf 1} (1971), 23--31.  Errata, p.\ 177.
 Reprinted in {\sl Writings of the Revolution}, E. Yourdon, ed.\ 
 (New York: Yourdon Press, 1982), 153--162.
 Reprinted as Chapter~23 of {\sl Selected Papers on
 Computer Languages\/} (see under Books).
\p P43.  Subspaces, subsets, and partitions.  {\sl Journal of
 Combinatorial Theory\/} (A) {\bf 10} (1971), 178--180.  
 Reprinted as Chapter~35 of {\sl Selected Papers on
 Discrete Mathematics\/} (see under Books).
\p P44.  The analysis of algorithms. {\sl Actes du Congr\`es
 International des Math\'e\-ma\-ticiens\/} 1970, {\bf 3} (Paris:
 Gau\-thier-Villars, 1971), 269--274.  
 Reprinted as Chapter~3 of {\sl Selected Papers on Analysis of Algorithms\/}
 (see under Books).
\p P45.  Examples of formal semantics.  In {\sl Symposium on Semantics of
 Algorithmic Languages}, E. Engeler, ed., {\sl Lecture Notes 
 in Mathematics\/ \bf 188} (Berlin: Springer, 1971), 212--235.  
 Reprinted as Chapter~18 of {\sl Selected Papers on
 Computer Languages\/} (see under Books).
\p P46.  Mathematical analysis of algorithms.  {\sl Proceedings of
 IFIP Congress 1971}, {\bf 1} (Amsterdam: North-Holland, 1972), 19--27.
 Reprinted as Chapter~1 of {\sl Selected Papers on Analysis of Algorithms\/}
 (see under Books).
\p P47.  An empirical study of FORTRAN programs.  {\sl Software---Practice
 and Experience\/ \bf 1} (1971), 105--133.  
 Reprinted as Chapter~24 of {\sl Selected Papers on
 Computer Languages\/} (see under Books).
\p P48.  Top-down syntax analysis.  {\sl Acta Informatica\/ \bf 1}
 (1971), 79--110.  Russian translation by Nadezhda I. V'\t\i ukova in
 {\sl Kiberneticheski\u\i\ Sbornik\/ \bf 15} (1978), 101--142.
 Reprinted as Chapter~14 of {\sl Selected Papers on
 Computer Languages\/} (see under Books).
\p P49.  (with Edward A. Bender)\xskip  Enumeration of plane partitions.
  {\sl Journal of Combinatorial Theory\/} (A) {\bf 13} (1972), 40--54.  
 Reprinted as Chapter~32 of {\sl Selected Papers on
 Discrete Mathematics\/} (see under Books).
\p P50.  (with R. L. Graham and T. S. Motzkin)\xskip  Complements and
 transitive closures.  {\sl Discrete Mathematics\/ \bf 2} (1972), 17--29.  
 Reprinted as Chapter~25 of {\sl Selected Papers on
 Discrete Mathematics\/} (see under Books).
\p P51.  (with N. G. de Bruijn and S. O. Rice)\xskip  The average height
 of planted plane trees.  In {\sl Graph Theory and Computing}, R. C. Read,
 ed.\ (Academic Press, 1972), 15--22.  
 Reprinted as Chapter~15 of {\sl Selected Papers on Analysis of Algorithms\/}
 (see under Books).
\p P52.  (with E. B. Kaehler)\xskip  An experiment in optimal sorting.
 {\sl Information Processing Letters\/ \bf 1} (1972), 173--176.
 Reprinted as Chapter~30 of {\sl Selected Papers on Analysis of Algorithms\/}
 (see under Books).
\p P53.  Ancient Babylonian algorithms.  {\sl Commun\-i\-ca\-tions of the
 ACM\/ \bf 15} (July 1972), 671--677.
 Errata, {\sl Commun\-i\-ca\-tions of the ACM\/ \bf 19} (February 1976), 108.  
 Reprinted with corrections and additions as Chapter~11 of {\sl Selected
 Papers on Computer Science\/} (see under Books).
\p P54.  George Forsythe and the development of Computer Science. 
 {\sl Commun\-i\-ca\-tions of the ACM\/ \bf 15} (August 1972), 721--726.
 Reprinted with revisions as Chapter~16 of {\sl Selected Papers on
 Computer Science\/} (see under Books); in the first two printings
 it was, however, called Chapter~14.
\p P55.  (with Robert W. Floyd)\xskip  The Bose--Nelson sorting problem.
 In {\sl A Survey of Combinatorial Theory}, Jagdish N. Srivastava, ed.\
 (Amsterdam: North-Holland, 1973), 163--172.
\p P56. The dangers of computer science theory. In {\sl Logic, Methodology,
 and Philosophy of Science\/ \bf 4}, ed.\ by P. Suppes et al.\ (North-Holland,
 1973), 189--195.
 Reprinted as Chapter~2 of {\sl Selected Papers on Analysis of Algorithms\/}
 (see under Books).
\p P57.  Permutations with nonnegative partial sums.  {\sl Discrete
 Mathematics\/ \bf 5} (1973), 367--371.  
 Reprinted as Chapter~28 of {\sl Selected Papers on
 Discrete Mathematics\/} (see under Books).
\p P58.  The triel:  A new solution.  {\sl Journal of Recreational
 Mathe\-ma\-tics\/ \bf 6} (1973), 1--7.  
\p P59.  (with Jill Knuth)\xskip  Mathematics and Art:  The dragon curve
 in ceramic tile.  {\sl Journal of Recreational Mathematics\/ \bf 6} (1973),
 165--167.  
\p P60.  (with Francis R. Stevenson)\xskip  Optimal measurement points for
 program frequency counts.  {\sl BIT\/ \bf 13} (1973), 313--322.  
 Reprinted as Chapter~5 of {\sl Selected Papers on Analysis of Algorithms\/}
 (see under Books).
\p P61.  Wheels within wheels. {\sl Journal of Combinatorial Theory\/} (B)
 {\bf 16} (1974), 42--46.  
 Reprinted as Chapter~24 of {\sl Selected Papers on
 Discrete Mathematics\/} (see under Books).
\p P62.  The asymptotic number of geometries.  {\sl Journal of
 Combinatorial Theory\/} (A) {\bf 16} (1974), 398--400.  
 Reprinted as Chapter~27 of {\sl Selected Papers on
 Discrete Mathematics\/} (see under Books).
\p P63.  Computer Science and its relation to Mathematics. {\sl American
 Mathematical Monthly\/ \bf 81} (April 1974), 323--343. A~shorter form of
 this article appeared in {\sl American Scientist\/ \bf 61} (1973), 707--713;
 reprinted in {\sl Computers and People\/ \bf 23},\,9 (September 1974), 8--11;
 and in {\sl Mathematics: People, Problems, Results}, ed.\ by Douglas M.
 Campbell and John C. Higgins, vol.~3 (Belmont, Calif.:  Wadsworth, 1984),
 37--47.  Hungarian translation in {\sl Matematikai Lapok\/ \bf 24} (1973,
 published 1975), 345--363.  Slovenian translation in {\sl Obzornik za
 Matematiko in Fiziko\/ \bf22} (1975), 129--138, 161--167. Slovak translation
 (abridged) in {\sl Pokroky Matematiky, Fiziky a Astronomie\/ \bf21} (1976),
 88--96.  Russian translation by Natal'\t\i a G. Gurevich in
 {\sl Sovremennye Problemy Matematiki\/ \bf11},\,12 (Moscow: Znanie, 1977),
 4--32. Reprinted in part in {\sl A Century of Mathematics Through the
 Eyes of the M{\eightsl ONTHLY}}, edited by John Ewing (Mathematical
 Association of America, 1994), 285--288.
 Reprinted with corrections as Chapter~1 of {\sl Selected Papers on
 Computer Science\/} (see under Books).
\p P64.  (with O. Amble)\xskip  Ordered hash tables.
 {\sl The Computer Journal\/ \bf 17} (May 1974), 135--142.  
 Reprinted as Chapter~7 of {\sl Selected Papers on Analysis of Algorithms\/}
 (see under Books).
\p P65.  (with Jayme L. Szwarcfiter)\xskip  A structured program to
 generate all topological sorting arrangements.  {\sl Information
 Processing Letters\/ \bf 2} (1974), 153--157; erratum, {\bf3} (1974), 64.
 Reprinted with revisions
 as Chapter~3 of {\sl Literate Programming\/} (see under Books).
\p P66.  (with Michael L. Fredman)\xskip  Recurrence relations based
 on minimization.  {\sl Journal of Mathematical Analysis and Applications\/
 \bf 48} (1974), 534--559.  
 Reprinted as Chapter~38 of {\sl Selected Papers on
 Discrete Mathematics\/} (see under Books).
\p P67.  Structured programming with {\bf go to} statements. 
 {\sl Computing Surveys\/ \bf 6} (December 1974), 261--301.  Reprinted
 with revisions in {\sl Current Trends in Programming Methodology},
 Raymond T. Yeh, ed., {\bf 1} (Englewood Cliffs, N.J.:  Prentice-Hall,
 1977), 140--194; {\sl Classics in Software Engineering}, Edward Nash
 Yourdon, ed.\ (New York: Yourdon Press, 1979), 259--321. Reprinted with
 ``final'' revisions as Chapter~2 of {\sl Literate Programming\/}
 (see under Books).
\p P68.  Computer programming as an art.  {\sl Communications of the ACM\/
 \bf 17} (December 1974), 667--673.  French translation, with three
 supplementary paragraphs, in {\sl L'Informatique Nouvelle}, No.\ 64
 (June 1975), 20--27.  Japanese translation by Makoto Arisawa
 in {\sl bit\/ \bf 7} (1975), 434--444; reprinted in
 {\sl Kunusu Sensei no Program-Ron\/} (see under Books), 2--19.
 English version reprinted
 with the supplementary paragraphs in {\sl ACM Turing Award
 Lectures: The First Twenty Years} (New York: ACM Press, 1987), 33--46;
 reprinted with corrections as Chapter~1 of {\sl Literate Programming\/} (see
 under Books).
 Russian translation by V. V. Martyn\t{\i}uk in {\sl Lektsii laureatov
 premii T'\t{\i}uringa\/} (Moscow: Mir, 1993), 48--64.
\p P69.  Estimating the efficiency of backtrack programs.
 {\sl Mathematics of Compu\-ta\-tion\/ \bf 29} (1975), 121--136.  
 Reprinted as Chapter~6 of {\sl Selected Papers on Analysis of Algorithms\/}
 (see under Books).
\p P70.  (with Ronald W. Moore)\xskip An analysis of alpha-beta pruning. 
 {\sl Artificial Intelligence\/ \bf 6} (1975), 293--326.  
 Reprinted as Chapter~9 of {\sl Selected Papers on Analysis of Algorithms\/}
 (see under Books).
\p P71.  (with James H. Morris, Jr.  and Vaughan R. Pratt)\xskip  Fast pattern
 matching in 
 strings.  {\sl SIAM Journal on Computing\/ \bf 6} (1977), 323--350. Errata,
 see W. Rytter, {\sl SIAM Journal on Computing\/ \bf 9} (1980), 509--512.
 Reprinted in {\sl Computer Algorithms: String Pattern Matching Strategies},
 ed.~by Jun-ichi Aoe (Los Alamitos, California: IEEE Computer Society Press,
 1994), 8--35.
\p P72.  Random matroids.  {\sl Discrete Mathematics\/ \bf 12} (1975),
 341--358.  
 Reprinted as Chapter~26 of {\sl Selected Papers on
 Discrete Mathematics\/} (see under Books).
\p P73. (with John F. Reiser)\xskip  Evading the drift in floating-point
 addition. {\sl Information Processing Letters\/ \bf 3} (1975), 84--87.
 Errata, p.\ 164.
\p P74.  (with Gururaj S. Rao)\xskip  Activity in an interleaved memory.
 {\sl IEEE Transactions on Computers\/ \bf C-24} (1975), 943--944.  
 Reprinted as Chapter~8 of {\sl Selected Papers on Analysis of Algorithms\/}
 (see under Books).
\p P75.  Notes on generalized Dedekind sums.  {\sl Acta Arithmetica\/
 \bf 33} (1977), 297--325.
 Reprinted as Chapter~10 of {\sl Selected Papers on Analysis of Algorithms\/}
 (see under Books).
\p P76.  (with Andrew C. Yao)\xskip  Analysis of the subtractive algorithm
 for greatest common divisors.  {\sl Proceedings of the National Academy
 of Sciences\/ \bf 72} (1975), 4720--4722.  
 Reprinted with additional material as Chapter~13 of {\sl Selected Papers
 on Analysis of Algorithms\/} (see under Books).
\p P77.  (with M. R. Garey, R. L. Graham, D. Johnson)\xskip  Complexity
 results for bandwidth minimization.  {\sl SIAM Journal on Applied
 Mathematics\/ \bf 34} (1978), 477--495.
 Reprinted with additional material as Chapter~32 of {\sl Selected Papers
 on Analysis of Algorithms\/} (see under Books).
\p P78.  (with Luis Trabb Pardo)\xskip  Analysis of a simple factorization
 algorithm.  {\sl Theoretical Computer Science\/ \bf 3} (1976), 321--348.  
 Reprinted with additional material as Chapter~20 of {\sl Selected Papers
 on Analysis of Algorithms\/} (see under Books).
\p P79.  Algorithms.  {\sl Scientific American\/ \bf236},\,4 (April 1977),
 63--66, 69--72, 77--78, 80.
 Also {\sl Scientific American Offprints}, number 360, 14 pp.
 Farsi translation by B. Parhami in {\sl Bulletin of the
  Iranian Mathematical Society\/ \bf 8} (1978), 122L--76L.  
 Reprinted with corrections as Chapter~3 of {\sl Selected Papers on
 Computer Science\/} (see under Books).
\p P80.  (with Andrew C. Yao)\xskip  The complexity of nonuniform random
 number generation.  In {\sl Algorithms and Complexity},
 J. F. Traub, ed.\ (New York:  Academic Press, 1976), 357--428.
 Russian translation by B. B. Pokhodze\u\i\ in {\sl Kiberneticheski\u\i\
 Sbornik\/ \bf 19} (1983), 97--158.
 Reprinted with additional material as Chapter~34 of {\sl Selected Papers
 on Analysis of Algorithms\/} (see under Books).
\p P81.  The computer as Master Mind.
 {\sl Journal of Recreational Mathematics\/ \bf 9} (1976), 1--6.  
\p P82.  Mathematics and Computer Science:  Coping with finiteness. 
 {\sl Science\/ \bf 194} (December 17, 1976), 1235--1242.  Reprinted
 with corrections in {\sl Electronics, the Continuing Revolution}, 
 edited by Philip H. Abelson and Allen L. Hammond, AAAS publication {\bf 77-4}
 (Washington, D.C.:\ American Association for the Advancement of Science,
 1977), 189--196; 
 and in {\sl Mathematics: People, Problems, Results}, ed.\ by Douglas M.
 Campbell and John C. Higgins, vol.~2 (Belmont, Calif.:  Wadsworth, 1984),
 209--222. Bulgarian translation by G. Chobanov and Z. Dokova in
 {\sl Fiziko-Matematichesko Spisanie\/ \bf 21} (Sofia, 1978), 58--74.
 German translation by Arthur Engel in {\sl Der Mathematik-Unterricht\/
 \bf 25},\,6 (1979), 5--26.
 Reprinted with corrections as Chapter~2 of {\sl Selected Papers on
 Computer Science\/} (see under Books).
\p P83.  (with Luis Trabb Pardo)\xskip  The early development of programming
 languages.  {\sl Encyclopedia of Computer Science and Technology},
 Jack Belzer, Albert G. Holzman, and Allen Kent, eds., {\bf 7}
 (New York: Marcel Dekker, Inc., 1977), 419--493.  
 Reprinted in {\sl A History of Computing in the Twentieth Century},
 N. Metropolis, J. Howlett, and G.-C. Rota, eds.\ (New York:
 Academic Press, 1980), 197--273.
 Reprinted with amendments as Chapter~1 of {\sl Selected Papers on
 Computer Languages\/} (see under Books).
\p P84.  (with Arne T. Jonassen)\xskip  A trivial algorithm whose analysis
 isn't.
 {\sl Journal of Computer and System Sciences\/ \bf 16} (1978), 301--322.
 Reprinted as Chapter~18 of {\sl Selected Papers on Analysis of Algorithms\/}
 (see under Books).
\p P85.  A generalization of Dijkstra's algorithm.  {\sl Information
 Processing Letters\/ \bf 6} (1977), 1--5.  
\p P86.  Evaluation of Porter's constant.  {\sl Computers 
 and Mathematics with Applications\/ \bf 2} (1976), 137--139.  
 Reprinted with additional material as Chapter~12 of {\sl Selected Papers on
 Analysis of Algorithms\/} (see under Books). 
\p P87.  (with Michael S. Paterson)\xskip  Identities from partition
 involutions. {\sl Fibonacci Quarterly\/ \bf 16} (June 1978), 198--212.
 Reprinted as Chapter~34 of {\sl Selected Papers on
 Discrete Mathematics\/} (see under Books).
\p P88.  (with Arnold Sch\"onhage)\xskip  The expected linearity of a simple
 equivalence algorithm.  {\sl Theoretical Computer Science\/ \bf 6}
 (1978), 281--315.  
 Reprinted with additional material as Chapter~21 of {\sl Selected Papers
 on Analysis of Algorithms\/} (see under Books).
\p P89.  Deletions that preserve randomness.  {\sl IEEE Transactions on
 Software Engineering\/ \bf SE-3} (1977), 351--359.  
 Reprinted as Chapter~19 of {\sl Selected Papers on Analysis of Algorithms\/}
 (see under Books).
\p P90.  The average time for carry propagation. {\sl Indagationes 
 Mathematic\ae\ \bf 40} (1978), 238--242.
 Reprinted as Chapter~26 of {\sl Selected Papers on Analysis of Algorithms\/}
 (see under Books).
\p P91.  Mathematical typography.  {\sl Bulletin of the American
 Mathematical Society\/} (new series) {\bf 1} (March 1979), 337--372.
 Reprinted with corrections as part 1 of {\sl \TeX\ and \slMF\/}
 (see under Books). Also reprinted in {\sl Dr.\ Dobb's Journal of Computer
 Calisthenics \&\ Orthodontia\/ \bf 5},\,3 (March 1980), 5--20.
 Reprinted as Chapter~2 of {\sl Digital Typography\/} (see under Books).
\p P92.  (with A. V. Anisimov)\xskip  Inhomogeneous sorting.
 {\sl International Journal of Computer and Information Sciences\/ \bf 8}
 (1979), 255--260.  Russian translation in {\sl Programmirovanie\/ \bf 5},\,1
 (1979), 11--14. English retranslation in {\sl Programming and Computer
 Software\/ \bf 5} (1979), 7--10.
\p P93.  Lexicographic permutations with restrictions.  {\sl Discrete Applied
 Mathematics\/ \bf 1} (1979), 117--125.
\p P94.  Algorithms in modern mathematics and computer science.  {\sl Lecture
 Notes in Computer Science\/ \bf 122} (1981), 82--99.
 Russian translation by G.~S. Tse\u\i tin
 in {\sl Algoritmy v Sovremennoi Matematike i Ee
 Pri\-lozheni\t{\i}akh}, Chast'~I (Novosibirsk: Akademi\t{\i}a Nauk SSSR,
 Sibirskoe Otdelenie, Vychislitel'ny\u\i\ Tsentr, 1982), 64--98.
 Revised version entitled ``Algorithmic thinking and mathematical thinking,''
 {\sl American Mathematical Monthly\/ \bf 92} (1985), 170--181.
 Reprinted with corrections as Chapter~4 of {\sl Selected Papers on
 Computer Science\/} (see under Books).
\p P95.  Supernatural numbers.  {\sl The Mathematical Gardner}, D. A. Klarner,
 ed.\ (Belmont, California: Wads\-worth International, 1981), 310--325;
 reprinted with new title {\sl Mathematical Recreations\/} (Dover, 1998).
 Russian translation by \t Iu.~A. Danilov in {\sl Matematicheski\u\i\
 tsvetnik}, edited by I.~M. \t Iaglom (Moscow: Mir, 1983),
 388--408. Japanese translation in {\sl Kunusu Sensei no Program-Ron\/}
 (see under Books), 22--43.
\p P96. The letter S. {\sl The Mathematical Intelligencer\/ \bf 2} (1980),
 114--122.
 Reprinted with corrections as Chapter~13 of {\sl Digital Typography\/}
 (see under Books).
\p P97. Deciphering a linear congruential encryption. {\sl IEEE Transactions
 on Information Theory\/ \bf IT-31} (1985), 49--52.
\p P98. (with Michael F. Plass)\xskip Breaking paragraphs into lines.
 {\sl Software---Practice and Experience\/ \bf 11} (1981), 1119--1184. 
 Reprinted with corrections as Chapter~3 of {\sl Digital Typography\/}
 (see under Books).
\p P99. Verification of link-level protocols. {\sl BIT\/ \bf 21} (1981),
 31--36.
\p P100. The concept of a meta-font.  {\sl Visible Language\/ \bf 16} (1982),
 3--27. French translation by M. R. Delorme in
 {\sl Communication et Langages\/} no.~55, (1983), 40--53; reprinted in
 {\sl Typographie et Informatique}, proceedings of INRIA conference held
 21--25 January 1985, organized by Jacques Andr\'e and Patrick Sallio
 (Rennes, France: INRIA/IRISA -- CCETT, 1985), 119--132.
 Reprinted as Chapter~15 of {\sl Digital Typography\/} (see under Books).
\p P101. Huffman's algorithm via algebra. {\sl Journal of Combinatorial
 Theory\/} (A) {\bf 32} (1982), 216--224.
 Russian translation by B. B. Pokhodze\u\i\
 in {\sl Kiberneticheski\u\i\ Sbornik\/ \bf 22} (1985), 159--169.
 Reprinted as Chapter~23 of {\sl Selected Papers on
 Discrete Mathematics\/} (see under Books).
\p P102. A permanent inequality. {\sl American Mathematical Monthly\/ \bf 88}
 (1981), 731--740, 798.
 Reprinted as Chapter~6 of {\sl Selected Papers on
 Discrete Mathematics\/} (see under Books).
\p P103. Dynamic Huffman coding. {\sl Journal of Algorithms\/ \bf 6} (1985),
 163--180.
\p P104. An analysis of optimum caching. {\sl Journal of Algorithms\/ \bf 6}
 (1985), 181--199.
 Reprinted as Chapter~17 of {\sl Selected Papers on Analysis of Algorithms\/}
 (see under Books).
\p P105. (with David R. Fuchs)\xskip Optimal prepaging and font caching.
 {\sl ACM Transactions on Programming Languages and Systems\/
 \bf 7} (1985), 62--79.
\p P106. The distribution of continued fraction approximations. {\sl Journal
 of Number Theory\/ \bf 19} (1984), 443--448.
 Reprinted as Chapter~11 of {\sl Selected Papers on Analysis of Algorithms\/}
 (see under Books).
\p P107. An algorithm for Brownian zeroes. {\sl Computing\/ \bf 33} (1984),
 89--94.
\p *P108. (with Michael F. Plass)\xskip Choosing better line breaks.
 In {\sl Document Preparation Systems}, Nievergelt et al., eds.\ (Amsterdam:
 North-Holland, 1982), 221--242.
\p P109. Literate programming. {\sl The Computer Journal\/ \bf 27} (1984),
 97--111. Reprinted with corrections as Chapter~4 of {\sl Literate
 Programming\/} (see under Books).
 Japanese translation by Toshiaki Kurokawa in {\sl bit\/ \bf 17} (1985),
 426--450; reprinted in {\sl Kunusu Sensei no Program-Ron\/}
 (see under Books), 82--128.
\p P110. Lessons learned from \MF. {\sl Visible Language\/ \bf 19} (1985),
 35--53.
 Reprinted with corrections as Chapter~16 of {\sl Digital Typography\/}
 (see under Books).
\p P111. The toilet paper problem. {\sl American Mathematical Monthly\/
 \bf 91} (1984), 465--470.
 Reprinted with additional material as Chapter~16 of {\sl Selected Papers
 on Analysis of Algorithms\/} (see under Books).
\p P112. The IBM 650: An appreciation from the field. {\sl Annals
 of the History of Computing\/ \bf 8} (1986), 50--55.
 Reprinted with corrections as Chapter~13 of {\sl Selected Papers on
 Computer Science\/} (see under Books).
\p P113. Semi-optimal bases for linear dependencies. {\sl Linear
 and Multilinear Algebra\/ \bf 17} (1985), 1--4.
\p P114. Efficient balanced codes. {\sl IEEE Transactions
 on Information Theory\/ \bf IT-32} (1986), 51--53.
 Reprinted in Mario Blaum, {\sl Codes for Detecting and Correcting
 Unidirectional Errors\/} (Los Alamitos, California:\ IEEE Computer
 Society Press, 1993).
 Reprinted as Chapter~29 of {\sl Selected Papers on
 Discrete Mathematics\/} (see under Books).
\p P115. (with Huang Bing-Chao)\xskip A one-way, stackless quicksort
 algorithm. {\sl BIT\/ \bf 26} (1986), 127--130.
\p P116. Digital halftones by dot diffusion. {\sl ACM Transactions on
 Graphics\/ \bf 6} (1987), 245--273.
 Reprinted with revisions as Chapter~22 of {\sl Digital Typography\/}
 (see under Books).
\p P117. Fibonacci multiplication. {\sl Applied Mathematics Letters\/ \bf1}
 (1988), 57--60.
\p *P118. (with Christos H. Papadimitriou and John N. Tsitsiklis)\xskip A note
 on strategy elimination in bimatrix games. {\sl Operations Research Letters\/
 \bf7} (1988), 103--107. Errata, see Gilboa et al., {\sl Operations
 Research Letters\/ \bf9} (1990), 85--89.
\p P119. A Fibonacci-like sequence of composite numbers. {\sl Mathematics
 Magazine\/ \bf63} (1990), 21--25.
\p P120. (with Boris Pittel)\xskip A recurrence related to trees.
 {\sl Proceedings of the American Mathematical Society\/ \bf105} (1989),
 335--349.
 Reprinted as Chapter~39 of {\sl Selected Papers on
 Discrete Mathematics\/} (see under Books).
\p P121. (with Herbert S. Wilf)\xskip The power of a prime that divides a
 generalized binomial coefficient. {\sl Journal f\"ur die reine und angewandte
 Mathematik\/ \bf396} (1989), 212--219.
 Reprinted as Chapter~36 of {\sl Selected Papers on
 Discrete Mathematics\/} (see under Books).
\p P122. (with Philippe Flajolet and Boris Pittel)\xskip The first cycles in
 an evolving graph. {\sl Discrete Mathematics\/ \bf75} (1989), 167--215.
 This volume is also published as {\sl Combinatorics 1988}, Proceedings
 of the Cambridge Conference in Honour of Paul Erd\H os, edited
 by B\'ela Bollob\'as (North-Holland, 1989).
 Reprinted as Chapter~40 of {\sl Selected Papers on
 Discrete Mathematics\/} (see under Books).
\p P123. Efficient representation of perm groups. {\sl Combinatorica \bf11}
 (1991), 33--43.
\p P124. The errors of \TeX. {\sl Software---Practice
 and Experience\/ \bf19} (1989), 607--685. Reprinted with additions
 and corrections as Chapters 10 and~11 of {\sl Literate Programming\/}
 (see under Books).
\p P125. (with Hermann Zapf)\xskip AMS Euler---A new typeface for mathematics.
 {\sl Scholarly Publishing\/ \bf20} (1989), 131--157. Abridged version
 in {\sl ABC--XYZapf}, edited by John Dreyfus and Knut Erichson
 (London: Wynkyn de Worde Society, 1989), 171--179.
 Reprinted with corrections as Chapter~17 of {\sl Digital Typography\/}
 (see under Books).
\p P126. (with Robert W. Floyd)\xskip Addition machines. {\sl SIAM Journal on
 Computing\/ \bf19} (1990), 329--340.
\p P127. (with Rajeev Motwani and Boris Pittel)\xskip Stable husbands. {\sl
 Random Structures \& Algorithms\/ \bf1} (1990), 1--14.
 Reprinted as Chapter~24 of {\sl Selected Papers on Analysis of Algorithms\/}
 (see under Books).
\p *P128. (with Richard Garfield and Herbert S. Wilf)\xskip A bijection for
 ordered factorizations. {\sl Journal of Combinatorial Theory\/} (A)
 {\bf 54} (1990), 317--318.
\p *P129. (with Herbert S. Wilf)\xskip A short proof of Darboux's lemma.
 {\sl Applied Mathematics Letters\/ \bf2} (1989), 139--140.
\p *P130. (with Leonidas J. Guibas and Micha Sharir)\xskip Randomized
 incremental construction of Delaunay and Voronoi diagrams.
 {\sl Algorithmica\/ \bf7} (1992), 381--413.
 Abbreviated version in {\sl Automata, Languages and Programming},
 edited by M.~S. Paterson, {\sl Lecture Notes in Computer Science\/ \bf443}
 (1990), 414--431.
\p *P131. (with Lee Sallows, Martin Gardner, Richard K. Guy)\xskip Serial
 isogons of 90 degrees. {\sl Mathematics Magazine\/ \bf64} (1991), 315--324.
\p P132. (with Arvind Raghunathan)\xskip The problem of compatible
 representatives.
 {\sl SIAM Journal on Discrete Mathematics\/ \bf5} (1992), 422--427.
 Reprinted as Chapter~33 of {\sl Selected Papers on Analysis of Algorithms\/}
 (see under Books).
\p P133. A simple program whose proof isn't. {\sl Beauty is Our Business},
 edited by W. H. J. Feijen et al., a festschrift for Edsger Dijkstra
 (Springer, 1990), 233--242.
\p P134. Nested satisfiability. {\sl Acta Informatica \bf28} (1990), 1--6.
\p P135. Textbook examples of recursion. {\sl Artificial
 Intelligence and Mathematical Theory of Computation}, papers in honor
 of John McCarthy, edited by Vladimir Lifschitz
 (Academic Press, 1991), 207--229.
 Reprinted with additional material as Chapter~22 of {\sl Selected Papers
 on Analysis of Algorithms\/} (see under Books).
\p P136. A note on digitized angles. {\sl Electronic Publishing---Origination,
 Dissemination, and Design\/ \bf3} (1990), 99--104.
 Reprinted as Chapter~23 of {\sl Digital Typography\/} (see under Books).
\p P137. Two notes on notation. {\sl American Mathematical
 Monthly\/ \bf99} (1992), 403--422; {\bf 102} (1995), 562.
 Reprinted as Chapter~2 of {\sl Selected Papers on
 Discrete Mathematics\/} (see under Books).
\p P138. Theory and practice. {\sl Theoretical Computer Science\/ \bf90}
 (1991), 1--15. Also published in {\sl Images of Programming}, dedicated to
 the memory of A.~P. Ershov, edited by
 D. Bj\o rner and V. Kotov (Amsterdam: North-Holland, 1991), 1--15.
 Reprinted with corrections as Chapter~9 of {\sl Selected Papers on
 Computer Science\/} (see under Books).
\p P139. Context-free multilanguages. {\sl Theoretical
 Studies in Computer Science}, edited by Jeffrey D. Ullman, a festschrift
 for Seymour Ginsburg (Academic Press, 1992), 1--13.
 Reprinted as Chapter~16 of {\sl Selected Papers on
 Computer Languages\/} (see under Books).
\p P140. (with Svante Janson, Tomasz \L uczak, Boris Pittel)\xskip The birth
 of the giant component. {\sl Random Structures \& Algorithms\/ \bf4}
 (1993), 233--358.
 Reprinted as Chapter~41 of {\sl Selected Papers on
 Discrete Mathematics\/} (see under Books).
\p P141. Convolution polynomials. {\sl Mathematica Journal\/ \bf2},\,4 (Fall
 1992), 67--78.
 Reprinted as Chapter~15 of {\sl Selected Papers on
 Discrete Mathematics\/} (see under Books).
\p P142. Johann Faulhaber and sums of powers. {\sl Mathematics of
 Computation\/ \bf 61} (1993), 277--294.
 Reprinted as Chapter~4 of {\sl Selected Papers on
 Discrete Mathematics\/} (see under Books).
\p P143. Bracket notation for the `coefficient-of' operator. {\sl A
 Classical Mind}, essays in honour of C. A. R. Hoare, edited by A. W. Roscoe
 (Prentice-Hall, 1994), 247--258.
 Reprinted as Chapter~3 of {\sl Selected Papers on
 Discrete Mathematics\/} (see under Books).
\p P144. Mini-indexes for literate programs. {\sl Software---Concepts
 and Tools\/ \bf15} (1994), 2--11.
 Reprinted as Chapter~11 of {\sl Digital Typography\/} (see under Books).
\p P145. Two-way rounding. {\sl SIAM Journal on Discrete Mathematics\/ \bf8},
 (1995), 281--290.
\p P146. (with Inger Johanne H{\aa}land)\xskip Polynomials involving the floor
 function. {\sl Mathematica Scandinavica\/ \bf76} (1995), 194--200.
 Reprinted as Chapter~16 of {\sl Selected Papers on
 Discrete Mathematics\/} (see under Books).
\p P147. Leaper graphs. {\sl The Mathematical Gazette\/ \bf78} (1994),
 274--297.
\p P148. The sandwich theorem. {\sl Electronic Journal of Combinatorics\/
 \bf1} (1994), article 1, 48~pp.
 Reprinted as Chapter~8 of {\sl Selected Papers on
 Discrete Mathematics\/} (see under Books).
\p P149. An exact analysis of stable allocation. {\sl Journal of Algorithms\/
 \bf20} (1996), 431--442.
 Reprinted as Chapter~23 of {\sl Selected Papers on Analysis of Algorithms\/}
 (see under Books).
\p P150. Aztec diamonds, checkerboard graphs, and spanning trees. {\sl Journal
 of Algebraic Combinatorics\/ \bf6} (1997), 253--257.
 Reprinted as Chapter~10 of {\sl Selected Papers on
 Discrete Mathematics\/} (see under Books).
\p P151. Irredundant intervals. {\sl ACM Journal of Experimental
 Algorithmics\/ \bf1} (1996), article~1, 19~pp.
\p *P152. (with David J. Jeffrey, Robert M. Corless, David E. G. Hare)\xskip
 Sur l'inversion de $y^\alpha e^y$ au moyen de nombres de Stirling associ\'es.
 {\sl Comptes Rendus de l'Acad\'emie des Sciences\/}, s\'erie~I, {\bf320}
 (1995), 1449--1452.
\p P153. Partitioned tensor products and their spectra. {\sl Journal of
 Algebraic Combinatorics\/ \bf6} (1997), 259--267.
 Reprinted as Chapter~11 of {\sl Selected Papers on
 Discrete Mathematics\/} (see under Books).
\p P154. The Knowlton--Graham partition problem. {\sl Journal of Combinatorial
 Theory\/} (A) {\bf73} (1996), 185--189.
 Reprinted as Chapter~30 of {\sl Selected Papers on
 Discrete Mathematics\/} (see under Books).
\p *P155. (with R. M. Corless, G. H. Gonnet, D. E. G. Hare, and D. J.
 Jeffrey)\xskip On the Lambert $W$ function. {\sl Advances in Computational
 Mathematics\/ \bf5} (1996), 329--359.
\p P156. Overlapping Pfaffians. {\sl Electronic Journal of Combinatorics\/
 \bf3},\,2 (1996), paper R5, 13~pp. Reprinted in {\sl The Foata Festschrift},
 ed.~by Jacques D\'esarm\'enien, Adalbert Kerber, and Volker Strehl (Gap:
 Imprimerie Louis-Jean, 1996), 151--163.
 Reprinted as Chapter~7 of {\sl Selected Papers on
 Discrete Mathematics\/} (see under Books).
\p P157. (with Svante Janson)\xskip Shellsort with three increments.
 {\sl Random Structures \& Algorithms\/ \bf10} (1997), 125--142.
 Reprinted as Chapter~25 of {\sl Selected Papers on Analysis of Algorithms\/}
 (see under Books).
\p P158. Linear probing and graphs. {\sl Algorithmica\/ \bf22} (1998),
 561--568.
 Reprinted as Chapter~27 of {\sl Selected Papers on Analysis of Algorithms\/}
 (see under Books).
\p P159. Dancing links. {\sl Millennial Perspectives in Computer Science},
 edited by Jim Davies, Bill Roscoe, and Jim Woodcock (Houndmills, Basingstoke,
 Hampshire:\ Palgrave, 2000), 187--214.
\p P160. (with Frank Ruskey)\xskip Efficient coroutine generation of
 constrained Gray sequences. {\sl From Object-Orientation to Formal Methods:
 Dedicated to the Memory of Ole-Johan Dahl}, edited by O.~Owe, S.~Krogdahl,
 and T.~Lyche, {\sl Lecture Notes in Computer Science\/ \bf2635} (Heidelberg:
 Springer-Verlag, 2004), 183--204.
 Reprinted as Chapter~25 of {\sl Selected Papers on
 Computer Languages\/} (see under Books).
\finishpage
%%%%%
\sectionbegin 4.  {Other Publications \rm(unrefereed contributions)
 ($\ast$ means written by coauthor or editor)}
\p *Q1.  (with J. D. Alanen)\xskip  A table of minimum functions for
 generating Galois fields GF$(p^n)$.  {\sl Sankhy\=a}, series~A,
 {\bf 23} (1961), 128.  
\p Q2.  Backus' language.  {\sl Commun\-i\-ca\-tions of the ACM\/ \bf 5}
 (April 1962), 185.
\p Q3.  The calculation of Easter.  {\sl Commun\-i\-ca\-tions of the ACM\/
 \bf 5} (April 1962), 209--210.  
\p Q4.  Non-Desarguesian planes of order $2^{2m+1}$.  {\sl Notices of the
 American Mathematical Society\/ \bf 9} (June 1962), 218.  
\p Q5.  History of writing compilers.  Digest of Technical Papers,
 {\sl ACM 62 National Conference\/} (September 1962), 43, 126.  
\p Q6.  Evaluation of polynomials by computer.
 {\sl Commun\-i\-ca\-tions of the ACM\/ \bf 6} (February 1963), 51.  
\p Q7.  Review of {\sl Computer Applications in the Behavioral Sciences}. 
 {\sl Computing Reviews\/ \bf 4} (May--June 1963), 120--122.  
\p Q8.  Letters on merging.  {\sl Commun\-i\-ca\-tions of the ACM\/ \bf 6}
 (October 1963), 585--587.  
\p Q9.  Addition chains and the evaluation of $n$th powers. 
 {\sl Notices of the American Mathematical Society\/ \bf 11} (February 1964),
 230--231.  
\p Q10.  Non-Desarguesian planes of order $2^{2m+1}$.  {\sl Notices of the
 American Mathematical Society\/ \bf 11} (June 1964), 445--446.
\p Q11.  Backus Normal Form vs.\ Backus Naur Form.  {\sl Commun\-i\-ca\-tions
 of the ACM\/ \bf 7} (December 1964), 735--736.  
 Reprinted as Chapter~2 of {\sl Selected Papers on
 Computer Languages\/} (see under Books).
\p Q12.  Man or boy?  {\sl Algol Bulletin\/ \bf 17} (Amsterdam: Mathematisch
 Centrum, July 1964) 7; {\bf 19} (January 1965), 8--9.  
 Reprinted as Chapter~6 of {\sl Selected Papers on
 Computer Languages\/} (see under Books).
\p Q13.  Teaching ALGOL 60.  {\sl Algol Bulletin\/ \bf 19}
 (Amsterdam: Mathematisch Centrum, January 1965), 4--6.  
 Reprinted as Chapter~3 of {\sl Selected Papers on
 Computer Languages\/} (see under Books).
\p Q14.  A list of the remaining trouble spots in ALGOL 60.  {\sl Algol
 Bulletin\/ \bf 19} (Amsterdam: Mathematisch Centrum, January 1965), 29--38. 
\p Q15.  Comments concerning PL/I language specifications as published
 in the IBM Manual (Form C28-6571-1).  {\sl PL/I Bulletin\/ \bf 1}
 (January 1966), 5--14.  
\p Q16.  Problem 5264, the triangle inequality and the parallelogram
 law.  {\sl American Mathematical Monthly\/ \bf 72} (1965), 193;
 solutions in {\bf 73} (1966), 211--212.
\p Q17.  Additional comments on a problem in concurrent programming control.
 {\sl Commun\-i\-ca\-tions of the ACM\/ \bf 9} (May 1966), 321--322;
 errata p.\ 878.  Reprinted in
 {\sl Commun\-i\-ca\-tions of the ACM\/ \bf 26} (January 1983), 22.
\p Q18.  Algorithm and program; information and data. 
 {\sl Commun\-i\-ca\-tions of the ACM\/ \bf 9} (September 1966), 654.  
 Reprinted in {\sl Selected Papers on Computer Science\/}
 (see under Books), 1--2.
\p Q19.  (with R. W. Floyd)\xskip  Improved constructions for the
 Bose-Nelson sorting problem.  {\sl Notices of the American
 Mathematical Society\/ \bf 14} (February 1967), 283.  
\p Q20.  What is an algorithm?  {\sl Datamation\/ \bf 13},\,10
 (October 1967), 30--32.  
\p Q21. Comments on programming languages. {\sl Simulation Programming
 Languages}, edited by J.~N. Buxton (North-Holland, 1968), passim.
\p Q22.  Evolution of number systems.  {\sl Datamation\/ \bf 15}
 (February 1969), 93--97; (April 1969), 307, 309; (May 1969), 229. 
 Reprinted with corrections in Elias M. Awad,
 {\sl Automatic Data Processing} (Prentice-Hall, 1970), 353--357.  
\p Q23.  Uncrossed knight's tours.  {\sl Journal of Recreational
 Mathematics\/ \bf 2} (1969), 155-157.
\p Q24.  (with Edward A. Bender)\xskip  Constructive enumeration of plane 
 partitions.  {\sl Notices of the American Mathematical Society\/ \bf 16}
 (June 1969), 659.  
\p Q25.  Discussion of Mr. Riordan's paper ``Abel identities and inverse 
 relations''.  In {\sl Combinatorial Mathematics and Its Applications}, 
 R. C. Bose and T. A. Dowling, eds., University of North Carolina Monograph
 Series in Probability and Statistics {\bf 4} (Chapel Hill, North Carolina,
 1969), 91--94.
 Reprinted as Chapter~14 of {\sl Selected Papers on
 Discrete Mathematics\/} (see under Books).
\p Q26.  Review of {\sl Game Playing with Computers}, by Donald D. Spencer.
 {\sl Journal of Recreational Mathematics\/ \bf 2} (1969), 237--238.  
\p Q27.  Review of {\sl Introduction to Combinatorial Mathematics},
 by C. L. Liu.  {\sl IEEE Transactions on Information Theory\/ \bf IT-17}
 (1971), 119--120.  
\p Q28.  (with Michael L. Fredman)\xskip  Recurrence relations based on
 minimization.  {\sl Notices of the American Mathematical Society\/
 \bf 18} (October 1971), 960.  
\p Q29.  (with Ronald L. Rivest)\xskip Bibliography on computer sorting. 
 {\sl Computing Reviews\/ \bf 13} (June 1972), 283--289.
\p Q30.  Sequences with precisely $k+1$ $k$-blocks; Schr\"oder's problem;
 groups.  Solutions to Problems E2307, E2315, E2328.
 {\sl American Mathematical Monthly\/ \bf 79} (1972), 773--774,
 910, 1138--1139.
\p Q31.  The history of sorting.  {\sl Datamation\/ \bf 18} (December 1972),
 64, 69--70.  
\p Q32.  S\o king etter noe i en EDB-maskin.  (Norwegian)
 {\sl Forskningsnytt\/ \bf 18},\,4 (Norges Almenvitenskapelige
 Forskningsr{\aa}d, 1973), 39--42.  
\p Q33.  A terminological proposal.  {\sl SIGACT News\/ \bf 6},\,1
 (January 1974), 12--18.  
 Reprinted as Chapter~28 of {\sl Selected Papers on Analysis of Algorithms\/}
 (see under Books).
\p Q34. Letter to the editor. {\sl Fibonacci Quarterly\/ \bf12} (1974),
 46, 79, 82.
\p Q35.  Review of {\sl The Origins of Digital Computers}, by Brian Randell.
 {\sl Historia Mathematica\/ \bf 1} (1974), 204--207.
\p Q36.  Postscript about NP-hard problems.  {\sl SIGACT News\/ \bf 6},\,2
 (April 1974), 15--16.  
 Reprinted as Chapter~29 of {\sl Selected Papers on Analysis of Algorithms\/}
 (see under Books).
\p Q37. Elementary Problem E2492 (binomial coefficients and mods).
 {\sl American Mathematical Monthly\/ \bf 81} (1974), 902;
 solution in {\bf 82} (1975), 855.
\p Q38. Problem 6049 (cyclic permutation generators).
 {\sl American Mathematical Monthly\/ \bf 82} (1975), 856;
 solution in {\bf 84} (1977), 397.
\p Q39. Problem 6050 (random maximization).
 {\sl American Mathematical Monthly\/ \bf 82} (1975), 856;
 solution in {\bf 85} (1978), 686--688.
\p Q40.  (with Charles T. Zahn Jr.)\xskip  Ill-chosen use of
 ``event''.  {\sl Commun\-i\-ca\-tions of the ACM\/ \bf 18} (June 1975), 360.  
\p Q41. Son of Seminumerical Algorithms.  {\sl SIGSAM Bulletin\/ \bf 9},\,4
 (November 1975), 10--11.  
\p Q42. Elementary Problem E2613 (compact sets).
 {\sl American Mathematical Monthly\/ \bf 83} (1976), 656;
 solution in {\bf 84} (1977), 827--828.
\p Q43.  Big Omicron and Big Omega and Big Theta.  {\sl SIGACT News\/
 \bf 8},\,2 (April--June 1976), 18--24.  
 Reprinted as Chapter~4 of {\sl Selected Papers on Analysis of Algorithms\/}
 (see under Books).
\p Q44.  Felix vs.\ Rover.  {\sl Journal of Recreational Mathematics\/
 \bf 9} (1976), 59--60.  
\p Q45. Elementary Problem E2636 (diphages and triphages).
 {\sl American Mathematical Monthly\/ \bf 84} (1977), 134;
 solution in {\bf 85} (1978), 385--386.
\p Q46.  Are toy problems useful?  {\sl Popular Computing\/ \bf 5},\,1
 (January 1977), 1, 3--10; {\bf5},\,2 (February 1977), 3--7.  
 Reprinted with corrections as Chapter~10 of {\sl Selected Papers on
 Computer Science\/} (see under Books).
\p Q47.  BCS examination.  {\sl The Computer Bulletin\/ \bf 2},\,9
 (September 1976), 29.  
\p Q48.  The complexity of songs.  {\sl SIGACT News\/ \bf 9},\,2 
 (Summer 1977), 17--24. Reprinted in {\sl Communications of the ACM\/
 \bf 27} (April 1984), 344--346; errata (June 1984), 593.
 Reprinted in {\sl Metafolkloristica}, edited by Franz Kinder and
 Boaz the Clown (Salt Lake City, Utah 84158-8183: Frank and Boaz, P.O. Box
 58183), 63--65. Reprinted in {\sl Humour the Computer}, edited by
 Andrew Davison (Cambridge, Massachusetts: MIT Press, 1995), 139--145.
\p Q49.  Solution to Problem 76-17, Conway's ``topswaps'' shuffle.  {\sl SIAM
 Review\/ \bf 19} (October 1977), 739--741.  
\p Q50.  Organ duets.  {\sl Music\/ \bf 12},\,1 (January 1978), 6.
\p Q51.  Lewis Carroll's
 {\sc WORD}-{\sc WARD}-{\sc WARE}-{\sc DARE}-{\sc DAME}-{\sc GAME},
 {\sl GAMES\/ \bf 2},\,4 (July 1978), 22--23.  
\p Q52.  {\sc BLOOD}, {\sc SWEAT}, and {\sc TEARS}.  {\sl GAMES\/
 \bf 2},\,4 (July 1978), 49.
\p Q53.  Computer-assisted indexing.  {\sl The Indexer\/ \bf 11} (April 1979),
 135.
\p Q54.  Disappearances (poem).  {\sl The Mathematical Gardner},
 D. A. Klarner, ed.\ (Belmont, California: Wads\-worth International, 1981),
 264; reprinted with new title {\sl Mathematical Recreations\/} (Dover, 1998).
 Reprinted in {\sl Mathematics: A Human Endeavor} by Harold~R. Jacobs,
 third edition (San Francisco: Freeman, 1994), 53.
 Reprinted in {\sl Kunusu Sensei no Program-Ron\/} (see under Books),
 192--193.
 Russian translation in {\sl Matematicheski\u\i\ tsvetnik\/} (Mir, 1983), 329.
\p *Q55. Donald E. Knuth speaks out (interview by David H. Ahl). {\sl Creative
 Computing\/ \bf 6},\,1 (January 1980), 72--75.
\p Q56. Problem 80-6, random 2D trees.
 {\sl Journal of Algorithms\/ \bf 1} (1980), 109;
 solution in {\bf 3} (1982), 368--371.
\p Q57. Problem 80-11, inorder depth versus preorder depth.
 {\sl Journal of Algorithms\/ \bf 1} (1980), 210.
\p Q58. (with Christos H. Papadimitriou)\xskip Duality in addition chains.
 {\sl Bulletin of the EATCS\/ \bf 13} (February 1981), 2--4.
 Reprinted as Chapter~31 of {\sl Selected Papers on Analysis of Algorithms\/}
 (see under Books).
\p Q59. Letter to the editor re bubble sort.
 {\sl Popular Computing\/ \bf 9},\,1 (January 1981), 7.
\p Q60. Penny flipping. {\sl Popular Computing\/ \bf 9},\,4 (April 1981),
 10, 12.
\p Q61. Problem 81-10, optimum caching with two page frames.
 {\sl Journal of Algorithms\/ \bf 2} (1981), 315.
\p *Q62. A conversation with Don Knuth (interview by Donald J. Albers and Lynn
 Arthur Steen). {\sl Two-Year College Mathematics Journal\/ \bf 13}
 (1982), 2--18, 128--141.
 Reprinted in {\sl Annals of the History of Computing\/ \bf 4} (1982),
 257--274. Reprinted in {\sl Mathematical People}, ed.~by Donald~J. Albers
 and G. L. Alexanderson (Boston: Birkh\"auser Boston, 1985), 182--203.
 Japanese translation by Nobuko Kishi in {\sl bit\/ \bf16} (1985), 370--377,
 506--512, 902--906, 1020--1025;  reprinted in {\sl Kunusu Sensei
 no Program-Ron\/} (see under Books), 130--167.
\p Q63. (with A. P. Ershov)\xskip Editors' foreword to the proceedings of a
 conference on ``Algorithms in Modern Mathematics and Computer Science,''
 Urgench, Uzbek SSR, September 16--22, 1979. {\sl Lecture Notes in Computer
 Science\/ \bf 122} (1981), iii--v.
 Russian translation in {\sl Algoritmy v sovremenno\u\i\ matematike i e\"e
 prilozheni\t{\i}akh}, Chast'~I (Novosibirsk: Akademi\t{\i}a Nauk SSSR,
 Sibirskoe Otdelenie, Vychislitel'ny\u\i\ Tsentr, 1982), 4--7.
\p Q64. Problem 82-3, late binding trees.
 {\sl Journal of Algorithms\/ \bf 3} (1982), 178--180;
 solution in {\bf 4} (1983), 385--393.
\p Q65. (with R. L. Graham)\xskip Elementary problem E2982, a double infinite
 sum for $\vert x\vert$. {\sl American Mathematical Monthly\/ \bf 90} (1983),
 54; solution in {\bf 96} (1989), 525--526.
\p Q66. Fixed-point glue setting: An example of {\tt WEB}. {\sl TUGboat\/
 \bf 3},\,1 (March 1982), 10--27. Errata, {\sl TUGboat\/ \bf12},\,2
 (June 1991), 313.
\p Q67. A reply from the author. {\sl Visible Language\/ \bf 16} (1982),
 358--359.
 [A response to 16 reviews of paper P100; the reviews appear on pp.\ 308--358.]
 Also {\sl Visible Language\/ \bf 17} (1983), 417. Reprinted in the
 addendum to Chapter~15 of {\sl Digital Typography\/} (see under Books).
\p Q68.  Review of {\sl History of Binary and other Nondecimal Numeration},
 by Anton Glaser. {\sl Historia Mathematica\/ \bf 10} (1983), 236--243.  
 Partially reprinted as Chapter~5 of {\sl Selected Papers on
 Discrete Mathematics\/} (see under Books).
\p Q69. \TeX\ incunabula. {\sl TUGboat\/ \bf 5},\,1 (May 1984), 4--11.
 Reprinted as Chapter~26 of {\sl Digital Typography\/} (see under Books).
\p Q70. My first experience with Indian scripts. {\sl CALTIS-84}, a
 conference on calligraphy, lettering, typography of Indic scripts
 (New Delhi: February 11--13, 1984), 49.
 Reprinted as Chapter~14 of {\sl Digital Typography\/} (see under Books).
\p Q71. Solution to Problem 83-3, a binomial double sum involving max.
 {\sl SIAM Review\/ \bf 26} (1984), 123--124.
\p Q72. Letter to the editor: Comments on quality in publishing.
 {\sl TUGboat\/ \bf 5},\,1 (May 1984), 67.
\p Q73. FORTRAN implementations (letter). {\sl Annals of the History of
 Computing\/ \bf 6} (October 1984), 402--403.
\p Q74. A course on \MF\ programming. {\sl TUGboat\/ \bf 5},\,2
 (November 1984), 105--118.
 Reprinted as Chapter~19 of {\sl Digital Typography\/} (see under Books).
\p Q75. Recipes and fractions. {\sl TUGboat\/ \bf 6},\,1 (March, 1985), 36--38.
 Reprinted as Chapter~5 of {\sl Digital Typography\/} (see under Books).
\p *Q76. (with Niklaus Wirth)\xskip Programming philosophy (interviews by Ken
 Takara). {\sl Computer Language\/ \bf 2},\,5 (May 1985), cover, 25--35.
\p Q77. Problem 1234: Sorted integers. {\sl Mathematics Magazine\/ \bf59}
 (1986), 44; solution in {\bf 60} (1987), 46--48.
\p *Q78. (with Jon Bentley)\xskip Programming Pearls:
 A {\tt WEB} program for sampling. {\sl Communications of the ACM\/ \bf 29},\,5
 (May 1986), 364--369. Reprinted as Chapter~5 of {\sl Literate Programming\/}
 (see under Books).
\p *Q79. (with Jon Bentley and M. Douglas McIlroy)\xskip Programming Pearls:
 A {\tt WEB} program for common words. {\sl Communications of the ACM\/
 \bf29},\,6 (June 1986), 471--483. Reprinted as Chapter~6 of
 {\sl Literate Programming\/} (see under Books).
\p Q80. Solution to problem 6480, A Catalonian sum.
 {\sl American Mathematical Monthly\/ \bf93} (1986), 220.
\p Q81. Problem 86-2, a random knockout tournament (with solution).
 {\sl SIAM Review \bf29} (1987), 127--129.
\p Q82. Theory and practice. {\sl Bulletin of the EATCS\/ \bf27}
 (October 1985), 14--21.
 Greek translation by N.~Kasi\-m\'atis in {\sl Mathimatik{\'\i}
 Epithe\'orisi\/}
 (Bulletin of Greek Mathematical Society) te\'ukhos 30 (1986), 3--15.
 Reprinted with corrections as Chapter~7 of {\sl Selected Papers on
 Computer Science\/} (see under Books).
\p Q83. Problem E3106, A curious sum for Euler's totient function.
 {\sl American Mathematical Monthly\/ \bf92} (1985), 590; solution in
 {\bf94} (1987), 795--797.
\p Q84. Foreword to {\sl The Kermit File Transfer Protocol\/} by
 Frank da Cruz (Bedford, Mass.: Digital Press, 1987), p.~xi.
\p *Q85. (interview by G. Michael Vose and Gregg Williams) Text
 Processing: Computer Science considerations. {\sl Byte\/ \bf 11},\,2
 (February 1986),  169--172.
\p Q86. Remarks to celebrate the publication of {\sl Computers \&
 Typesetting}. {\sl TUGboat\/ \bf7} (1986), 95--98.
 Re\-printed as Chapter~28 of {\sl Digital Typography\/} (see under Books).
\p Q87. Solution to problem E3061, Empty cells.
 {\sl American Mathematical Monthly\/ \bf94} (1987), 189.
\p Q88. (with Pierre MacKay)\xskip Mixing right-to-left texts with
 left-to-right texts. {\sl TUGboat \bf 8} (1987), 14--25.
 Reprinted as Chapter~4 of {\sl Digital Typography\/} (see under Books).
\p Q89. Solution to problem E3062, A versatile identity.
 {\sl American Mathematical Monthly\/ \bf94} (1987), 376--377.
\p Q90. The \TeX\ logo in various fonts. {\sl TUGboat\/ \bf 7} (1986), 101.
 Reprinted as Chapter~6 of {\sl Digital Typography\/} (see under Books).
\p Q91. Macros for Jill. {\sl TUGboat\/ \bf8} (1987), 309--314.
 Reprinted as Chapter~8 of {\sl Digital Typography\/} (see under Books).
\p Q92. Problem for a Saturday morning. {\sl TUGboat\/ \bf 8} (1987), 73, 210.
 Reprinted as Chapter~9 of {\sl Digital Typography\/} (see under Books).
\p Q93. Fonts for digital halftones. {\sl TUGboat\/ \bf8} (1987), 135--160.
 Reprinted with revisions as Chapter~21 of {\sl Digital Typography\/}
 (see under Books).
\p Q94. A punk meta-font. {\sl TUGboat\/ \bf9} (1988), 152--168.
 Reprinted as Chapter~20 of {\sl Digital Typography\/} (see under Books).
\p Q95. Response to the Steele Prize. {\sl Notices of the Amer.\ Math.\ Soc.\
 \bf34} (1987), 227--228.
\p Q96. Exercises for {\sl \TeX: The Program}. {\sl TUGboat\/ \bf11}
 (1990), 165--170, 499--511.
 Reprinted as Chapter~10 of {\sl Digital Typography\/} (see under Books).
\p Q97. The difference between art and science. {\sl Reader's Digest\/}
 (July 1987), 24.
\p Q98. Printing out selected pages. {\sl TUGboat\/ \bf8} (1987), 217.
 Reprinted as Chapter~7 of {\sl Digital Typography\/} (see under Books).
\p Q99. N-ciphered texts. {\sl Word Ways\/ \bf20} (1987), 173--174, 191--192.
\p Q100. Solution to problem E3166, A polynomial identity.
 {\sl American Mathematical Monthly\/ \bf95} (1988), 662--663.
\p Q101. Problem 1280: A sum of floors. {\sl Mathematics Magazine\/ \bf60}
  (1987), 329; solution in {\bf61} (1988), 319--320.
\p Q102. Algorithmic themes. {\sl A Century of Mathematics in America},
  Peter~L. Duren, ed., {\bf1} (Providence, R.I.:
  American Mathematical Society, 1988), 439--445.
 Reprinted with corrections as Chapter~5 of {\sl Selected Papers on
 Computer Science\/} (see under Books).
\p Q103. Introduction to {\sl Mathematical Circus\/} by Martin Gardner,
 MAA Spectrum edition (Washington, D.C.: Mathematical Association of
 America, 1992), xi--xii.
\p Q104. Notes on the errors of \TeX. {\sl TUGboat\/ \bf10} (1989), 529--531.
 Revised version, entitled ``Learning from our errors,'' in
 {\sl Software Development and Reality Construction}, edited by
 Christiane Floyd, Heinz Z\"ullighoven, Reinhard Budde, and Reinhard
 Keil-Slawik (Berlin: Springer-Verlag, 1992), 28--30.
\p Q105. (with Barry Hayes and Carlos Subi)\xskip Elementary problem E3267,
 a solitaire game.
 {\sl American Mathematical Monthly\/ \bf 95} (1988), 456--457;
 solution in {\bf100} (1993), 292--294.
\p Q106. (with Ilan Vardi)\xskip Advanced problem 6581, the asymptotic
 expansion of the middle binomial coefficient.
 {\sl American Mathematical Monthly\/ \bf95} (1988), 774. Solution,
 {\sl American Mathematical Monthly\/ \bf97} (1990), 629--630.
\p Q107. Typesetting Concrete Mathematics. {\sl TUGboat\/ \bf10} (1989),
 31--36; errata, p.~342.
 Reprinted as Chapter~18 of {\sl Digital Typography\/} (see under Books).
\p *Q108. (with Jill C. Knuth)\xskip \TeX. {\sl Encyclopedia of Computer
 Science}, third edition, edited by Anthony Ralston
 and Edwin~D. Reilly (New York: Van Nostrand Reinhold, 1993), 1353--1355.
 Fourth edition, edited by Anthony Ralston, Edwin~D. Reilly, and David
 Hemmendinger (London:\ Nature Publishing Group, 2000), 1756--1759.
 {\sl Concise Encyclopedia of Computer Science}, edited by Edwin~D.
 Reilly (Chichester:\ John Wiley \& Sons, 2004), 749--751.
\p *Q109. (with Jill C. Knuth)\xskip \MF. {\sl Encyclopedia of Computer
 Science}, third edition, edited by Anthony Ralston
 and Edwin~D. Reilly (New York: Van Nostrand Reinhold, 1993), 869--870.
 Fourth edition, edited by Anthony Ralston, Edwin~D. Reilly, and David
 Hemmendinger (London:\ Nature Publishing Group, 2000), 1154--1155.
\p Q110. Elementary problem E3335, a deranged recurrence.
 {\sl American Mathematical Monthly\/ \bf 96} (1989), 525;
 solution in {\bf97} (1990), 927.
\p Q111. Solution to problem 6575, an identity involving sums and
 products. {\sl American Mathematical Monthly\/ \bf 97} (1990), 256.
\p Q112. The new versions of \TeX\ and \MF. {\sl TUGboat\/ \bf10} (1989),
 325--328. Erratum, {\sl TUGboat\/ \bf11} (1990), 12. Reprinted in
 {\sl Die \TeX nische Kom\"odie\/ \bf2},\,1 (March 1990), 16--22.
 French translation by Alain Cousquer, ``\TeX\ 3.0 ou le \TeX\ nouveau
 va arriver,'' {\sl Cahiers GUTenberg}, N$\,^\circ$~4 (December 1989), 39--45.
 Reprinted as Chapter~29 of {\sl Digital Typography\/} (see under Books).
\p Q113. Virtual fonts: More fun for Grand Wizards. {\sl TUGboat\/ \bf11}
 (1990), 13--23.
 Reprinted as Chapter~12 of {\sl Digital Typography\/} (see under Books).
\p Q114. The genesis of attribute grammars. {\sl Lecture Notes in Computer
 Science \bf461} (1990), 1--12. % WAGA conference proceedings, Paris 1990
 Reprinted as Chapter~19 of {\sl Selected Papers on
 Computer Languages\/} (see under Books).
\p Q115. Memories of Andrei Ershov. {\sl Programmirovanie\/ \bf 16},\,1
 (1990), 113--114.
 Russian translation in {\sl Andre{\u\i} Petrovich Ershov -- ycheny{\\u\i}
 i chelobek\/} (Novosibirsk: Siberian branch, Russian Academy of Science,
 2006), 263--265. (Several letters from Knuth to Ershov and vice-versa, from
 1970, 1976, 1977, and 1978, are also translated in this volume.)
\p Q116. Solution to Problem 79-5, Asymptotic behavior of a sequence.
 {\sl SIAM Review\/ \bf22} (1980), 101--102.
\p Q117. Arthur Lee Samuel. {\sl TUGboat \bf11} (1990), 497--498.
\p Q118. The future of \TeX\ and \MF. {\sl TUGboat \bf11} (1990), 489.
 Reprinted in {\sl Nederlandstalige \TeX\ Gebruikersgroep MAPS\/ \bf90.2}
 (May 1990), 145.
 Reprinted in {\sl \TeX line\/ \bf12} (London: December 1990), 1.
 Reprinted in {\sl Die \TeX nische Kom\"odie\/ \bf2},\,4 (December 1990),
 23--25.
 French translation by \'Eric Picheral, ``L'avenir de \TeX\ et de \MF,'' {\sl
 Cahiers GUTenberg}, N$^\circ$~8 (March 1991), 1--2.
 Reprinted as Chapter~30 of {\sl Digital Typography\/} (see under Books).
\p Q119. (with Boris Pittel)\xskip Elementary problem E3411,
 two sums over compositions.
 {\sl American Mathematical Monthly\/ \bf 97} (1990), 916--917;
 solution in {\bf 99} (1992), 578--579.
\p Q120. Elementary problem E3303, a binary summation.
 {\sl American Mathematical Monthly\/ \bf 96} (1989), 54.
 Solution, {\sl American Mathematical Monthly\/ \bf 97} (1990), 348--349.
\p Q121. Elementary problem E3309, a binomial coefficient inequality.
 {\sl American Mathematical Monthly\/ \bf 96} (1989), 154;
 solution in {\bf 97} (1990), 614.
\p Q122. (with Philippe Flajolet)\xskip Elementary problem E3415,
 a hypergenerating function. {\sl American Mathematical Monthly\/ \bf 98}
 (1991), 54; solution in {\bf 100} (1993), 84--85.
\p Q123. Advanced problem 6649, a generalized gamma function with
 independent branches.
 {\sl American Mathematical Monthly\/ \bf 98} (1991), 168;
 solution in {\bf 101} (1994), 77--78.
\p Q124. (with John McCarthy)\xskip Elementary problem E3429, small pills.
 {\sl American Mathematical Monthly\/ \bf 98} (1991), 264;
 solution in {\bf 99} (1992), 684.
\p Q125. Elementary problem E3463, points in a circle.
 {\sl American Mathematical Monthly\/ \bf 98} (1991), 852;
 solution in {\bf100} (1993), 693--694.
\p Q126. Computer programming and computer science. {\sl Academic Press
 Dictionary of Science and Technology\/} (Harcourt Brace Jovanovich, 1992),
 490. Reprinted with corrections in {\sl Selected Papers on Computer Science\/}
 (see under Books), 2--3.
\p Q127. (with Lee Sallows)\xskip Problem 1296, Universal magic squares.
 {\sl Journal of Recreational Mathematics\/ \bf 16} (1984), 138;
 solution in {\bf17} (1985), 145--146.
\p Q128. Introduction to {\sl New Book of Puzzles\/} by Jerry Slocum and
 Jack Botermans (New York: W.~H. Freeman, 1992), 6--7.
\p Q129. An interview with Donald Knuth (by Roswitha Graham and Barbara
 Beeton). {\sl TUGboat\/ \bf13} (1992), 419--425.
\p Q130. Icons for \TeX\ and \MF. {\sl TUGboat\/ \bf14} (1993), 387--389.
 Reprinted as Chapter~27 of {\sl Digital Typography\/} (see under Books).
\p Q131. $5\times5\times5$ word cubes by computer. {\sl Word Ways\/ \bf26}
 (1993), 95--97.
\p Q132. The Stanford GraphBase: A platform for combinatorial algorithms.
 {\sl Proceedings of the Fourth Annual ACM--SIAM Symposium on Discrete
 Algorithms\/} (1993), 41--43.
\p Q133. This Week's Citation Classic: Artistic programming. {\sl
 Current Contents}, Physical, Chemical \& Earth Sciences {\bf33},\,34
 (August 23, 1993), 8; also {\sl
 Current Contents}, Engineering, Technology \& Applied Sciences {\bf24},\,34
 (August 23, 1993), 8.
 Reprinted as Chapter~14 of {\sl Selected Papers on
 Computer Science\/} (see under Books); in the first two printings
 it was, however, called Chapter~15.
\p Q134. (with John Hershberger) Problem 85-3, on merging sequences.
 {\sl Journal of Algorithms\/ \bf 6} (1985), 284.
\p Q135. Problem 90-1, reversing the transformation from sequential
 representation to short codes for adjacency lists for undirected graphs.
 {\sl Journal of Algorithms\/ \bf12} (1991), 183.
\p Q136. Problem 10280, a random binary operation.
 {\sl American Mathematical Monthly\/ \bf100} (1993), 76;
 solution in {\bf102} (1995), 561--562.
\p Q137. Problem 10298, a divisibility property of Stirling numbers.
 {\sl American Mathematical Monthly\/ \bf100} (1993), 400;
 solution in {\bf103} (1996), 80--81.
\p Q138. Problem 10401, a knight's surprise.
 {\sl American Mathematical Monthly\/ \bf101} (1994), 682--683;
 solution in {\bf104} (1997), 669.
\p *Q139. ETAOIN SHRDLU non-crashing sets. {\sl Word Ways \bf27} (1994), 138.
\p Q140. Speech upon receiving honorary degree from St.~Petersburg University.
 {\sl Programming and Computer Software\/ \bf20} (1994), 290.
 Russian translation by B. B. Pokhodze\u\i, {\sl Programmirovanie\/
 \bf 20},\,6 (1994), 89--91.
 Reprinted as Chapter~15 of {\sl Selected Papers on
 Computer Science\/} (see under Books), but not in the first two printings
 of that volume.
\p Q141. Foreword to {\sl An Introduction to the Analysis of Algorithms\/} by
 Robert Sedgewick and Philippe Flajolet (Reading, Mass.: Addison-Wesley,
 1995), v.
\p Q142. The Chinese domino challenge. {\sl Math Horizons\/} (April 1995),
 8--9.
\p *Q143. (with Nob Yoshigahara) Pentagon puzzle (Japanese). {\sl Quark Visual
 Science Magazine}, No.~156 (Tokyo: Kodansha, June 1995), 127.
\p Q144. Predictions for the year 2000, on programming. {\sl Byte\/
 \bf20},\thinspace9 (September 1995), 110.
\p Q145. Foreword to $A=B$ by Marko Petkov\v{s}ek, Herbert S. Wilf, and
 Doron Zeilberger (A K Peters, 1996), ix.
\p Q146. Open letter to coordinators of \TeX\ implementations, 13 October 1981.
 {\sl TUGboat\/ \bf2},\,3 (November 1981), 5--6.
\p Q147. A note on hyphenation. {\sl TUGboat\/ \bf4} (1983), 64.
\p Q148. It happened. {\sl TUGboat\/ \bf8} (1987), 6.
\p Q149. The initial reception of {\sl Concrete Mathematics}.
 {\sl SIGACT News\/  \bf20},\,1 (Winter 1989), 48.
\p Q150. The Samson-Mueller (Davis-Putnam) algorithm. {\sl SIGACT News\/
 \bf9},\,1 (January--March 1977), 8--9.
\p Q151. Problem 10470, minimal covers.
 {\sl American Mathematical Monthly\/ \bf102} (1995), 655;
 solution in {\bf105} (1998), 771--773.
\p Q152. Problem 1479, a recursive optimization.
 {\sl Mathematics Magazine\/ \bf68} (1995), 306;
 solution in {\bf69} (1996), 306.
\p Q153. TUG'95 Questions and answers with Prof.~Donald E. Knuth (edited by
 Christina Thiele) {\sl TUGboat\/ \bf17} (1996), 7--22.
 Reprinted in {\sl GUST\/ \bf8} (1997), 9--23.
 Reprinted as Chapter~31 of {\sl Digital Typography\/} (see under Books).
\p Q154. An interview with Donald Knuth (by Jack Woehr). {\sl Dr.\
 Dobb's Journal\/ \bf21},\,4 (April 1996), 16--18, 20, 22.
\p Q155. Knuth meets NTG members (edited by Christina Thiele) {\sl MAPS:
 Minutes and APpendiceS\/ \bf16} (Nederlandstalige \TeX\ Gebruikersgroep,
 1996), 38--49. Reprinted in {\sl TUGboat\/ \bf17} (1996), 342--355.
 Reprinted as Chapter~33 of {\sl Digital Typography\/} (see under Books).
\p Q156. Questions and answers at Charles University (edited by Barbara Beeton
 and Christina Thiele). {\sl TUGboat\/ \bf17} (1996), 355--367.
 Reprinted as Chapter~32 of {\sl Digital Typography\/} (see under Books).
\p Q157. Problem 10546, binomial coefficient parity.
 {\sl American Mathematical Monthly\/ \bf103} (1996), 695;
 solution in {\bf105} (1998), 867--868.
\p Q158. Problem 10568, subtracting square roots repeatedly.
 {\sl American Mathematical Monthly\/ \bf104} (1997), 68;
 solution in {\bf106} (1999), 167.
\p Q159. Problem 10576, a card-matching game.
 {\sl American Mathematical Monthly\/ \bf104} (1997), 169;
 solution in {\bf106} (1999), 168--169.
\p Q160. Problem 97-6, A sum over binary sequences. {\sl SIAM Review\/ \bf39}
 (1997), 317; solution in {\bf40} (1998), 372--374.
\p Q161. Crystallization of algorithms: The Art of Computer Programming.
 [Interview by Makoto Nagao; in Japanese.] {\sl Computer Today\/} no.~77
 (January 1997), 46--51.
\p Q162. Concerns about American technology policy. [Interview by Hisayasu
 Yoshizawa; in Japanese.] {\sl Nikkei Electronics\/} no.~683 (24 February
 1997), 145--148.
\p Q163. Dr.~Knuth meets Mitsumasa Anno. [Discussion between Mitsumasa Anno,
 Donald E. Knuth, and Akihiro Nozaki; in Japanese.] {\sl Sugaku Seminar\/
 \bf36},\,3 (March 1997), 40--44.
\p Q164. Roundtable discussion with Prof.~Knuth. [Discussion between
 Makoto Arisawa, Toshiaki Kurokawa, Nobuko Kishi, and Donald E. Knuth,
 with additional comments by Jill C. Knuth; in Japanese.]
 {\sl bit\/ \bf29},\,4 (April 1997), 46--51.
\p Q165. Opinion: Harmony between theory and practice. [In Japanese.]
 {\sl bit\/ \bf29},\,5 (May 1997), 3.
\p*Q166. (with R. M. Corless and D. J. Jeffrey)\xskip A sequence of series
 for the Lambert $W$ function. {\sl Proceedings of the International
 Symposium on Symbolic and Algebraic Computation ISSAC '97\/} (ACM Press,
 1997), 197--204.
\p Q167. Problem 1534, sums of ceilings of floors.
 {\sl Mathematics Magazine\/ \bf70} (1997), 381;
 solution in {\bf71} (1998), 390--391.
\p Q168. Problem 1539, a sharp tail inequality.
 {\sl Mathematics Magazine\/ \bf71} (1998), 66;
 solution in {\bf72} (1999), 65--66.
\p Q169. Problem 10593, matrices related to universal hashing.
 {\sl American Mathematical Monthly\/ \bf104} (1997), 456;
 solution in {\bf106} (1999), 473--474.
\p Q170. Problem 10609, a partial Abelian sum.
 {\sl American Mathematical Monthly\/ \bf104} (1997), 664;
 solution in {\bf106} (1999), 690--691.
\p Q171. Teach calculus with Big $O$ (letter to the editor).
 {\sl Notices of the American Mathematical Society\/ \bf45},\thinspace6
 (June/July 1998), 687--688. % online search for Bachmann will find it!
\p Q172. Biblical ladders. {\sl The Mathematician and
 Pied Puzzler\/} (Wellesley, Mass.:\ A K Peters, 1999), 29--34.
\p Q173. An interview with Donald Knuth: ``A little bit of your soul in it,''
 (interview by John Boe), {\sl Writing on the Edge\/} {\bf9} (1998), 10--25.
\p Q174. Solution to problem 10424, a sum of Ira Gessel. {\sl American
 Mathematical Monthly\/ \bf104} (1997), 467.
\p Q175. Letter to the editor: Teach calculus with Big $O$.
 {\sl Notices of the American Mathematical Society\/ \bf45} (1998), 687--688.
\p Q176. Solution to problem 97-19, three binomial convolutions. {\sl SIAM
 Review\/ \bf 40} (1998), 991.
\p Q177. (with Vaughan Pratt) Problem 10689, an algebraic definition
 of the real numbers. {\sl American Mathematical Monthly\/ \bf105} (1998), 769;
 solution in {\bf107} (2000), 755.
\p Q178. Problem 10691, highly variable lists. %cyclic minimaxing.
 {\sl American Mathematical Monthly\/ \bf105} (1998), 859;
 solution in {\bf110} (2003), 59--60.
\p Q179. (with P. Spirakis, C. Papadimitriou, and others) HERCMA 2001 ---
 Round table discussion. In {\sl Hellenic European Research on Computer
 Mathematics and Its Applications (HERCMA 2001)}, edited by Elias~A.
 Lipitakis (Athens:\ LEA, 2002), 910--923.
\p Q180. Problem 10720, exploring all binary mazes.
 {\sl American Mathematical Monthly\/ \bf106} (1999), 264;
 solution in {\bf110} (2003), 60--61.
\p Q181. Problem 10726, explosive growth.
 {\sl American Mathematical Monthly\/ \bf106} (1999), 362;
 solution in {\bf107} (2000), 469--470.
\p Q182. (with Robert W. Floyd) Problem H-94, golden hashing.
 {\sl Fibonacci Quarterly\/ \bf4} (1966), 258. % added for completeness!
\p Q183. Problem 10832, the reciprocals of Stirling's errors,
 {\sl American Mathematical Monthly\/ \bf107} (2000), 863;
 solution in {\bf108} (2001), 877--878.
\p Q184. Interview: Donald E. Knuth (by Raph Levien). {\sl TUGboat\/ \bf21}
 (2000), 103--110. Reprinted from the original online version,
 {\tt http:/\kern-.1em/www.advogato.org/ article/28.html}.
\p Q185. Problem 10858, Fibonacci sequences with complex twists.
 {\sl American Mathematical Monthly\/ \bf108} (2001), 271;
 solution in {\bf111} (2004), 166--167, 922.
\p Q186. Problem 1621, Fibonacci numbers from binomial coefficients.
 {\sl Mathematics Magazine\/ \bf74} (2001), 154;
 solution in {\bf75} (2002), 149--150.
\p Q187. Problem 10871, balanced neighborhood squares.
 {\sl American Mathematical Monthly\/ \bf108} (2001), 372;
 solution in {\bf100} (2003), 161--162.
\p Q188. Problem 10875, animals in a cage.
 {\sl American Mathematical Monthly\/ \bf108} (2001), 469;
 solution in {\bf110} (2003), 243--245.
\p Q189. All questions answered (edited by Allyn Jackson). [Transcript
 of a lecture at the Technical University of Munich, 5~October 2001.]
 {\sl Notices of the American Mathematical Society\/ \bf49},\,3 (March
 2002), 318--324. Reprinted in {\sl Mathematics Newsletter\/ \bf12}
 (Ramanujan Mathematical Society, 2002), 33--42.
\p Q190. Problem 10906, recounting the rationals.
 {\sl American Mathematical Monthly\/ \bf108} (2001), 872;
 solution in {\bf110} (2003), 642--643.
\p Q191. (with O. P. Lossers) Solution to problem 10757, generalized quotients
 of continued fractions. {\sl American Mathematical Monthly\/ \bf108}
 (2001), 875.
\p Q192. Problem 10913, related transpositions with different periods,
 {\sl American Mathematical Monthly\/ \bf108} (2001), 977;
 solution in {\bf110} (2003), 844--845.
\p Q193. Knuth comments on code. {\sl Byte\/
 \bf21},\thinspace9 (September 1996), 40.
\p Q194. Der Perfektionist (interview by Harald B\"ogeholz and Andreas
 Stiller). {\sl c't magazin f\"ur computer technik 2002},\,5
 (25 February--10 March 2002), 190--193.
\p Q195. `Geleitwort' to {\sl Das {\sltt MMIX\/}-Buch\/} by
 Heidi Anlauff, Axel B\"ottcher, and Martin Ruckert
 (Berlin: Springer, 2002), v--vi.
\p Q196. U.K. TUG, Oxford, Sunday, 12 September 1999, question \& answer
 session with Donald Knuth. {\sl TUGboat\/ \bf22} (2001), 15--19.
\p Q197. All questions answered. [Transcript
 of a lecture at the University of Oslo, 30 August 2002.]
 {\sl TUGboat\/ \bf23} (2002), 249--261.
\p Q198. Solution to problem 10825, A Fibonacci--Lucas extremum.
 {\sl American Mathematical Monthly\/ \bf109} (2002), 762--763.
\p Q199. Problem 10985, Some Bernstein polynomials.
 {\sl American Mathematical Monthly\/ \bf110} (2003), 58;
 solution in {\bf111} (2004), 447, 922.
\p Q200. Problem 11021, a modular triple.
 {\sl American Mathematical Monthly\/ \bf110} (2003), 542, 963;
 solution in {\bf112} (2005), 279--280.
\p Q201. Robert W Floyd, in memoriam. {\sl SIGACT News\/ \bf34},\thinspace4
 (December 2003), 3--13. Reprinted in {\sl IEEE Annals of the History of
 Computing\/ \bf26},\thinspace2 (April--June 2004), 75--83.
\p Q202. (with David S. Johnson and Zvi Galil) Changes at the {\sl Journal
 of Algorithms}. {\sl SIGACT News\/ \bf35},\thinspace1 (March 2004), 85.
\p Q203. Problem 11078, cube-free sums.
 {\sl American Mathematical Monthly\/ \bf111} (2004), 361;
 solution in {\bf113} (2006), 368--369.
\p Q204. Three Catalan bijections. {\sl Institut Mittag-Leffler Reports},
 No.~04, 2004/2005, Spring (2005), 19pp.
\p Q205. Problem 11142, Largest weighted Stirling numbers.
 {\sl American Mathematical Monthly\/ \bf112} (2005), 273--274;
 solution to appear.
\p Q205. Problem 11151, Affinity groups at a roundtable.
 {\sl American Mathematical Monthly\/ \bf112} (2005), 367;
 solution to appear.
\p Q206. Interview with Donald E. Knuth (by Gianluca Pignalberi).
 {\sl Free Software Magazine}, Issue~7 (August 2005), 13--15.
 Reprinted in {\sl TUGboat\/ \bf26} (2005), 183--185.
 Italian translation in {\sl Ars{\TeX}nica\/ \bf1} (2006), 5--7.
\p Q207. Problem 1721, Fibonacci graphs. {\sl Mathematics Magazine\/ \bf78}
 (2005), 239; solution in {\bf79} (2006), 219--220.
\p Q208. Searching graphs (a brainteaser). {\sl ACM Transactions on
 Algorithms\/ \bf1} (2005), 158---159; solution in {\bf2} (2006), 132--133.
\p Q209. $\hbox{\tt KNIFE}+\hbox{\tt FORK}+\hbox{\tt SPOON}+\hbox{\tt SOUP}=
 \hbox{\tt SUPPER}$. {\sl Journal of Recreational Mathematics\/ \bf33}
 (2004--2005), 67.
\p Q210. \TeX's infinite glue is projective. {\sl TUGboat\/ \bf27} (2006),
 to appear.
\p Q211. Problem 11243, Perfect parity patterns.
 {\sl American Mathematical Monthly\/ \bf113} (2006), 759;
 solution to appear.
\p Q212. Problem 11264, $d$-swaps.
 {\sl American Mathematical Monthly\/ \bf114} (2007), 77;
 solution to appear.


\finishpage

%%%%%
\sectionbegin 5. {Reports of Limited Circulation \rm
($\ast$ means notes prepared by auditors of lectures)}
\p R1.  Tic Tac Toe on the 650.  Case Computing Center
 (Cleveland, Ohio, 1957), 8 pp.
\p R2.  Case Soap III\null.  Case Computing Center, ser.\ IV, {\bf 1}
 (Cleveland, Ohio, February 1958), 28 pp.  
\p R3.  Runcible I.  Case Computing Center, ser.\ V, {\bf 1}
 (Cleveland, Ohio, March 1959), 67 pp.  
\p R4.  (Editor)  {\sl Engineering and Science Review}.  
 Case Institute of Technology, vol.\ 2, no.\ 1, 3, 4 (1959).  Associate
 Editor, vol.\  3 (1960).  Article: ``The revolutionary potrzebie''
 (November 1958), 18--20. Features: ``Th$_5$E$_4$ CH$_3$EmIC$_2$Al$_2$
 Ca$_3$P\kern-1pt$_4$Er'' (March 1959), p.\ 32;
 ``The plot thickens,'' (November 1959), p.\ 45;
 ``Math ace,'' (May 1960), p.\ 24.
\p R5.  (Editor)  {\sl Case Handbook}.  Case
 Institute of Technology, 1959.  
\p R6.  SuperSoap.  Case Computing Center, 
 ser.\ IV, {\bf 2} (Cleveland, Ohio, August 1959), 55 pp.  
\p R7.  The internals of Algol 205.  Burroughs Corporation, 1960.  30 pp.  
\p R8.  Balgol 220 (annotations on listing).  Burroughs Corporation, 1960.  
\p R9.  $m$th powers of algebraic roots.  California Institute of
 Technology, Mathematics Department (Pasadena, California, April 1961). 7 pp.  
\p R10.  Burroughs Algebraic Compiler for the 205.  Burroughs
 Corporation publication no.\ 205-21003-D (Detroit, Michigan,
 October 1961), 30 pp.  
\p R11.  FORTRAN II for the Univac Solid State computers.  UNIVAC
 Division Sperry Rand (Bluebell, Pennsylvania, October 1962), 85 pp.
\p R12. (with William C. Lynch)\xskip QADAAD (Quick And Dirty Assembler And
 Documentor), an assembly system for the Solid State II\null. Internal
 memorandum, August, 1962.
\p R13. (with William C. Lynch)\xskip
 FORTRAN II Routine Block Chart (Annotated).  UNIVAC Division Sperry
 Rand, publication no.\ UP-3843.1 (Bluebell, Pennsylvania, 1963), 50 pp.  
\p R14.  A good scrambling function suitable for hardware.  Burroughs
 Corporation Electrodata Division, Engineering Tech.\ Memo.\ No.\ 234
 (Pasadena, California, October 1963), 22 pp.  
\p R15.  Textbook of Combinatorial Mathematics (a partial translation
 of {\sl Lehrbuch der Combinatorik}, by E. Netto). 
 Library, California Institute of Technology (1963), approx.\ 150 pp.  
\p R16.  Finite semifields and projective planes.
 Ph.D. thesis, California Institute of Technology (Pasadena, California, 1963),
 70 pp.
\p R17.  Computer languages.  California Institute of Technology,
 Seminar series on computer applications to biology
 (Pasadena, California, February 1963), 15 pp.;
 (December 1964), 12 pp.  
\p R18.  Notes on Chebyshev approximation theory.  California Institute
 of Tech\-nology, Mathematics Department (Pasadena, California, 1964).  7 pp.
\p R19.  Lectures in software design.  Burroughs Corporation
 Electrodata Division (Pasadena, California, 1964), approx.\ 200 pp.  
\p R20.  Notes on complex variable theory.  California Institute of
 Technology, Mathematics  Department (Pasadena, California, 1965), 36 pp.
\p R21.  The Thue-Siegel-Roth theorem.  California Institute of
 Technology, Mathematics Department (Pasadena, California, 1966).  41 pp.  
\p R22.  Exploration of the direct product (Kronecker product,
 tensor product) of matrices:  Results of Math 5B class project.
 California Institute of Technology, Mathematics Department
 (Pasadena, California, March 1966), 10 pp.  
\p R23.  Estimating the running time of BACKTRACK programs. 
 IDA-CRD Working Paper No.\ 279 (Princeton, New Jersey,
 November 1969), 42 pp.  
\p R24.  The Art of Computer Programming --- errata et addenda. 
 Stanford Computer Science Report 194 (Stanford, California,
 January 1971), 28 pp.  
\p R25.  (with R. L. Sites)\xskip  MIX/360 User's Guide.  
 Stanford Computer Science Report 197 (Stanford, California, March 
 1971), 11 pp.  
\p *R26.  The analysis of algorithms.  In {\sl The Teaching
 of Programming at University Level}, B. Shaw, ed. (University of
 Newcastle-Upon-Tyne, 1971), 49--62.  
\p R27.  (with V. Chv\'atal and D. A. Klarner)\xskip  Selected
 combinatorial research problems.  Stanford Computer Science
 Report 292 (Stanford, California, June 1972), 29 pp.  
\p R28.  Matroid partitioning.  Stanford Computer Science Report 342
 (Stanford, California, 1973), 12 pp.
\p *R29.  Selected topics in Computer Science.  Lecture Note Series,
 Matematisk institutt, Universitetet i~Oslo (Oslo, Norway, 1973),
 Nr.~1 and Nr.~2. [Notes by Ole Amble, Ole-Johan Dahl, Erik Holb{\ae}k-Hanssen,
 Arne Jonassen, Torvald Kjeldaas, Stein Krogdahl, \AA mund Lunde,
 Arne Maus, and Arne Wang. Part~1 contains: analysis of quicksort; dynamic
 storage allocation; flowcharts and Kirchhoff's first law; theory of matroids;
 complexity analysis of equivalence algorithms. Part~2 contains: strong
 components; hard problems; backtracking algorithms; pattern matching in
 strings; generation of combinatorial patterns.]
\p R30.  A review of {\sl Structured Programming}.  Stanford Computer
 Science Report 371 (Stanford, California, June 1973).  
\p *R31.  ``Stable marriage'' --- problemet og samanhangen med hashing og
 ``coupon collecting''.  (Norwegian) University of Bergen (Bergen, Norway,
 June 1973), 20 pp.  
\p R32.  Sorting and Searching --- errata and addenda.  Stanford Computer
 Science Report 392 (Stanford, California, October 1973), 35 pp.  
\p R33.  The State of {\sl The Art of Computer Programming}.  
 Stanford Computer Science Report 551 (Stanford, California,
 June 1976), 57 pp.  
\p *R34.  (with Michael J. Clancy)\xskip A programming and
 problem-solving seminar.  Stanford Computer Science Report 606
 (Stanford, California, April 1977), 99 pp.
\p R35.  Tau Epsilon Chi, a system for technical text.  Stanford Computer
 Science Report 675 (Stanford, California, September 1978), 198 pp.
 Reprinted with corrections as part 2 of {\sl \TeX\ and \slMF\/}
 (see under Books).
\p *R36.  (with Chris van Wyk)\xskip A programming and problem-solving
 seminar.  Stanford Computer Science Report 707 (Stanford, California,
 January 1979), 83 pp.
\p R37.  The errata of computer programming.  Stanford Computer Science
 Report 712 (Stanford, California, January 1979), 57 pp. Reprinted in
 {\sl Dr.~Dobb's Journal of Computer Calisthenics \&\ Orthodontia\/ \bf 5},\,6
 (June 1980), 27--39.
\p R38.  \MF, a system for alphabet design.  Stanford Computer Science
 Report 762 (Stanford, California, September 1979), 105~pp.  Reprinted with
 corrections as part 3 of {\sl \TeX\ and \slMF\/} (see under Books).
\p R39.  The Computer Modern family of typefaces.  Stanford Computer Science
 Report 780 (Stanford, California, January 1980), 406 pp.
\p *R40. (with Allan A. Miller)\xskip A programming and
 problem-solving seminar. Stanford Computer Science Report 863
 (Stanford, California, June 1981), 81 pp.
\p R41. The last whole errata catalog. Stanford Computer Science Report 868
 (Stanford, California, July 1981), 41 pp.
\p R42. The {\tt WEB} system of structured documentation. Stanford Computer
 Science Report 980 (Stanford, California, September 1983), 206~pp. Most of
 this report has been reprinted in {\sl Weaving a Program: Literate
 Programming in {\sltt WEB}\/} by Wayne Sewell (New York: Van Nostrand
 Reinhold, 1989), 271--434.
\p R43. (with David R. Fuchs)\xskip \TeX ware. Stanford Computer Science
 Report 1097 (Stanford, California, April 1986), $10+30+53+53$~pp.
\p R44. A torture test for \TeX. Stanford Computer Science Report 1027
 (Stanford, California, November 1984), 142~pp.
\p *R45. (with Joseph S. Weening)\xskip A programming and problem-solving
 seminar. Stanford Computer Science Report 989 (Stanford, California,
 December 1983), ii+91~pp.
\p *R46. (with John D. Hobby)\xskip A programming and problem-solving seminar.
 Stanford Computer Science Report 990 (Stanford, California, December 1983),
 61~pp.
\p *R47. (with Ramsey W. Haddad)\xskip A programming and problem-solving
 seminar. Stanford Computer Science Report 1055 (Stanford, California,
 June 1985), 103~pp.
\p R48. A torture test for \MF. Stanford Computer Science Report 1095
 (Stanford, California, January 1986), 78~pp.
\p *R49. (with Tomas G. Rokicki)\xskip A programming and problem-solving
 seminar. Stanford Computer Science Report 1154 (Stanford, California,
 April 1987), 89~pp.
\p *R50. (with Tracy L. Larrabee and Paul M. Roberts)\xskip Mathematical
 writing. Stanford Computer Science Report 1193 (Stanford, California,
 January 1988), 117~pp. Reprinted with corrections by the Mathematical
 Association of America (see under Books).
\p R51. 3\,:\,16, an approach to Bible study. In
 {\sl A Sixth Conference on Mathematics from a Christian Perspective},
 edited by Robert L. Brabenec (proceedings of a conference at Calvin College,
 May 1987, sponsored by the Association of Christians in the Mathematical
 Sciences), 3--25.
\p R52. (with Tomas G. Rokicki and Arthur L. Samuel)\xskip \MF\/ware.
 Stanford Computer Science Report 1255
 (Stanford, California, April 1989), $30+42+87+48$~pp. Reprinted in
 {\sl \TeX niques\/ \bf13} (1990).
\p R53. (with Silvio Levy)\xskip The {\tt CWEB} system of structured
 documentation. Stanford Computer Science Report 1336 (Stanford, California,
 October 1990), 200~pp. Also issued as University of Minnesota Supercomputer
 Institute Research Report UMSI 91/56 (Minneapolis, Minnesota, February 1991).
\p *R54. (interview translated into French by Philippe Gabrini) La foi d'un
 scientifique. {\sl La Vie Chr\'etienne}, Journal de l'\'Eglise
 Presbyt\'erienne au Canada (October--November 1991), 11--12.
\p *R55. (with Kenneth A. Ross)\xskip A programming and problem-solving
 seminar. Stanford Computer Science Report 1269 (Stanford, California,
 July 1989), 87~pp.
\p R56. Stable husbands (extended abstract).
 In {\sl 25th International Seminar on the Teaching of Computing Science},
 B. Randell, ed.~(University of Newcastle-Upon-Tyne, 1992), III.1--III.7.  
\p R57. The Stanford Graphbase (extended abstract).
 In {\sl 25th International Seminar on the Teaching of Computing Science},
 B. Randell, ed.~(University of Newcastle-Upon-Tyne, 1992), III.8--III.14.  
\p R58. (interview by Dan Doernberg) Donald E. Knuth: Programming for a
 human being instead of a computer. {\sl New Book Bulletin}, Computer
 Literacy Bookshops (Spring 1994), 4--5; (Summer 1994), 4--5.
 [{\tt http:/\kern-.1em/www.clbooks.com/nbb/knuth.html}]
 Czech translation by Ji\v{r}\'{\i} Zlatu\v{s}ka, {\sl \'UVT MU Zprovodaj\/
 \bf6},\,1 (1995), 1--4; {\bf6},\,2 (1995), 15--20.
\p R59. Digital typography. {\sl Kyoto Prizes 1996\/} (Kyoto: Inamori
 Foundation, 1997),  82--109.
 Reprinted as Chapter~1 of {\sl Digital Typography\/} (see under Books).
\p R60. (conducted by Philip L. Frana) An interview with Donald E. Knuth.
 OH 332 (Minneapolis, Minnesota: Charles Babbage Institute, 8 November 2001),
 27pp.
\finishpage
\end
