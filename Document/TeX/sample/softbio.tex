% this biography requested by Wiley in 1993 for a software encyclopedia

\magnification\magstep1

\font\logo=logo10 % font used for the METAFONT logo

\def\MF{{\logo META}\-{\logo FONT}}
\def\TeX{T\hbox{\hskip-.1667em\lower.424ex\hbox{E}\hskip-.125em X}}
\def\bib{\par\noindent\hangindent 20pt}

\line{\bf KNUTH, DONALD E. (1938--\ \ )\hfill}

\bigskip\noindent
Donald E. Knuth studied mathematics as an undergraduate at Case
Institute of Technology, where he also worked at the Computing Center
writing assemblers and compilers. After graduation in 1960 he began
advanced work at California Institute of Technology, where he received
a~Ph.D. in Mathematics in 1963 and then remained on the mathematics
faculty. Throughout this period he continued to be involved with
software development, serving as consultant to Burroughs Corporation
from 1960--1968, and as editor of Programming Languages for ACM
publications from 1964--1967. He joined Stanford University as
Professor of Computer Science in 1968, and was appointed to Stanford's
first endowed chair in computer science nine years later. In 1993 he
became Professor Emeritus of The Art of Computer Programming.

Knuth began in 1962 to prepare textbooks about programming techniques,
and this work has evolved into a planned seven-volume series entitled
{\sl The Art of Computer Programming}. Volumes 1--3 appeared in 1968,
1969, and 1973, and he is now working full time on the remaining
volumes. He took ten years off from this project to work on digital
typography, developing the \TeX\ system for document preparation and
the \MF\ system for alphabet design. Noteworthy byproducts of those
activities were the {\tt WEB} and {\tt CWEB} languages for structured
documentation, and the accompanying methodology of Literate
Programming.

His research papers have been instrumental in establishing several
subareas of software engineering: LR$(k)$ parsing; attribute grammars;
the Knuth--Bendix algorithm; empirical studies of user programs and
profiles; analysis of algorithms. In general, his works have been
directed towards the search for a proper balance between theory and
practice.

\bigskip
\line{REFERENCES\hfil}

\bib
Graham, Ronald L., Knuth, Donald E., and Patashnik, Oren (1989) {\sl
Concrete Mathematics}. Reading, Mass.: Addison\kern.05em--Wesley.

\medskip
\bib
Knuth, Donald E. (1968) {\sl Fundamental Algorithms}. Reading, Mass.:
Addison\kern.05em--Wesley. 

\medskip
\bib
Knuth, Donald E. (1969) {\sl Seminumerical Algorithms}. Reading, Mass.:
Addison\kern.05em--Wesley. 

\medskip
\bib
Knuth, Donald E. (1973) {\sl Sorting and Searching}. Reading, Mass.:
Addison\kern.05em--Wesley. 

\medskip
\bib
Knuth, Donald E. (1986) {\sl Computers \& Typesetting}, five volumes.
Reading, Mass.: Addison\kern.05em--Wesley. 

\medskip
\bib
Knuth, Donald E. (1992) {\sl Literate Programming}.
Chicago: University of Chicago Press.

\medskip
\bib
Knuth, Donald E. (1993) {\sl The Stanford GraphBase}.
New York: ACM Press.

\medskip
\bib
Knuth, Donald E. (1996) {\sl Selected Papers on Computer Science}.
Cambridge: Cambridge University Press.

\medskip
\bib
Knuth, Donald E. (2000) {\sl Selected Papers on Analysis of Algorithms}.
Chicago: University of Chicago Press.

\medskip
\bib
Knuth, Donald E. (2003) {\sl Selected Papers on Computer Languages}.
Chicago: University of Chicago Press.

\bye



