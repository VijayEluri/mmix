% This test file generates the output shown on the opposite page.
% It's a bit complex because it tries to illustrate lots of stuff.
% TeX ignores commentary (like this) that follows a `%' sign.
% First the standard output style is changed slightly:
\hsize=29pc % The lines in this book are 29 picas wide.
\vsize=42pc % The page body is 42 picas (not counting footlines).
\footline={\tenrm Footline\quad\dotfill\quad Page \folio}
\pageno=1009 % This is the starting page number (don't ask why).
% See Chapter 23 for the way to make other page format changes via
% \hoffset, \voffset, \nopagenumbers, \headline, or \raggedbottom.

\topglue 1in % This makes an inch of blank space (1in=2.54cm).
\centerline{\bf A Bold, Centered Title}
\smallskip % This puts a little extra space after the title line.
\rightline{\it avec un sous-titre \`a la fran\c caise}

% Now we use \beginsection to introduce part 1 of the document.
\beginsection 1. Plain \TeX nology % The next line must be blank!

The first paragraph of a new section is not indented.
\TeX\ recognizes the end of a paragraph when it comes to a blank
line in your manuscript file. % or to a `\par': see below.

Subsequent paragraphs {\it are\/} indented.\footnote*{The amount
of indentation can be changed by changing a parameter called
{\tt\char`\\parindent}. Turn the page for a summary of \TeX's most
important parameters.} (See?) The computer breaks a paragraph's
text into lines in an interesting way---see reference~[1]---and h%
yphenates words automatically when necessary.

\midinsert % This begins inserted material, e.g., a figure.
\narrower\narrower % This brings the margins in (see Chapter 14).
\noindent \llap{``}If there hadn't been room for this material on
the present page, it would have been inserted on the next one.''
\endinsert % This ends the insertion and the effect of \narrower.

\proclaim Theorem T. The typesetting of $math$ is discussed in
Chapters 16--19, and math symbols are summarized in Appendix~F.

\noindent
()()()()()()()()()()()()\break ()()()()()()()()()()()()()\break()()()()()()()()()()()
()()()()()()()()()()()()()()()()()()()()()
()()()\par
()()()\par
()()()()\break ()()()()()()()()()()()()()()
mmmmmmmmm\break mmmmmmmmmmmmmmmm\break mmmmmmmmmmmmmmmm
\break mmmmmmmmmmmmmmmm\break mmmmmmmmmmmmmm
nnnnnnnnnnnnnnnnnnnnnnnnnn\break nnnnnnnnnnnnnnnnnn
\break nnnnnnnnnnnnnnnnnnnnnnnnnnn

\beginsection 2. Bibliography\par % `\par' acts like a blank line.

\frenchspacing % (Chapter 12 recommends this for bibliographies.)
\item{[1]} D.~E. Knuth and M.~F. Plass, ``Breaking paragraphs
into lines,'' {\sl Softw. pract. exp. \bf11} (1981), 1119--1184.

|\+|\<text$_1$>|&|\<text$_2$>|&|$\,\cdots\,$|\cr|


\bye % This is the way the file ends, not with a \bang but a \bye.