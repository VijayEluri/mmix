% this biography requested by IEEE in 1994 for their awards ceremony program

\magnification\magstephalf
\parskip7pt
\parindent0pt
\font\logo=logo10 % font used for the METAFONT logo
\def\MF{{\logo META}\-{\logo FONT}}
\def\TeX{T\hbox{\hskip-.1667em\lower.424ex\hbox{E}\hskip-.125em X}}

Donald E. Knuth was born on January 10, 1938 in Milwaukee, Wisconsin. He
studied mathematics as an undergraduate at Case Institute of Technology, where
he also wrote software at the Computing Center. The Case faculty took
the unprecendented step of awarding him a Master's degree together with the
B.S. he received in 1960. After graduate studies at California Institute of
Technology, he received a~Ph.D. in Mathematics in 1963 and then remained
on the mathematics faculty. Throughout this period he continued to be involved
with software development, serving as consultant to Burroughs Corporation from
1960--1968 and as editor of Programming Languages for ACM publications from
1964--1967.

He joined Stanford University as Professor of Computer Science in 1968, and
was appointed to Stanford's first endowed chair in computer science nine years
later. As a university professor he introduced a variety of new courses into
the curriculum, notably Data Structures and Concrete Mathematics. In 1993 he
became Professor Emeritus of The Art of Computer Programming. He has supervised
the dissertations of 28~students.

Knuth began in 1962 to prepare textbooks about programming techniques, and this
work evolved into a projected seven-volume series entitled {\sl The Art of
Computer Programming}. Volumes 1--3 appeared in 1968, 1969, and 1973, and he is
now working full time on the remaining volumes. Approximately one million
copies have already been printed, including translations into six languages.
He took ten years off from this project to work on digital typography,
developing the \TeX\ system for document preparation and the \MF\ system for
alphabet design. Noteworthy byproducts of those activities were the {\tt WEB}
and {\tt CWEB} languages for structured documentation, and the accompanying
methodology of Literate Programming. \TeX\ is now used to produce most of the
world's scientific literature in physics and mathematics.

His research papers have been instrumental in establishing several subareas of
computer science and software engineering: LR$(k)$ parsing; attribute grammars;
the Knuth--Bendix algorithm for axiomatic reasoning; empirical studies of user
programs and profiles; analysis of algorithms. In general, his works have been
directed towards the search for a proper balance between theory and practice.

Professor Knuth received the ACM Turing Award in 1974 and became a Fellow
of the British Computer Society in 1980, an Honorary
Member of the IEEE in 1982. He is a member of the American Academy of Arts and
Sciences, the National Academy of Sciences, the National Academy of
Engineering, and a foreign associate of l'Acad\'emie des Sciences (Paris),
Det Norske Videnskaps-Akademi (Oslo), the Bayerische Akademie der
Wissenschaften (Munich), the Royal Society (London), and the Russian
Academy of Sciences (Moscow).
He holds 5~patents and has published
approximately 160~papers in addition to his 25~books. He received the Medal of
Science from President Carter in 1979, the American Mathematical Society's
Steele Prize for expository writing in 1986, the New York Academy of Sciences
Award in 1987, the J.~D. Warnier Prize for software methodology in 1989,
the Adelsk\"old Medal from the Swedish Academy of Sciences in 1994, the
Harvey Prize from the Technion in 1995, and the Kyoto Prize for advanced
technology in 1996. He was
a charter recipient of the IEEE Computer Pioneer Award in 1982, after having
received the IEEE Computer Society's W.~Wallace McDowell Award in 1980; he
received the IEEE's John von Neumann Medal in 1995. He
holds honorary doctorates from Oxford University, the University of Paris,
the Royal Institute of Technology in Stockholm, 
the University of St.~Petersburg,
the University of Marne-la-Vall\'ee,
Masaryk University,
St.~Andrews University,
Athens University of Economics and Business,
the University of Macedonia in Thessaloniki,
the University of T\"ubingen,
the University of Oslo,
the University of Antwerp,
the Swiss Federal Institute of Technology in Z\"urich,
the Armenian Academy of Sciences,
the University of Bordeaux,
and more than a dozen colleges and universities in America (including
Brown, Dartmouth, Pennsylvania, and Harvard in the ``Ivy League'').

Professor Knuth lives on the Stanford campus with his wife, Jill. They have two
children, John and Jennifer. Music is his main avocation.

\bye
