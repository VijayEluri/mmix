%It is really hard to edit the unix command in Text
%Unless you have developed a systematic way to handle all the meta-character.
\hsize=29pc
\vsize=42pc
\topglue 0.5in

\centerline{\bf Tutorial---Linux Kernel Source Code Reading}
\smallskip
\centerline{Eddie Wu}
\bigskip

\beginsection 1. Background \par
I try to document the steps when I set up a environment for reading Linux
Kernel Source Code. Hopefully next time when I need to do the same thing 
again I can reuse this documnet to do it effectivly. If it can also help others
when performing the similar task, it will be great.

\beginsection 2. Steps Outline \par
\indent Download Linux source code\par
\indent Install ctags\par
\indent Install ncurses\par
\indent Install cscope\par
\indent Re-install vim\par
\indent Get file list of source code\par
\indent Use ctags\par
\indent Use cscope\par

\beginsection 2. Download Linux source code \par

\beginsection 2.1. Download Linux source code through git \par
\indent
1). Download Git from {\tt http://git-scm.com/\#download}.

2). Upload the installation file to the target directory and unzipped it as 
follows.\par
\indent{\tt tar -zxvf file.name}

3). Executing the following commands.\par
\indent     {\tt ./configure}\par
\indent 	{\tt make}\par
\indent 	{\tt sudo make install}\par
 	
4). create file prepare.txt with the following content.
	echo \$PATH;\par
	export PATH=\$PATH:/home/oracle/exer/git-1.6.3.3/git-1.6.3.3/:\par
	echo \$PATH;\par
 	
5). execute the command to set the path every time.\par
	. ./prepare.txt
	
6). execute command below to get the source code of linux.
	git-clone  $\backslash$\par
	git://git.kernel.org/pub/scm/linux/kernel/git/torvalds/linux-2.6.git  $\backslash$\par
	linux-2.6
\par *Here linux-2.6 is the top folder for Linux source code.

\beginsection 2.2. Download Linux source code through http \par
git is blocked by firewall, instead I download the source code from 
{\tt http://www.kernel.org/}.  click the "F" link, which stands for the 
Full Source Code.

\beginsection 2. install ctags \par 
and add /usr/local/bin into the PATH

\beginsection 3. install ncurses. \par 
I have encountered the following issue when installing the cscope,
The error message says:\par
``line 52 in build.c, can not find the referece to curse.h''\par
Root cause is that ncurses library is not installed in the system.

1). download ncurses-5.7 from \par
http://ftp.gnu.org/pub/gnu/ncurses/  

2). Intall it with the following commands.\par
{\tt ./configure}\par
{\tt make}\par
{\tt sudo make install}\par

\beginsection 4. install cscope. \par 
After intsall ncurses, it is straightforward to install scope.

\beginsection 5. Re-install vim.\par
Check whether cscope feature is enable in vim as follows.\par
{\tt [oracle@ora10gapp1 vim71]$ vim --version | grep cscope}\par
{\tt +cryptv +cscope +cursorshape +dialog\_con +diff +digraphs -dnd -ebcdic}\par

if cscope is not enable in your local vim installation, then you need to
 reinstall vim with the following setting.\par
{\tt ./configure --enable-cscope}


\beginsection 6. Get file list of source code.\par
\beginsection 7. Use ctags in vim.\par
\beginsection 8. Use cscope in vim.\par

7. 4/3/2008 3:33PM
cscope 

cd /home/oracle/exer/linux/linux-2.6.30.1;\par
LNX=`pwd`;\par
echo \$LNX;\par
find  \$LNX                                                      $\backslash$\par
-path "\$LNX/arch/*" ! -path "\$LNX/arch/x86*" -prune -o         $\backslash$\par
-path "\$LNX/firmware*" -prune -o                                $\backslash$\par
-path "\$LNX/Documentation*" -prune -o                           $\backslash$\par
-path "\$LNX/scripts*" -prune -o                                 $\backslash$\par
-path "\$LNX/drivers*" -prune -o                                 $\backslash$\par
-path "\$LNX/samples*" -prune -o                                 $\backslash$\par
-path "\$LNX/sound*" -prune -o                                   $\backslash$\par
-path "\$LNX/net*" -prune -o                                     $\backslash$\par
-name "*.[chxsS]" -print >/home/oracle/exer/linux/cscope.files
\bye
IT succeed. but very quick and there is no output. So I do not realize 
that it actually succeed.

ctags [options] [file(s)]
ctags -L <file>
ctags -L /home/oracle/exer/linux/cscope.files

you need to open vi/vim in the folder where the tags file is located. otherwise
vi can not find the tags file. but you can add the folloing setting in ~/.vimrc
to specify the location of tags file, then you can open vim in any folder and 
are still able to access the tags file.
:set tags=/home/oracle/exer/linux/linux-2.6.30.1/tags

vi 
:tag do\_brk;

:ts can show you all the options. then you can choose from the list of options.

Ctrl-] go to the definition 
Ctrl-t return to last place.


linux-2.6.30.1
change 1:
the asm for different arch are put into corresponding arch, it is no longer
in the inlcude folder.    

So all the includes are relative to too possible path, one is relative to 
./include folder, another is arhc-specific, e.g. ./arch/x86/include/, later 
folder only have a sub-folder called asm.	
    
    
Documentation/kbuild/makefiles.txt explain all the details of the Kbuild 
system.    
    
    
    
    
    

    
 :cscope add /home/oracle/eddie/cscope.out
 
find  : Query cscope.  All cscope query options are available
        except option \#5 ("Change this grep pattern").

    USAGE   :cs find {querytype} {name}

        {querytype} corresponds to the actual cscope line
        interface numbers as well as default nvi commands:

            0 or s: Find this C symbol
            1 or g: Find this definition
            2 or d: Find functions called by this function
            3 or c: Find functions calling this function
            4 or t: Find this text string
            6 or e: Find this egrep pattern
            7 or f: Find this file
            8 or i: Find files \#including this file
need to manually add the following config in ~/.vimrc
:cscope add /home/oracle/eddie/cscope/cscope.out        

example usage in vim:
:cs find c tty\_ldisc\_deref             
:cs find s      tty\_ldisc\_wait
  
  
  
  
  
  7/27/2009 3:44PM





2. Then it is easy to install cscope.

3. next step is to install vim from source code (it is already done)
ora10gapp2> vim \-\-version | grep scope
{\tt +}cryptv +cscope +cursorshape +dialog\_con +diff +digraphs -dnd -ebcdic 

ora10gapp2> ls -l `which vim`
-rwxr-xr-x 1 root root 1704148 2008-02-12 03:00 /usr/bin/vim
ora10gapp2> ls -l `which vi` 
-rwxr-xr-x 1 root root 660564 2008-02-12 03:00 /bin/vi

/home/oracle/exer/linux/     \# the directory with 'cscope.files'
cscope -b -q -k    
add the following in .vimrc
:cscope add /home/oracle/exer/linux/cscope.out 
start vim again.
:cs find c sys\_mmap2

If there are many candidate options and you do not go through them all.
just use "q" to exit from the choices list and input the number you 
choosed.


Surprising enough, no head file ever include the auto.conf.�
  
  
  
  
\bye  

It seems that many functions are defined to be used by drivers.
since I remove the drivers from source code, I can not see how
these functions are used.

Now I know the usage of macro __ASSEMBLY__
it gererate different code for ASM and C code. so that the same
code can match different grammar.
#ifdef __ASSEMBLY__
#define _AC(X,Y)        X
#define _AT(T,X)        X
#else           
#define __AC(X,Y)       (X##Y)
#define _AC(X,Y)        __AC(X,Y)
#define _AT(T,X)        ((T)(X))
#endif
so,
#define __START_KERNEL_map      _AC(0xffffffff80000000, UL)
will get (0xffffffff80000000UL) in C code.
but 0xffffffff80000000 in assembly code.

make_page_present combines many things together. 
TODO: (See how high memory is used and how to kmap it)   

Some time we may encounter some macro. to find the definition for it.
The best way is to use cscope as follows. (e stands for egrep pattern).

:cs find e define SYSCALL_DEFINE1   

for the struct definition, same thind hold.
:cs find e struct inode
























