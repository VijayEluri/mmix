
\ansno1.1:
 A \TeX nician (underpaid); sometimes also called a \TeX acker.

\ansno2.1:
 |Alice said, ``I always use an en-dash instead of a hyphen when|\break
|specifying page numbers like `480--491' in a bibliography.''| \
(The wrong answer to this question ends with |'480-49l' in a bibliography."|)

\ansno2.2:
 You get em-dash and hyphen (----), which looks awful.

\ansno2.3:
 fluffier firefly fisticuffs, flagstaff fireproofing,
chiffchaff and riffraff.

\ansno2.4:
 |``\thinspace`|; and either |`{}``| or |{`}``| or something similar.
Reason: There's usually less space {\sl preceding\/} a single left quote than
there is preceding a double left quote. \ (Left and right are opposites.)

\ansno2.5:
 Eliminating ^|\thinspace| would mean that a user need not learn
the term; but it is not advisable to minimize terminology by ``overloading''
math mode with tricky constructions. For example, a user who wishes to
take advantage of \TeX's ^|\mathsurround| feature would be thwarted by
non-mathematical uses of dollar signs. \ (Incidentally, neither |\thinspace|
nor ^|\,| are built into \TeX; both are defined in terms of more
primitive features, in Appendix~B.)

\ansno3.1:
 |\I|, |\exercise|, and |\\|. (The last of these is of type~2, i.e.,
a control symbol, since the second backslash is not a letter; the first
backslash keeps the second one from starting its own control sequence.)

\ansno3.2:
 |math\'ematique| and |centim\`etre|.^^|\'|^^|\`|

\ansno3.3:
 According to the index, |\|\] is primitive but
|\|\<return> isn't. The command `|\def\^^M{\ }|' in
Appendix~B is what actually defines |\|\<return>, since a
return is representable as |^^M|. Asking \TeX\ to |\show\^^M|
\looseness-1
produces the response `|>| |\^^M=macro:->\|\]|.|'.

\ansno3.4:
 There are 256 of length~2; most of these are undefined when \TeX\
begins. \ (\TeX\ allows any character to be an escape, but it does not
distinguish between control sequences that start with different escape
characters.) \
If we assume that there are 52 letters, there are exactly $52^2$
possible control sequences of length~3 (one for each pair of letters, from
|AA| to |zz|). But Chapter~7 explains how to use ^|\catcode| to change any
character into a ``^{letter}''; therefore it's possible to use any of
$256^2$ potential control sequences of length~3.

\ansno4.1:
 |Ulrich Dieter, {\sl Journal f\"ur die reine und angewandte|\parbreak
        |Mathematik\/ \bf201} (1959), 37--70.|\par\nobreak\smallskip\noindent
It's convenient to use a single group for both |\sl| and |\bf| here. The
`|\/|' is a refinement that you might not understand until you read the
rest of Chapter~4.

\ansno4.2:
 |{\it Explain ... typeset a\/ {\rm roman} word ... sentence.}|
Note the position of the italic correction in this case.

\ansno4.3:
 |\def\ic#1{\setbox0=\hbox{#1\/}\dimen0=\wd0|\parbreak
|\setbox0=\hbox{#1}\advance\dimen0 by -\wd0}|.

\ansno4.4:
 Control word names are made of letters, not digits.

\ansno4.5:
 Say |\def\sl{\it}| at the beginning, and delete other definitions
of\/ |\sl| that might be present in your format file (e.g., there might be
one inside a |\tenpoint| macro).

\ansno4.6:
 |\font\squinttenrm=cmr10 at 5pt|\parbreak
        |\font\squinttenrm=cmr10 scaled 500|

\ansno5.1:
 |{shelf}ful| or |shelf{}ful|, etc.; or even |shelf\/ful|, which
yields a shelf\/ful instead of a shelf{\kern0pt}ful.
In fact, the latter idea---to
insert an ^{italic correction}---is preferable because \TeX\ will ^^|\/|
reinsert the ff ligature by itself after ^{hyphenating} |shelf{}ful|. \
(Appendix~H points out that ligatures are put into a hyphenated word that
contains no ``^{explicit kerns},'' and an italic correction is an
explicit kern.) \ But the italic correction may be too much (especially in an
italic font); |shelf{|^|\kern||0pt}ful| is often best.

\ansno5.2:
 `\]|{|\]|}|\]' or `\]|{}|\]|{}|\]', etc. Plain \TeX\ also has a
^|\space| macro, so you can type |\space\space\space|.  \ (These aren't
strictly equivalent to `|\|\]|\|\]|\|\]', since they adjust the spaces by
the current ``^{space factor},'' as explained later.)

\ansno5.3:
 In the first case, you get the same result as if the innermost
braces had not appeared at all, because you haven't used the grouping to
change fonts or to control spacing or anything. \TeX\ doesn't mind if you
want to waste your time making groups for no particular reason.
But in the second case, the necessary braces were forgotten. You get the
letter `S' centered on a line by itself, followed by a paragraph that
begins with `o should this.' on the next line.

\ansno5.4:
 You get the same result as if another pair of braces were present
around `|\it centered|', except that the period is typeset from the
italic font. \ (Both periods look about the same.) \ The |\it| font
will not remain in force after the |\centerline|, but this is
something of a coincidence: \TeX\ uses the braces to determine what
text is to be centered, but then it removes the braces. The
|\centerline| operation, as defined in Appendix~B\null, puts the
resulting braceless text inside {\sl another\/} group; and that's why
|\it| disappears after |\centerline|. \ (If you don't understand this,
just don't risk leaving out braces in tricky situations, and you'll be OK.)

\ansno5.5:
 |\def\ital#1{{\it#1\/}}|. \ Pro:~Users might find this easier to
learn, because it works more like |\centerline| and they don't have to
remember to make the italic correction. \ Con:~To avoid the italic correction
just before a {\it comma} or {\it period}, users should probably be taught
another control sequence; for example, with
\begintt
\def\nocorr{\kern0pt }
\endtt
a user could type `|\ital{comma} or \ital{period\nocorr},|'. The alternative
of putting a period or comma in italics, to avoid the italic correction,
doesn't look as good. A long sequence of italics would be inefficient for
\TeX, since the entire text for the argument to |\ital| must be read into
memory only to be scanned again.

\ansno5.6:
 |{1 {2 3 4 5} 4 6} 4|.

\ansno5.7:
 |\def\beginthe#1{\begingroup\def\blockname{#1}}|\parbreak
|\def\endthe#1{\def\test{#1}%|\parbreak
|  \ifx\test\blockname\endgroup|\parbreak
|  \else\errmessage{You should have said|\parbreak
|    \string\endthe{\blockname}}\fi}|

\ansno6.1:
 Laziness and/or obstinacy.

\ansno6.2:
 There's an unwanted space after `called---', because (as the book
says) \TeX\ treats the end of a line as if it were a blank space. That
blank space is usually what you want, except when a line ends with a
hyphen or a dash; so you should {\sc WATCH OUT} for lines that end with
hyphens or dashes.

\ansno6.3:
 It represents the heavy bar that shows up in
your output. \ (This bar wouldn't be present if\/ ^|\overfullrule| had been
set to |0pt|, nor is it present in an underfull box.)

\ansno6.4:
 This is the ^|\parfillskip| space that ends the paragraph.
In plain \TeX\ the parfillskip is zero when the last line of the paragraph
is full; hence no space actually appears before the rule in the output
of Experiment~3. But all hskips show up as spaces in an overfull box
message, even if they're zero.

\ansno6.5:
 Run \TeX\ with \hbox{|\hsize=1.5in|} \hbox{|\tolerance=10000|}
\hbox{|\raggedright|} \hbox{|\hbadness=-1|} and then |\input story|. \TeX\ will
report the badness of all lines (except the final lines of paragraphs, where
fill glue makes the badness zero).

\ansno6.6:
 |\def\extraspace{\nobreak \hskip 0pt plus .15em\relax}|\parbreak
|\def\dash{\unskip\extraspace---\extraspace}|\par\nobreak\smallskip\noindent
(If you try this with the story at 2-inch and 1.5-inch sizes, you will
notice a substantial improvement. The |\unskip| allows people to leave a
space before typing |\dash|.  \TeX\ will try to hyphenate before |\dash|,
but not before `|---|'; cf.\ Appendix~H\null. The ^|\relax| at the end of
|\extraspace| is a precaution in case the next word is `|minus|'.)

\ansno6.7:
 \TeX\ would have deleted five tokens: |1|, |i|, |n|, \],
|\centerline|.  (The space was at the end of line~2, the |\centerline| at the
beginning of line~3.)

\ansno6.8:
 A control sequence like |\centerline| might well define a control
sequence like |\ERROR| before telling \TeX\ to look at |#1|. Therefore
\TeX\ doesn't interpret control sequences when it scans an argument.

\ansno7.1:
 Three forbidden characters were used. One should type
\begintt
Procter \& Gamble's ... \$2, a 10\% gain.
\endtt
(Also the facts are wrong.)

\ansno7.2:
 Reverse slashes (backslashes) are fairly uncommon in formulas or
text, and |\\| is very easy to type; it was therefore felt best not to
reserve |\\| for such limited use. Typists can define |\\| to be whatever
they want (including |\backslash|).

\ansno7.3:
 1, 2, 3, 4, 6, 7, 8, 10, 11, 12, 13. ^{Active characters} (type 13)
are somewhat special; they behave like control sequences in most cases
(e.g., when you say `^|\let||\x=~|' or `^|\ifx||\x~|'), but they behave like
character tokens when they appear in the token list of\/ ^|\uppercase|
or ^|\lowercase|, and when unexpanded after ^|\if| or ^|\ifcat|.

\ansno7.4:
 It ends with either |>| or |}| or any character of category 2;
then the effects of all |\catcode| definitions within the group are wiped
out, except those that were ^|\global|. \TeX\ doesn't have any built-in
knowledge about how to pair up particular kinds of grouping characters.
New category codes take effect as soon as a |\catcode| assignment has been
digested. For example,
\begintt
{\catcode`\>=2 >
\endtt
is a complete group. But without the space after `|2|' it would not be
complete, since \TeX\ would have read the~`|>|' and converted it to a
token before knowing what category code was being specified; \TeX\ always
reads the token following a constant before evaluating that ^{constant}.

\ansno7.5:
 If you type `|\message{\string~}|' and `|\message{\string\~}|', \TeX\
responds with `|~|' and `|\~|', respectively. ^^|\message|
To get |\|$_{12}$ from |\string| you therefore need to make backslash an
active character. One way to do this is
\begintt
{\catcode`/=0 \catcode`\\=13 /message{/string\}}
\endtt
(The ``^{null control sequence}'' that you get when there are no
tokens between |\csname| and |\endcsname| is not a solution to this exercise,
because |\string| converts it to `|\csname\endcsname|'. There is, however,
another solution: If \TeX's ^|\escapechar| parameter---which will be
explained in one of the next dangerous bends---is negative or greater
than~255, then `|\string\\|' works.)

\ansno7.6:
 |\|$_{12}$ |a|$_{12}$ |\|$_{12}$ \]$_{10}$ |b|$_{12}$.

\ansno7.7:
 |\def\ifundefined#1{\expandafter\ifx\csname#1\endcsname\relax}|%
\hfil\break Note that a control sequence like this must be used with care;
it cannot be included in ^{conditional} text, because the |\ifx| will not
be seen when |\ifundefined| isn't expanded.

\ansno7.8:
 First |\uppercase| produces `|A\lowercase{BC}|'; then you get `|Abc|'.

\ansno7.9:
 `\thinspace|\copyright\ \uppercase\expandafter{\romannumeral\year}|%
\thinspace'. \ (This is admittedly tricky; the `^|\expandafter|' expands
the token after the `|{|', not the token after the group.)

\ansno7.10:
 (We assume that parameter |#2| is not simply an active character,
and that ^|\escapechar| is between 0 and~255.)
\begintt
\def\gobble#1{} % remove one token
\def\appendroman#1#2#3{\expandafter\def\expandafter#1\expandafter
  {\csname\expandafter\gobble\string#2\romannumeral#3\endcsname}}
\endtt

\ansno8.1:
 The |%| would be treated as a comment character, because its
category code is~14; thus, no |%| token or |}| token would get through
to the gullet of \TeX\ where numbers are treated. When a character is
of category 0, 5, 9, 14, or~15, the extra |\| must be used; and the
|\| doesn't hurt, so you can always use it to be safe.

\ansno8.2:
 (a)~Both characters terminate the current line; but a character of
category~5 might be converted into \]$_{10}$ or a \cstok{par} token, while
a character of category~14 never produces a token.  (b)~They produce
character tokens stamped with different category numbers.  For example,
|$|$_3$ is not the same token as |$|$_4$, so \TeX's digestive processes
will treat them differently.  (c)~Same as~(b), plus the fact that control
sequence names treat letters differently.  (d)~No. (e)~Yes; characters of
category~10 are ignored at the beginning of every line, since every line
starts in state~$N$. (f)~No.

\ansno8.3:
 \TeX\ had just read the control sequence |\vship|, so it
was in state~$S$, and it was just ready to read the space before `|1in|'.
Afterwards it ignored that space, since it was in state~$S$; but if
you had typed |I\obeyspaces| in response to that error message,
you would have seen the space. Incidentally, when \TeX\ prints
the ^{context of an error message}, the bottom pair of lines comes from
a text file, but the other pairs of lines are portions of token lists
that \TeX\ is reading (unless they begin with `|<*>|', when they
represent text inserted during ^{error recovery}).

\ansno8.4:
 |$|$_{3}$ |x|$_{11}$ |^|$_7$ |2|$_{12}$ |$|$_{3}$ |~|$_{13}$ \]$_{10}$
\cstok{TeX} |b|$_{11}$ |v|$_{11}$ \]$_{10}$. The final space comes from the
\<return> placed at the end of the line. Code |^^6| yields |v| only
when not followed by |0|--|9| or |a|--|f|.
The initial space is ignored, because state~$N$
governs the beginning of the line.

\ansno8.5:
 |H|$_{11}$ |i|$_{11}$ |!|$_{12}$ \]$_{10}$ \cstok{par}
\cstok{par}. The `\]' comes from the \<return> at the
end of the first line; the second and third lines each contribute
a \cstok{par}.

\ansno8.6:
 The two |^^B|'s are not recognized as consecutive superscript
characters, since the first |^^B| is converted to code~2 which doesn't
equal the following character |^|. Hence
the result is seven tokens: |^^B|$_7$ |^^B|$_7$
|M|$_{11}$ \cstok{\^{}\^{}B} \]$_{10}$ |^^M|$_{11}$ \cstok{M\^{}\^{}M}.
The last of these is a control word whose name has two letters.
The \<space> after |\M| is deleted before \TeX\ inserts the \<return> token.

\ansno8.7:
 Both alternatives work fine in text; in particular, they combine
as in |\lq\lq| to form ligatures. But the definition in Appendix~B works
also in connection with constants; e.g., |\char\lq\%| and
|\char\rq140| are valid. \ (Incidentally, the construction |\let\lq=`|
would not work with constants, since the quotes in a ^\<number> must
come from character tokens of category~12; after |\let\lq=`| the control
sequence token |\lq| will not expand into a character token, nor {\sl is\/}
it a character token!) ^^|\let| ^^{implicit character}

\ansno9.1:
 |na\"\i ve| or |na{\"\i}ve| or |na\"{\i}ve|.

\ansno9.2:
 Belov\`ed prot\'eg\'e; r\^ole co\"ordinator; souffl\'es, cr\^epes,
p\^at\'es, etc.

\ansno9.3:
 |\AE sop's \OE uvres en fran\c cais|.

\ansno9.4:
 |{\sl Commentarii Academi\ae\ scientiarum imperialis|\hfil\break
|petropolitan\ae\/} became {\sl Akademi\t\i a Nauk SSSR, Doklady}.|

\ansno9.5:
 |Ernesto Ces\`aro,
P\'al Erd\H os,
\O ystein Ore,
Stanis\l aw \'Swier%|\break|czkowski,
Serge\u\i\ \t Iur'ev,
Mu\d hammad ibn M\^us\^a al-Khw\^arizm\^\i.|

\ansno9.6:
 The proper umlaut is |\H|, which isn't available in |\tt|, so
it's necessary to borrow the accent from another font. For example,
\hbox{|{\tt P\'al Erd{\bf\H{\tt o}}s}|} uses a bold accent, which
is suitably dark.

\ansno9.7:
 |{\it Europe on {\sl\$}15.00 a day\/}|

\ansno9.8:
 The extra braces keep font changes local. An argument makes the
use of\/ |\'| more consistent with the use of other accents like |\d|, which
are manufactured from other characters without using the |\accent|
primitive.

\ansno10.1:
 Exactly $7227\pt$.

\ansno10.2:
 $\rm-.013837\,in$, $\rm0.\,mm$, $\rm+42.1\,dd$, $\rm3\,in$,
$\rm29\,pc$, $\rm123456789\,sp$.
\ (The lines of text in this manual are 29~picas wide.)

\ansno10.3:
 The first is not allowed, since octal notation cannot be used with
a decimal point. The second is, however, legal, since a \<number> can be
hexadecimal according to the rule mentioned in Chapter~8; it means
$\rm12\,cc$, which is $\rm144\,dd\approx154.08124\,pt$. The third is also
accepted, since a \<digit string> can be empty; it is a complicated
way to say $\rm0\,sp$.

\ansno10.4:
 {\obeylines|\def\tick#1{\vrule height 0pt depth #1pt}|
|\def\\{\hbox to 1cm{\hfil\tick4\hfil\tick8}}|
|\vbox{\hrule\hbox{\tick8\\\\\\\\\\\\\\\\\\\\}}|
\noindent(You might also try putting ticks at every millimeter, in order %
to see how good your system is; %
some output devices can't handle 101~rules all at once.)}

\ansno10.5:
 For example, say `|\magnification=\magstep1 \input story \end|'
to get magnification 1200; |\magstep2| and |\magstep3| are 1440 and 1728.
Three separate runs are needed, since there can be at most one
magnification per job. The output may look funny if the fonts don't exist
at the stated magnifications.

\ansno10.6:
 Magnification is by a factor of 1.2. Since font |\first| is |cmr10|
at $12\pt$, it will be |cmr10| at $14.4\pt$ after magnification;
font |\second| will be |cmr10| at $12\pt$. \ (\TeX\ changes
`|12truept|' into `|10pt|', and the final output magnifies it back to
$12\pt$.)

\ansno11.1:
 This |E| is inside a box that's inside a box.

\ansno11.2:
 The idea is to construct a box and to look inside. For example,
\begintt
\setbox0=\hbox{\sl g\/} \showbox0
\endtt
reveals that |\/| is implemented by placing a kern after the character.
Further experiment shows that this kern is inserted even when the italic
correction is zero.

\ansno11.3:
 The height, depth, and width of the enclosing box should be just large
enough to enclose all of the contents, so the result is:
\begintt
\hbox(8.98608+0.0)x24.44484
.\tenrm T
.\kern 1.66702
.\hbox(6.83331+0.0)x6.80557, shifted -2.15277
..\tenrm E
.\kern 1.25
.\tenrm X
\endtt
(You probably predicted a height of |8.9861|; \TeX's internal calculations are
in |sp|, not |pt|/100000, so the rounding in the fifth decimal place is not
readily predictable.)

\ansno11.4:
 No applications of such symmetrical boxes to English-language
printing were apparent; it seemed pointless to carry extra generality
as useless baggage that would rarely if ever be used, merely for the sake of
symmetry. In other words, the author wore a computer science cap instead
of a mathematician's mantle on the day that \TeX's boxes were born.
Time will tell whether or not this was a fundamental error!

\ansno11.5:
 The following solution is based on a general |\makeblankbox|
macro that prints the edges of a box using rules of given thickness
outside and inside that box; the box dimensions are those of\/ |\box0|.\par
|\def\dolist{\afterassignment\dodolist\let\next= }|\parbreak
|\def\dodolist{\ifx\next\endlist \let\next\relax|\parbreak
|  \else \\\let\next\dolist \fi|\parbreak
|  \next}|\par
|\def\endlist{\endlist}|\par
|\def\hidehrule#1#2{\kern-#1%|\parbreak
|  \hrule height#1 depth#2 \kern-#2 }|\par
|\def\hidevrule#1#2{\kern-#1{\dimen0=#1|\parbreak
|    \advance\dimen0 by#2\vrule width\dimen0}\kern-#2 }|\par
|\def\makeblankbox#1#2{\hbox{\lower\dp0\vbox{\hidehrule{#1}{#2}%|\parbreak
|    \kern-#1 % overlap the rules at the corners|\parbreak
|    \hbox to \wd0{\hidevrule{#1}{#2}%|\parbreak
|      \raise\ht0\vbox to #1{}% set the vrule height|\parbreak
|      \lower\dp0\vtop to #1{}% set the vrule depth|\parbreak
|      \hfil\hidevrule{#2}{#1}}%|\parbreak
|    \kern-#1\hidehrule{#2}{#1}}}}|\par
|\def\maketypebox{\makeblankbox{0pt}{1pt}}|\par
|\def\makelightbox{\makeblankbox{.2pt}{.2pt}}|\par
|\def\\{\if\space\next\ % assume that \next is unexpandable|\parbreak
| \else \setbox0=\hbox{\next}\maketypebox\fi}|\par
|\def\demobox#1{\setbox0=\hbox{\dolist#1\endlist}%|\parbreak
|  \leavevmode\copy0\kern-\wd0\makelightbox}|\par

\ansno11.6:
 |\def\frac#1/#2{\leavevmode\kern.1em|\parbreak
|\raise.5ex\hbox{\the\scriptfont0 #1}\kern-.1em|\parbreak
|/\kern-.15em\lower.25ex\hbox{\the\scriptfont0 #2}}|

\ansno12.1:
 $9+16$ units, $9+32$ units, $12+0$ units. \ (But \TeX\ would
consider so much stretching to be ``infinitely bad.'')

\ansno12.2:
 `What happens now?' is placed in a line of width |\hsize|, with
twice as much space at the left as at the right; `and now?' is put flush right
on the following line.

\ansno12.3:
 The first two give an ``overfull box'' if the argument doesn't fit
on a line; the third allows the argument to stick out into the margins
instead. \ (Plain \TeX's ^|\centerline| is |\centerlinec|; the stickout effect
shows up in the narrow-column experiment of Chapter~6.) \ If the argument
contains no infinite glue, |\centerlinea| and |\centerlineb| produce the same
effect; but |\centerlineb| will center an argument that contains `fil' glue.

\ansno12.4:
 |Mr.~\& Mrs.~User
were married by Rev.~Drofnats, who preached on
Matt.~19\thinspace:\thinspace3--9.| \ (Such thin spaces are traditional
for ^{Biblical references} to chapter and verse, but you weren't really
expected to know that. Plain \TeX\ defines ^|\thinspace| to be a kern,
not glue; hence no break between lines will occur at a thinspace.)

\ansno12.5:
 |Donald~E.\ Knuth, ``Mathematical typography,'' {\sl Bull.\
Amer.\ Math.\ Soc.\ \bf1} (1979), 337--372.| \ (But the `|\|' after `|E.|'
isn't necessary, because of a rule you will learn if you venture
around the next dangerous bend.)

\ansno12.6:
 There are several ways; perhaps the easiest are to type
`|\hbox{NASA}.|'\ or `|NASA\null.|' \ (The ^|\null| macro is an abbreviation
for `|\hbox{}|'.)

\ansno12.7:
 1000, except: 999 after |O|, |B|, |S|, |D|, and |J|; 1250 after the
comma; 3000 after the exclamation point, the right-quote marks, and the
periods. If a period had come just after the |B| (i.e., if the text had
said `|B. Sally|'), the space factor after that period would have
been~1000, not~3000.

\ansno12.8:
 |\box3| is $2\pt$ high, $4\pt$ deep, $3\pt$ wide.
Starting at the reference point of\/ |\box3|, go right $.75\pt$ and down
$3\pt$ to reach the reference point of\/ |\box1|; or go right $1\pt$
to reach the reference point of\/ |\box2|.

\ansno12.9:
 The stretch and shrink components of\/ |\baselineskip| and
|\lineskip| should be equal, and the |\lineskiplimit| should
equal the normal |\lineskip| spacing, to guarantee continuity.

\ansno12.10:
 Yes it did, but only because none of his boxes had a negative
height or depth. He would have been safer if he had set
|\baselineskip=-1000pt|, |\lineskip=0pt|, and
|\lineskiplimit=16383pt|. \ (Plain \TeX's ^|\offinterlineskip| macro does this.)

\ansno12.11:
 The interline glue will be zero, and the natural height is
$1+1-3+2=1\pt$ (because the depth of\/ |\box2| isn't included in the natural
height); so the glue will ultimately become |\vskip-1pt| when it's set.
Thus, |\box3| is $3\pt$ high, $2\pt$ deep, $4\pt$ wide. Its reference
point coincides with that of\/ |\box2|; to get to the reference point
of\/ |\box1| you go up $2\pt$ and right $3\pt$.

\ansno12.12:
 The interline glue will be $6\pt$ minus $3\,{\rm fil}$; the final
depth will be zero, since |\box2| is followed by glue; the natural
height is $12\pt$; and the shrinkability is $5\,{\rm fil}$. So |\box4|
will be $4\pt$ high, $0\pt$ deep, $1\pt$ wide, and it will contain
five items: |\vskip|\penalty0\hbox{|-1.6pt|}, |\box1|, |\vskip1.2pt|,
|\moveleft4pt\box2|, |\vskip-1.6pt|. Starting at the reference point of
|\box4|, you get to the reference point of\/ |\box1| by going up $4.6\pt$,
or to the reference point of\/ |\box2| by going up $.4\pt$ and left $4\pt$.
\ (For example, you go up $4\pt$ to get to the upper left corner of
|\box4|; then down $-1.6\pt$, i.e., up $1.6\pt$, to get to the upper left
corner of\/ |\box1|; then down $1\pt$ to reach its reference point.  This
problem is clearly academic, since it's rather ridiculous to include
infinite shrinkability in the baselineskip.)

\ansno12.13:
 Now |\box4| will be $4\pt$ high, $-4\pt$ deep, $1\pt$ wide, and it
will contain |\vskip|\penalty0\hbox{|-2.4pt|}, |\box1|, |\vskip-1.2pt|,
|\moveleft4pt\box2|, |\vskip-2.4pt|. From the baseline of\/ |\box4|, go up
exactly $5.4\pt$ to reach the baseline of\/ |\box1|, or exactly $3.6\pt$
to reach the baseline of\/ |\box2|.

\ansno12.14:
 |\vbox to| $x$|{}| produces height $x$;
|\vtop to| $x$|{}| produces depth $x$; the other dimensions are zero.
\ (This holds even when $x$ is negative.)

\ansno12.15:
 There are several possibilities:
\begintt
\def\nullbox#1#2#3{\vbox to#1{\vss\hrule height-#2depth#2width#3}}
\endtt
works because the rule will be of zero thickness. Less tricky is
\begintt
\def\nullbox#1#2#3{\vbox to#1{\vss\vtop to#2{\vss\hbox to#3{}}}}
\endtt
Both of these are valid with negative height and/or depth, but they do
not produce negative width. If the width might be negative, but not the
height or depth, you can use, e.g.,
|\def\nullbox#1#2#3{\hbox to#3{\hss\raise#1\null\lower#2\null}}|.
It's impossible for |\hbox| to construct a box
whose height or depth is negative; it's impossible for |\vbox| or
|\vtop| to construct a box whose width is negative.\par
However, there's actually a trivial solution to the general problem, based on
features that will be discussed later:
\begintt
\def\nullbox#1#2#3{\setbox0=\null
  \ht0=#1 \dp0=#2 \wd0=#3 \box0 }
\endtt

\ansno12.16:
 |\def\llap#1{\hbox to 0pt{\hss#1}}|

\ansno12.17:
 You get `A' at the extreme left and `puzzle.\null' at the extreme right,
because the space between words has the only stretchability that is finite;
the infinite stretchability cancels out. \ (In this case, \TeX's rule
about ^{infinite glue} differs from what you would get in the limit if the
value of $1\,{\rm fil}$ were finite but getting larger and larger.
The true limiting behavior would stretch the text `A~puzzle.\null' in the
same way, but it would also move that text infinitely far away beyond the right
edge of the page.)

\ansno13.1:
 Simply saying |\hbox{...}| won't work, since that box will just
continue the previous vertical list without switching modes. You need
to start the paragraph explicitly, and the straightforward way to
do that is to say |\indent\hbox{...}|.
But suppose you want to define
a macro that expands to an hbox, where this macro is to be used in the
midst of a paragraph as well as at the beginning; then you don't want
to force users to type |\indent| before calling your macro at the
beginning of a paragraph, nor do you want to say |\indent| in the
macro itself (since that might insert unwanted indentations).  One
solution to this more general problem is to say
`|\|\]^|\unskip||\hbox{...}|', since |\|\] makes the mode
horizontal while |\unskip| removes the unwanted space. Plain \TeX\
provides a ^|\leavevmode| macro, which solves this problem in what is
probably the most efficient way: |\leavevmode| is an abbreviation for
`|\unhbox\voidbox|', where |\voidbox| is a permanently empty box register.

\ansno13.2:
 The output of\/ |\tracingcommands| shows that four blank space tokens
were digested; these originated at the ends of lines 2,~3, 4, and~5. Only
the first had any effect, since blank spaces are ignored in math formulas
and in vertical modes.

\ansno13.3:
 The |end-group character| finishes the paragraph and the |\vbox|,
and |\bye| stands for `|\par\vfill...|', so the next three commands are
\begintt
{math mode: math shift character $}
{restricted horizontal mode: end-group character }}
{vertical mode: \par}
\endtt

\ansno13.4:
 It contains only mixtures of vertical glue and horizontal rules
whose reference points appear at the left of the page; there's no text.

\ansno13.5:
 Vertical mode can occur only as the outermost mode; horizontal
mode and display math mode can occur only when immediately enclosed by
vertical or internal vertical mode; ordinary math mode cannot be
immediately enclosed by vertical or internal vertical mode; all other
cases are possible.

\ansno14.1:
|(cf.~Chapter~12).|\parbreak
|Chapters 12 and~21.|\parbreak
|line~16 of Chapter~6's {\tt story}|\parbreak
|lines 7 to~11|\parbreak
|lines 2,~3, 4, and~5.|\parbreak
|(2)~a big black bar|\parbreak
|All 256~characters are initially of category~12,|\parbreak
|letter~{\tt x} in family~1.|\parbreak
|the factor~$f$, where $n$~is 1000~times~$f$.|

\ansno14.2:
 `|for all $n$~greater than~$n_0$|' avoids distracting breaks.

\ansno14.3:
 `|exercise \hbox{4.3.2--15}|' guarantees that there is no break
after the ^{en-dash}. But this precaution is rarely necessary, so
`|exercise 4.3.2--15|' is an acceptable answer. No |~| is needed;
`\hbox{4.3.2--15}' is so long that it causes no offense
at the beginning of a line.

\ansno14.4:
 The space you get from |~| will stretch or shrink with the
other spaces in the same line, but the space inside an hbox has
a fixed width since that glue has already been set once and for all.
Furthermore the first alternative permits the word Chap-\break
ter to be ^{hyphenate}d.

\ansno14.5:
 `|\hbox{$x=0$}|' is unbreakable, and we will see later that
`|${x=0}$|' cannot be broken. Both of these solutions set the glue
surrounding the equals sign to some fixed value, but such glue normally
wants to stretch; furthermore, the |\hbox| solution might include undesirable
blank space at the beginning or end of a line, if\/ ^|\mathsurround| is
nonzero. A third solution `|$x=\nobreak0$|' avoids both defects.

\ansno14.6:
 |\exhyphenpenalty=10000| prohibits all such breaks, according
to the rules found later in this chapter. Similarly, |\hyphenpenalty=10000|
prevents breaks after implicit (discretionary) hyphens.

\ansno14.7:
 The second and fourth lines are indented by an additional ``quad''
of space, i.e., by one extra em in the current type style.
\ (The control sequence |\quad| does an ^|\hskip|; when \TeX\ is in
vertical mode, |\hskip| begins a new paragraph and puts glue after the
indentation.) \ If\/ |\indent| had been used instead, those lines wouldn't
have been indented any more than the first and third, because |\indent| is
implicit at the beginning of every paragraph.  Double indentation on the
second and fourth lines could have been achieved by `|\indent\indent|'.

\ansno14.8:
 |ba\ck/en| and |Be\ttt/uch|, where the macros |\ck/| and |\ttt/|
are defined by
\begintt
\def\ck/{\discretionary{k-}{k}{ck}}
\def\ttt/{tt\discretionary{-}{t}{}}
\endtt
The English word `eighteen' might deserve similar treatment.
\TeX's hyphenation algorithm will not make such spelling changes automatically.

\ansno14.9:
 |\def\break{\penalty-10000 }|

\ansno14.10:
 You get a forced break as if\/ |\nobreak| were not present, because
|\break| cannot be cancelled by another penalty. In general if you
have two penalties in a row, their combined effect is the same as a single
penalty whose value is the minimum of the two original values, unless
both of those values force breaks. \ (You get two breaks from
|\break\break|; the second one creates an empty line.)

\ansno14.11:
 Breaks are forced when $p\le-10000$, so there's no point in
subtracting a large constant whose effect on the total demerits is
known {\sl a priori}, especially when that might cause arithmetic overflow.

\ansno14.12:
 $(10+131)^2+0^2+10000=29881$ and $(10+1)^2+50^2+10000=12621$.
In both cases the ^|\adjdemerits| were added because the lines were
visually incompatible (decent, then very loose, then decent); plain
\TeX's values for ^|\linepenalty| and |\adjdemerits| were used.

\ansno14.13:
 Because \TeX\ discards a glue item that occurs just before
|\par|. Ben should have said, e.g., `|\hfilneg\ \par|'.

\ansno14.14:
 Just say |\parfillskip|\stretch|=|\stretch|\parindent|. Of course,
\TeX\ will not be able to find appropriate line breaks unless each
paragraph is sufficiently long or sufficiently lucky; but with an
appropriate text, your output will be immaculately
symmetrical.{\parfillskip=\parindent\par}

\ansno14.15:
 Assuming that the author is deceased and/or set in his or her
ways, the remedy is to insert `|{\parfillskip=0pt\par\parskip=0pt\noindent}|'
in random places, after each 50 lines or so of text. \ (Every space
between words is usually a feasible breakpoint, when you get sufficiently
far from the beginning of a paragraph.)

\ansno14.16:
 |{\leftskip=-1pt \rightskip=1pt| \<text> |\par}|\par
\nobreak\medskip\noindent
(This applies to a full paragraph; if you want to correct only
isolated lines, you have to do it by hand.)

\ansno14.17:
 `|\def\line#1{\hbox to\hsize{\hskip\leftskip#1\hskip\rightskip}}|'
is the only change needed. \ (Incidentally,
^{displayed equations} don't take account of\/ |\leftskip| and |\rightskip|
either; it's more difficult to change that, because so many variations
are possible.)

\ansno14.18:
 The author's best solution is based on a variable |\dimen|
register |\x|:
\begintt
\setbox1=\hbox{I}
\setbox0=\vbox{\parshape=11 -0\x0\x -1\x2\x -2\x4\x -3\x6\x
   -4\x8\x -5\x10\x -6\x12\x -7\x14\x -8\x16\x -9\x18\x -10\x20\x
  \ifdim \x>2em \rightskip=-\wd1
  \else \frenchspacing \rightskip=-\wd1 plus1pt minus1pt
   \leftskip=0pt plus 1pt minus1pt \fi
  \parfillskip=0pt \tolerance=1000 \noindent I turn, ... hand.}
\centerline{\hbox to \wd1{\box0\hss}}
\endtt
Satisfactory results are obtained with font |cmr10| when |\x| is set to
$8.9\pt$, $13.4\pt$, $18.1\pt$, $22.6\pt$, $32.6\pt$, and $47.2\pt$,
yielding triangles that are respectively 11,~9, 8, 7, 6, and~5 lines tall.

\ansno14.19:
 |\item{}| at the beginning of each paragraph that wants hanging
indentation.

\ansno14.20:
 |\item{$\bullet$}|

\ansno14.21:
 Either change |\hsize| or |\rightskip|. The trick is to change it back
again at the end of a paragraph. Here's one way, without grouping:
\begintt
\let\endgraf=\par \edef\restorehsize{\hsize=\the\hsize}
\def\par{\endgraf \restorehsize \let\par=\endgraf}
\advance\hsize by-\parindent
\endtt

\ansno14.22:
 |\dimen0=\hsize \advance\dimen0 by 2em|\parbreak
|\parshape=3 0pt\hsize 0pt\hsize -2em\dimen0|

\ansno14.23:
 The three paragraphs can be combined into a single paragraph, if
you use `|\hfil\vadjust{\vskip\parskip}\break\indent|' instead of
`|\par|' after the first two.  Then of course you say, e.g.,
|\hangindent=-50pt \hangafter=-15|. \ (The same idea can be applied in
connection with |\looseness|, if you want \TeX\ to make one of three
paragraphs looser but if you don't want to choose which one it will be.
However, long paragraphs fill \TeX's memory; please use restraint.) \
See also the next exercise.

\ansno14.24:
 Use |\hangcarryover| between paragraphs, defined as follows:
\begintt
\def\hangcarryover{\edef\next{\hangafter=\the\hangafter
    \hangindent=\the\hangindent}
  \par\next
  \edef\next{\prevgraf=\the\prevgraf}
  \indent\next}
\endtt

\ansno14.25:
 It will set the current paragraph in the minimum number of lines
that can be achieved without violating the tolerance; and, given that
number of lines, it will break them optimally. \ (However, nonzero
looseness makes \TeX\ work harder, so this is not recommended if you
don't want to pay for the extra computation. You can achieve almost the
same result much more efficiently by setting ^|\linepenalty||=100|, say.)

\ansno14.26:
 150, 100, 0, 250. \ (When the total penalty is zero, as between lines
3 and~4 in this case, no penalty is actually inserted.)

\ansno14.27:
 |\interlinepenalty| plus |\clubpenalty| plus |\widowpenalty| (and
also plus |\brokenpenalty|, if the first line ends with a discretionary break).

\ansno14.28:
 The tricky part is to avoid ``opening up'' the paragraph by
adding anything to its height; yet this star is to be contributed after
a line having an unknown depth, because the depth of the line depends
on details of line breaking that aren't known until afterwards.
The following solution uses ^|\strut|, and assumes that the line containing
the marginal star does not have depth exceeding ^|\dp||\strutbox|, the
depth of a ^|\strut|.
\begintt
\def\strutdepth{\dp\strutbox}
\def\marginalstar{\strut\vadjust{\kern-\strutdepth\specialstar}}
\endtt
Here |\specialstar| is a box of height zero and depth |\strutdepth|,
and it puts an asterisk in the left margin:
\begintt
\def\specialstar{\vtop to \strutdepth{
    \baselineskip\strutdepth
    \vss\llap{* }\null}}
\endtt

\ansno14.29:
 |\def\insertbullets{\everypar={\llap{$\bullet$\enspace}}}|\par
\nobreak\smallskip\noindent
(A similar device can be used to insert hanging indentation,
and/or to number the paragraphs automatically.)

\ansno14.30:
 First comes |\parskip| glue (but you might not see it on the current
page if you say |\showlists|, since glue disappears at the top of each
page). Then comes the result of\/ |\everypar|, but let's assume that
|\everypar| doesn't add anything to the horizontal list, so that
you get an empty horizontal list; then there's no partial paragraph
before the display. The displayed equation follows the normal rules
(it occupies lines 1--3 of the paragraph, and uses the indentation and
length of line~2, if there's a nonstandard shape).  Nothing follows the
display, since a blank space is ignored after a closing `|$$|'.\par
Incidentally, the behavior is different if you start a paragraph with
`|$$|' instead of with |\noindent$$|, ^^{display at beginning of paragraph}
since \TeX\ inserts a paragraph indentation that will appear on a line by
itself (with |\leftskip| and |\parfillskip| and |\rightskip| glue).

\ansno14.31:
 A break at |\penalty50| would cancel |\hskip2em\nobreak\hfil|,
so the next line would be forced to start with the reviewer's name flush left.
\ (But ^|\vadjust||{}| would actually be better than |\hbox{}|; it
uses \TeX\ more efficiently.)

\ansno14.32:
 Otherwise the line-breaking algorithm might prefer two final lines to
one final line, simply in order to move a hyphen from the second-last line up
to the third-last line where it doesn't cause demerits. This in fact caused
some surprises when the |\signed| macro was being tested; |\tracingparagraphs=1|
was used to diagnose the problem.

\ansno14.33:
 Distributing the extra space evenly would lead to three lines of
the maximum badness (10000). It's better to have just one bad line
instead of three, since \TeX\ doesn't distinguish degrees of badness when
lines are really awful. In this particular case the ^|\tolerance| was 200,
so \TeX\ didn't try any line breaks that would stretch the first two lines;
but even if the tolerance had been raised to 10000, the optimum setting would
have had only one underfull line. If you really want to spread the
space evenly you can do so by using ^|\spaceskip| to increase the
amount of stretchability between words.

\ansno14.34:
 |\def\raggedcenter{\leftskip=0pt plus4em \rightskip=\leftskip|%
\parbreak|\parfillskip=0pt \spaceskip=.3333em \xspaceskip=.5em|\parbreak
        |\pretolerance=9999 \tolerance=9999 \parindent=0pt|\parbreak
        |\hyphenpenalty=9999 \exhyphenpenalty=9999 }|

\ansno15.1:
 The last three page-break calculations would have been
\begintt
% t=503.0 plus 8.0 minus 4.0 g=528.0 b=3049 p=150 c=3199#
% t=514.0 plus 8.0 minus 4.0 g=528.0 b=533 p=-100 c=433#
% t=542.0 plus 11.0 minus 6.0 g=528.0 b=* p=0 c=*
\endtt
so the break would have occurred at the same place. The badness would have
been~533, but the page would still have looked tolerable. \ (On the other
hand if that paragraph had been two lines shorter instead of one,
the first two lines of the next ``dangerous bend'' paragraph
would have appeared on that page; the natural height $t=531\pt$ would have
been able to shrink to $g=528\pt$ because the three ``medskips'' on
the page would have had a total shrinkability of $6\pt$. This would certainly
have been preferable to a stretched-out page whose badness was~3049; but the
author might have seen it and written another sentence or two, so that
the paragraph would not have been broken up. After all, this manual is supposed
to be an example of good practice.)

\ansno15.2:
 The next legal break after the beginning of a dangerous bend
paragraph occurs $28\pt$ later, because there is $6\pt$ additional space for
a |\medskip|, followed by two lines of $11\pt$ each. \TeX\ does not
allow breaking between those two lines; the ^|\clubpenalty| is set briefly
to 10000 in Appendix~E\null, since the dangerous bend symbol is two lines tall.

\ansno15.3:
 A page always contains at least one box, if there are no
insertions, since the legal breakpoints are discarded otherwise.
Statement~(a) fails if the height of the topmost box exceeds $10\pt$.
Statement~(b) fails if the depth of the bottommost box exceeds $2.2\pt$, or
if some glue or kern comes between the bottommost box and the page break
(unless that glue or kern exactly cancels the depth of the box).

\ansno15.4:
 |\topinsert\vskip2in\rightline{\vbox{\hsize|\stretch|...|\stretch
|artwork.}}\endinsert|
does the job. But it's slightly more efficient to avoid ^|\rightline| by
changing ^|\leftskip| as follows:
`|\leftskip=\hsize \advance\leftskip by-3in|'.
Then \TeX\ doesn't have to read the text of the caption twice.

\ansno15.5:
 It would appear on page~25, since it does fit there. A |\midinsert|
will jump ahead of other insertions only if it is not carried over to another
page; for example, if the second 3-inch insertion were a |\midinsert|, it would
not appear on page~26, because it is converted to a |\topinsert| as soon as the
|\midinsert| macro notices that the insertion is too big for page~25.

\ansno15.6:
 Set |\count1| to 50,
then |\dimen2| to~$50\pt$,
then |\count1| to~6,
then |\skip2| to~$-10\pt$ plus~$6\,{\rm fil}$ minus~$50\pt$,
then |\skip2| to~$60\pt$ plus~$-36\,{\rm fil}$ minus~$-300\pt$,
then |\skip2| to~$1\,{\rm sp}$ minus~$-6\,{\rm sp}$,
then |\count6| to~1,
then |\skip1| to~$25\pt$ plus~$1\,{\rm sp}$ minus~$1\,{\rm fill}$,
then |\skip2| to~$25\pt$ minus~$-150\pt$,
then |\skip1| to~$0\pt$ plus~$1\,{\rm sp}$ minus~$1\,{\rm fill}$.

\ansno15.7:
 If\/ |\skip4| has infinite stretchability, |\skip5| will be zero;
otherwise it will be $0\pt$ plus~$1\pt$.

\ansno15.8:
 |\advance\dimen2 by\ifnum\dimen2<0 -\fi.5\dimen3|\parbreak
|\divide\dimen2 by\dimen3 \multiply\dimen2 by\dimen3|

\ansno15.9:
 |\count1| takes the values 5, then~2 (the old 5 is saved),
then~4 (which is made global), then~8 (and 4~is saved); finally the value~4 is
restored, and that is the answer. \ (For further remarks, see the discussion
of\/ |\tracingrestores| in Chapter~27.)

\ansno15.10:
 |\hbox{\hbox{A}A}|. After `|\unhbox5|', |\box5| is void; |\unhcopy5|
yields nothing.

\ansno15.11:
 |\hbox{A}|. But after `|{\global\setbox3=\hbox{A}\setbox3=\box3}|',
|\box3| will be void.

\ansno15.12:
 |\newcount\notenumber|\parbreak
|\def\clearnotenumber{\notenumber=0\relax}|\parbreak
|\def\note{\advance\notenumber by 1|\parbreak
|  \footnote{$^{\the\notenumber}$}}|

\ansno15.13:
 Yes, in severe circumstances.  (1)~Previous footnotes might
 have left no room for any more footnotes on the page.
(2)~If |\vadjust{\eject}| occurs on the same line as a footnote, before that
footnote, the reference will be forcibly detached. (3)~Other |\vadjust|
commands on that line could also interpose breakpoints before the insertion.

\ansno16.1:
 |$\gamma+\nu\in\Gamma$|.

\ansno16.2:
 ^|\le|, ^|\ge|, and ^|\ne|. \ (These are short for ``less-or-equal,''
``greater-or-equal,'' and ``not-equal.'') \ You can also use the names
^|\leq|, ^|\geq|, and ^|\neq|. \ (The fourth most common symbol is, perhaps,
`$\infty$', which stands for ``^{infinity}'' and is called `^|\infty|'.)

\ansno16.3:
 In the former, the `|_2|' applies to the plus sign ($x + _2F_3$);
but in the latter, it applies to an empty subformula ($x + {}_2F_3$).

\ansno16.4:
 The results are `${x^y}^z$' and `$x^{y^z}$'; the $z$ in the first
alternative is the same size as the $y$, but in the second it is smaller.
Furthermore, the $y$ and $z$ in the first case aren't quite at the same height.
\ (Good typists never even think of the first construction, because
mathematicians never want it.)

\ansno16.5:
 The second alternative doesn't work properly when there's a
subscript at the same time as a prime. Furthermore, some mathematicians
use |\prime| also in the subscript position; they write, for example,
$F'(w,z)=\partial F(w,z)/\partial z$ and $F_\prime(w,z)=\partial F(w,z)/
\partial w$.

\ansno16.6:
 |$R_i{}^{jk}{}_l$|.

\ansno16.7:
 |10^{10}|; \stretch|2^{n+1}|; \stretch|(n+1)^2|; \stretch|\sqrt{1-x^2}|;
\stretch|\overline{w+\overline z}|; \stretch|p_1^{e_1}|; \stretch
|a_{b_{c_{d_e}}}|; \stretch|\root3\of{h''_n(\alpha x)}|.
\ (Of course, you should enclose these formulas in dollar signs so that
\TeX\ will process them in math mode.  Superscripts and subscripts can be
given in either order; for example, |h''_n| and |h_n''| both work the
same.  You should not leave out any of the braces shown here; for example,
`|$10^10$|' would yield `$10^10$'. But it doesn't hurt to insert
additional braces around letters or numbers, as in `|({n}+{1})^{2}|'. The
indicated blank spaces are necessary unless you use extra braces;
otherwise \TeX\ will complain about undefined control sequences
|\overlinez| and |\alphax|.)

\ansno16.8:
 He got `If$ x = y\ldots$' because he forgot to leave a space
after `|If|'; ^{spaces} disappear between dollar signs. He should
also have ended the sentence with `|$y$.|'; punctuation that belongs
to a sentence should not be included in a formula, as we will see
in Chapter~18. \ (But you aren't expected to know that yet.)

\ansno16.9:
 |Deleting an element from an $n$-tuple leaves an $(n-1)$-tuple.|

\ansno16.10:
 $Q,f,g,j,p,q,y$. \ (The analogous ^{Greek} letters are
^^{italic letters with descenders} ^^{descenders}
$\beta,\gamma,\zeta,\eta,\mu,\xi,\rho,\phi,\varphi,\chi,\psi$.)

\ansno16.11:
 |$z^{*2}$| and |$h_*'(z)$|.

\ansno16.12:
 |$3{\cdot}1416$|. \ (One of the earlier examples in this
chapter showed that ^|\cdot| is a binary operation; putting it in braces
makes it act like an ordinary symbol.)\par
If you have lots of constants like this, for example in a table, there's a way
to make ordinary periods act like |\cdot| symbols: Just define
^|\mathcode||`.| to be |"0201|, assuming that the fonts of plain \TeX\ are
being used.  However, this could be dangerous, since ordinary
periods are used frequently in displayed equations; the |\mathcode| change
should be confined to places where every period is to be a |\cdot|.

\ansno16.13:
 |$e^{-x^2}$|, |$D\sim p^\alpha M+l$|, and |$\ghat\in(H^{\pi_1^{-1}})'$|.
\ (If you are reading the dangerous bend sections, you know that the
recommended way to define |\ghat| is `|\def\ghat{{\hat g}}|'.)

\ansno17.1:
 $x+y^{2/(k+1)}$\quad(|$x+y^{2/(k+1)}$|).

\ansno17.2:
 $((a+1)/(b+1))x$\quad(|$((a+1)/(b+1))x$|).

\ansno17.3:
 He got the displayed formula$$x=(y^2\over k+1)$$ because he forgot
that an unconfined |\over| applies to everything.  \ (He should probably
have typed `|$$x=\left(y^2\over k+1\right)$$|', using ideas that will be
presented later in this chapter; this not only makes the parentheses
larger, it keeps the `$x=$' out of the fraction, because |\left| and
|\right| introduce subformulas.)

\ansno17.4:
 `|$7{1\over2}\cents$|' or `|7$1\over2$\cents|'. \ (Incidentally,
the definition used here was |\def\cents{\hbox{\rm\rlap/c}}|.)
^^|\rlap|^^|\cents|

\ansno17.5:
 Style $D'$ is used for the subformula $p_2^{e'}$, hence style~$S'$
is used for the superscript~$e'$ and the subscript~2, and style~$\SS'$
is used for the supersuperscript prime. The square root sign and the $p$
appear in text size; the 2 and the~$e$ appear in script size; and the
$\prime$ is in scriptscript size.

\ansno17.6:
 |$${1\over2}{n\choose k}$$|;
|$$\displaystyle{n\choose k}\over2$$|.
All of these braces are necessary.

\ansno17.7:
 |$${p \choose 2} x^2 y^{p-2} - {1 \over 1-x}{1 \over 1-x^2}.$$|

\ansno17.8:
 |$$\sum_{i=1}^p\sum_{j=1}^q\sum_{k=1}^ra_{ij}b_{jk}c_{ki}$$|.

\ansno17.9:
 |$$\sum_{{\scriptstyle 1\le i\le p \atop \scriptstyle 1\le j\le q}
    \atop \scriptstyle 1\le k\le r} a_{ij} b_{jk} c_{ki}$$|.

\ansno17.10:
 |$\displaystyle\biggl({\partial^2\over\partial x^2}+|\hfil\break
|{\partial^2\over\partial y^2}\biggr)\bigl|\||\varphi(x+iy)\bigr|\||^2=0$|.

\ansno17.11:
 Formulas that are more than one line tall are usually two lines tall,
not 1$1\over2$ or 2$1\over2$ lines tall.

\ansno17.12:
 |$\bigl(x+f(x)\bigr) \big/ \bigl(x-f(x)\bigr)$|. \ Notice especially
the `|\big/|'; an ordinary ^{slash} would look too small between the
|\big| parentheses.

\ansno17.13:
 |$$\pi(n)=\sum_{k=2}^n\left\lfloor\phi(k)\over k-1\right\rfloor.$$|

\ansno17.14:
 |$$\pi(n)=\sum_{m=2}^n\left\lfloor\biggl(\sum_{k=1}^{m-1}\bigl|
\hfil\break
|\lfloor(m/k)\big/\lceil m/k\rceil\bigr\rfloor\biggr)^{-1}\right\rfloor.$$|

\ansno17.15:
 A displayed formula equivalent to |$${D}{{T}\over{T}^{{S}^{SS}}}$$|.

\ansno17.16:
 |\def\sqr#1#2{{\vcenter{\vbox{\hrule height.#2pt|\parbreak
        |        \hbox{\vrule width.#2pt height#1pt \kern#1pt|\parbreak
        |           \vrule width.#2pt}|\parbreak
        |        \hrule height.#2pt}}}}|\parbreak
        |\def\square{\mathchoice\sqr34\sqr34\sqr{2.1}3\sqr{1.5}3}|

\ansno17.17:
|\def\euler{\atopwithdelims<>}|.

\ansno17.18:
 The |\textfont0| that was current at the beginning of the formula
will be used, because this redefinition is local to the braces. \
(It would be a different story if `^|\global||\textfont|' had appeared instead;
that would have changed the meaning of\/ |\textfont0| at all levels.)

\ansno17.19:
 \hex{2208} and \hex{220F}.

\ansno17.20:
 |\mathchardef\alpha="710B|. Incidentally, |{\rm\alpha}| will
then give a spurious result, because character position \hex{0B} of
roman fonts does not contain an alpha; you should warn
your users about what characters they are allowed to type under the
influence of special conventions like ^|\rm|.

\ansno17.21:
 If\/ |\delcode`{| were set to some nonnegative delimiter code, you
would get no error message when you wrote something like `|\left{|'.
This would be bad because strange effects would happen when certain
subformulas were given as arguments to macros, or when they appeared
in alignments. But it has an even worse defect, because a user who
gets away with `|\left{|' is likely to try also `|\bigl{|', which
fails miserably.

\ansno17.22:
 Since |\bigl| is defined as a macro with one parameter,
it gets just `|\delimiter|' as the argument. You have to write
`|\bigl{\delimiter"426830A}|' to make this work. On the other hand,
|\left| will balk if the following character is a left brace. Therefore
it's best to have control sequence names for all delimiters.

\ansno18.1:
 |$R(n,t)=O(t^{n/2})$, as $t\to0^+$.| \
(N.B.: `|O(|', not `|0('|.)

\ansno18.2:
 |$$p_1(n)=\lim_{m\to\infty}\sum_{\nu=0}^\infty|\parbreak
|  \bigl(1-\cos^{2m}(\nu!^n\pi/n)\bigr).$$|\par
\smallskip\noindent $\bigl[$Mathematicians may enjoy
interpreting this formula; cf.~G.~H. ^{Hardy},
{\sl Messenger of Mathematics\/ \bf35} (1906), 145--146.$\bigr]$

\ansno18.3:
 |\def\limsup{\mathop{\overline{\rm lim}}}|\parbreak
|\def\liminf{\mathop{\underline{\rm lim}}}|\par
\smallskip\noindent
[Notice that the limits `$n\to\infty$' appear at different levels, in both
of the displays, because `sup' and the underbar descend below the baseline.
It is possible to unify the limit positions by using ^{phantoms}, as explained
later in this chapter. For example,
\begintt
\def\limsup{\mathop{\vphantom{\underline{}}\overline{\rm lim}}}
\endtt
would give lower limits in the same position as |\liminf|.]

\ansno18.4:
 $x\equiv0(\pmod y^n)$. He should have typed
`|$x\equiv0\pmod{y^n}$|'.

\ansno18.5:
 |$${n\choose k}\equiv{\lfloor n/p\rfloor\choose|\parbreak
|  \lfloor k/p\rfloor}{n\bmod p\choose k\bmod p}\pmod p.$$|

\ansno18.6:
 |$\bf\bar x^{\rm T}Mx={\rm0}\iff x=0$|.  \ (If you typed a space between
|\rm| and~|0|, you wasted a keystroke; but don't feel guilty about it.)

\ansno18.7:
 |$S\subseteq{\mit\Sigma}\iff S\in{\cal S}$|. In this case the
braces are redundant and could be eliminated; but you shouldn't try to do
{\sl everything\/} with fewest keystrokes, or you'll outsmart yourself
some day.

\ansno18.8:
 |$${\it available}+\sum_{i=1}^n\max\bigl({\it full}(i),|\parbreak
        |{\it reserved}(i)\bigr)={\it capacity}.$$|
\smallskip\noindent [If\/ |\it| had been used throughout
the formula, the subscript~$i$ and superscript~$n$ would have caused error
messages saying `^|\scriptfont| |4| |is| |undefined|',
since plain \TeX\ makes |\it| available only in text size.]

\ansno18.9:
 |{\obeylines \sfcode`;=3000|^^|\sfcode|\parbreak
|{\bf for $j:=2$ step $1$ until $n$ do}|\parbreak
|\quad {\bf begin} ${\it accum}:=A[j]$; $k:=j-1$; $A[0]:=\it accum$;|\parbreak
|\quad {\bf while $A[k]>\it accum$ do}|\parbreak
|\qquad {\bf begin} $A[k+1]:=A[k]$; $k:=k-1$;|\parbreak
|\qquad {\bf end};|\parbreak
|\quad $A[k+1]:=\it accum$;|\parbreak
|\quad {\bf end}.\par}|\par
\smallskip\noindent
[This is something like the ``poetry'' example in Chapter~14, but much
more difficult. Some manuals of style say that ^{punctuation} should inherit
the font of the preceding character, so that three kinds of semicolons
should be typeset; e.g., these experts recommend `$k:=j-1$; \
$A[0]:={}${\it accum;} \ {\bf end;}'. The author heartily disagrees.]

\ansno18.10:
 |Let $H$~be a Hilbert space, \
$C$~a closed bounded convex subset of~$H$, \
$T$~a nonexpansive self map of~$C$.
Suppose that as $n\to\infty$, \ $a_{n,k}\to0$ for each~$k$,
and $\gamma_n=\sum_{k=0}^\infty(a_{n,k+1}-|\allowbreak|a_{n,k})^+\to0$.
 Then for each $x$~in~$C$,  \
$A_nx=\sum_{k=0}^\infty a_{n,k}T^kx$ converges weakly
to a fixed point of~$T$.|\par
[If any mathematicians are reading this, they might either appreciate
or resent the following attempt to edit the given paragraph
into a more acceptable style: ``%
Let $C$~be a closed, bounded, convex subset of a Hilbert space~$H$,
and let $T$~be a nonexpansive self map of~$C$.
Suppose that as $n\to\infty$, we have $a_{n,k}\to0$ for each~$k$,
and $\gamma_n=\sum_{k=0}^\infty(a_{n,k+1}-a_{n,k})^+\to0$.
Then for each $x$~in~$C$, the infinite sum
$A_nx=\sum_{k=0}^\infty a_{n,k}T^kx$ converges weakly
to a fixed point of~$T$.'']

\ansno18.11:
 |$$\int_0^\infty{t-ib\over t^2+b^2}e^{iat}\,dt=|\parbreak
        |    e^{ab}E_1(ab),\qquad a,b>0.$$|

\ansno18.12:
 |$$\hbar=1.0545\times10^{-27}\rm\,erg\,sec.$$|

\ansno18.13:
 There are ten atoms (the first is $f$ and last is $y^2$); their types,
and the interatomic spacing, are respectively
\begindisplay \def\0{\thinspace}%
  \def\1{\thinspace{\tt\bslash,}\thinspace}%
  \def\2{\thinspace{\tt\bslash>}\thinspace}%
  \def\3{\thinspace{\tt\bslash;}\thinspace}
Ord\0Open\0Ord\0Punct\1Ord\0Close\3Rel\3Ord\2Bin\2Ord.
\enddisplay

\ansno18.14:
 |$\left]-\infty,T\right[\times\left]-\infty,T\right[$|. \ (Or one could
say ^|\mathopen| and ^|\mathclose| instead of\/ |\left| and |\right|;
then \TeX\ would not choose the size of the delimiters, nor would it consider
the subformulas to be of type Inner.) \
% that formula was quoted from MR review of paper by Mario Marino
% in Ricerche Mat. 24 (1975), no.~1, 152--171
Open intervals are more clearly expressed in print
by using parentheses instead of reversed brackets; for example,
compare `$(-\infty,T)\times(-\infty,T)$' to the given formula.

\ansno18.15:
 The first |+| will become a Bin atom, the second an Ord; hence
the result is $x$, medium space, $+$, medium space, $+$, no space, 1.

\ansno18.16:
 |$x_1+x_1x_2+\cdots+x_1x_2\ldots x_n$| \ and\hfil\break
|$(x_1,\ldots,x_n)\cdot(y_1,\ldots,y_n)=x_1y_1+\cdots+x_ny_n$|.

\ansno18.17:
 The commas belong to the sentence, not to the formula; his
decision to put them into math mode meant that \TeX\ didn't put large
enough spaces after them. Also, his formula `$i=1, 2, \ldots, n$' allows
no breaks between lines, except after the $=$, so he's risking
overfull box problems. But suppose the sentence had been more terse:
\begindisplay
Clearly $a_i<b_i$ \ ($i=1,2,\ldots,n$).
\enddisplay
Then his idea would be basically correct:
\begintt
Clearly $a_i<b_i$ \ ($i=1,2,\ldots,n$).
\endtt

\ansno18.18:
 $\ldots$ |never\footnote*{Well \dots, hardly ever.} have| $\ldots$

\ansno18.19:
 Neither formula will be broken between lines, but the thick spaces
in the second formula will be set to their natural width while the thick
spaces in the first formula will retain their stretchability.

\ansno18.20:
 Set ^|\relpenalty||=10000| and ^|\binoppenalty||=10000|.
And you also need to change the definitions of\/ ^|\bmod| and ^|\pmod|,
which insert their own penalties.

\ansno18.21:
 |$\bigl\{\,x^3\bigm|\||h(x)\in\{-1,0,+1\}\,\bigr\}$|.

\ansno18.22:
 |$\{\,p\mid p$~and $p+2$ are prime$\,\}$|, assuming that
^|\mathsurround| is zero. The more difficult alternative
`|$\{\,p\mid p\ {\rm and}\ p+2\rm\ are\ prime\,\}$|' is not a solution,
because line breaks do not occur at |\|\] ^^|\space|
(or at glue of any kind) within math formulas. Of course it may be best to
display a formula like this, instead of breaking it between lines.

\ansno18.23:
 |$$f(x)=\cases{1/3&if $0\le x\le1$;\cr 2/3&if $3\le x\le4$;\cr|\hfil
\break|0&elsewhere.\cr}$$|

\ansno18.24:
 |$$\left\lgroup\matrix{a&b&c\cr d&e&f\cr}\right\rgroup|\hfil\break
|\left\lgroup\matrix{u&x\cr v&y\cr w&z\cr}\right\rgroup$$|.

\ansno18.25:
 |\pmatrix{y_1\cr \vdots\cr y_k\cr}|.

\ansno18.26:
 |\def|\stretch|\undertext|\stretch|#1{$\underline|\stretch
|{\smash|\stretch|{\hbox|\stretch|{#1}}}$}| will underline the
\def\undertext#1{$\underline{\smash{\hbox{#1}}}$}%
words and cross \undertext{through} the descenders; or you could insert
|\vphantom{y}| before the |\hbox|, thereby lowering all of the underlines
to a position below all descenders. Neither of these gives exactly what is
wanted. \ (See also ^|\underbar| in Appendix~B\null.) \ Underlining is actually
not very common in fine typography, since font changes usually work just
as well or better, when you want to emphasize something. If you really want
underlined text, it's best to have a special font in which all
the letters are underlined.

\ansno18.27:
 |$n^{\rm th}$ root|. \ (Incidentally, it is also acceptable
to type `|$n$th|', getting `$n$th', in such situations; the fact that
the $n$ is in italics distinguishes it from the suffix. Typed manuscripts
generally render this with a hyphen, but `$n$-th' is frowned on nowadays
when an italic~$n$ is available.) ^^{nth}

\ansno18.28:
 |${\bf S^{\rm-1}TS=dg}(\omega_1,|\stretch|\ldots,|\stretch|\omega_n)
=\bf\Lambda$|.
\ $\bigl($Did you notice the difference between ^|\omega| ($\omega$)
and~|w| ($w$)?$\bigr)$

\ansno18.29:
 |$\Pr(\,m=n\mid m+n=3\,)$|. \ (Analogous to a set.) ^^|\Pr|

\ansno18.30:
 |$\sin18^\circ={1\over4}(\sqrt5-1)$|. ^^|\circ|

\ansno18.31:
 |$k=1.38065\times10^{-16}\rm\,erg\,K^{-1}$|.

\ansno18.32:
 |$\bar\Phi\subset NL_1^*/N=\bar L_1^*|\parbreak
        |  \subseteq\cdots\subseteq NL_n^*/N=\bar L_n^*$|.

\ansno18.33:
 |$I(\lambda)=\int\!\!\int_Dg(x,y)e^{i\lambda h(x,y)}\,dx\,dy$|.\hfil
\break
(Although three |\!|'s work out best between consecutive integral signs in
displays, the text style seems to want only two.) ^^{double integral}
^^{integral, multiple}

\ansno18.34:
 |$\int_0^1\!\cdots\int_0^1f(x_1,\ldots,x_n)\,dx_1\ldots\,dx_n$|.

\ansno18.35:
 |$$x_{2m}\equiv\cases{Q(X_m^2-P_2W_m^2)-2S^2&($m$ odd)\cr|\parbreak
        |       \noalign{\vskip2pt} % spread the lines apart a little|\parbreak
        |       P_2^2(X_m^2-P_2W_m^2)-2S^2&($m$ even)\cr}\pmod N.$$|

\ansno18.36:
 |$$(1+x_1z+x_1^2z^2+\cdots\,)\ldots(1+x_nz+x_n^2z^2+\cdots\,)|\parbreak
        |  ={1\over(1-x_1z)\ldots(1-x_nz)}.$$| \ (Notice the uses of\/ |\,|.)

\ansno18.37:
 |$$\prod_{j\ge0}\biggl(\sum_{k\ge0}a_{jk}z^k\biggr)|\parbreak
        |  =\sum_{n\ge0}z^n\,\Biggl(\sum_|\parbreak
        |     {\scriptstyle k_0,k_1,\ldots\ge0\atop|\parbreak
        |      \scriptstyle k_0+k_1+\cdots=n}|\parbreak
        |   a_{0k_0}a_{1k_1}\ldots\,\Biggr).$$|\par
\nobreak\smallskip\noindent Some people would prefer to have the latter
parentheses larger; but |\left| and |\right| come out a bit too large in this
case. It's not difficult to define ^|\bigggl| and ^|\bigggr| macros, analogous
to the definitions of\/ |\biggl| and |\biggr| in Appendix~B.

\ansno18.38:
 |$${(n_1+n_2+\cdots+n_m)!\over n_1!\,n_2!\ldots n_m!}|\parbreak
        |  ={n_1+n_2\choose n_2}{n_1+n_2+n_3\choose n_3}|\parbreak
        |    \ldots{n_1+n_2+\cdots+n_m\choose n_m}.$$|

\ansno18.39:
 |$$\def\\#1#2{(1-q^{#1_#2+n})} % to save typing|\parbreak
        |\Pi_R{a_1,a_2,\ldots,a_M\atopwithdelims[]b_1,b_2,\ldots,b_N}|\parbreak
        |  =\prod_{n=0}^R{\\a1\\a2\ldots\\aM\over\\b1\\b2\ldots\\bN}.$$|
^^|\atopwithdelims|

\ansno18.40:
 |$$\sum_{p\rm\;prime}f(p)=\int_{t>1}f(t)\,d\pi(t).$$|

\ansno18.41:
 |$$\{\underbrace{\overbrace{\mathstrut a,\ldots,a}|\parbreak
        |      ^{k\;a\mathchar`'\rm s},|\parbreak
        |    \overbrace{\mathstrut b,\ldots,b}|\parbreak
        |      ^{l\;b\mathchar`'\rm s}}_{k+l\rm\;elements}\}.$$|\par
\smallskip\noindent Notice how ^{apostrophes} (instead of primes) were obtained.

\ansno18.42:
 |$$\pmatrix{\pmatrix{a&b\cr c&d\cr}&|\parbreak
        |             \pmatrix{e&f\cr g&h\cr}\cr|\parbreak
        |           \noalign{\smallskip}|\parbreak
        |           0&\pmatrix{i&j\cr k&l\cr}\cr}.$$|

\ansno18.43:
 |$$\det\left|\||\,\matrix{|\parbreak
        |  c_0&c_1\hfill&c_2\hfill&\ldots&c_n\hfill\cr|\parbreak
        |  c_1&c_2\hfill&c_3\hfill&\ldots&c_{n+1}\hfill\cr|\parbreak
        |  c_2&c_3\hfill&c_4\hfill&\ldots&c_{n+2}\hfill\cr|\parbreak
        |  \,\vdots\hfill&\,\vdots\hfill&|\parbreak
        |       \,\vdots\hfill&&\,\vdots\hfill\cr|\parbreak
        |  c_n&c_{n+1}\hfill&c_{n+2}\hfill&\ldots&c_{2n}\hfill\cr|\parbreak
        |  }\right|\||>0.$$|

\ansno18.44:
 |$$\mathop{{\sum}'}_{x\in A}f(x)\mathrel{\mathop=^{\rm def}}|\parbreak
        |  \sum_{\scriptstyle x\in A\atop\scriptstyle x\ne0}f(x).$$|\par
\smallskip\noindent
This works because |{\sum}| is type Ord (so its superscript is not set
above), but ^|\mathop||{{\sum}'}| is type Op (so its subscript is set below).
The limits are centered on $\sum'$, however, not on $\sum$. If you don't
like that, the remedy is more difficult; one solution is to use
|\sumprime_{x\in A}| where ^|\sumprime| is defined as follows:
\par\nobreak\medskip
|\def\sumprime_#1{\setbox0=\hbox{$\scriptstyle{#1}$}|\parbreak
|  \setbox2=\hbox{$\displaystyle{\sum}$}|\parbreak
|  \setbox4=\hbox{${}'\mathsurround=0pt$}|\parbreak
|  \dimen0=.5\wd0 \advance\dimen0 by-.5\wd2|\parbreak
|  \ifdim\dimen0>0pt|\parbreak
|    \ifdim\dimen0>\wd4 \kern\wd4 \else\kern\dimen0\fi\fi|\parbreak
|  \mathop{{\sum}'}_{\kern-\wd4 #1}}|

\ansno18.45:
 |$$2\uparrow\uparrow k\mathrel{\mathop=^{\rm def}}|\parbreak
        |  2^{2^{2^{\cdot^{\cdot^{\cdot^2}}}}}|\parbreak
        |    \vbox{\hbox{$\Big\}\scriptstyle k$}\kern0pt}.$$|\par

\ansno18.46:
 If you have to do a lot of commutative diagrams, you will want to
define some macros like those in the first few lines of this solution.
The ^|\matrix| macro resets the baselines to ^|\normalbaselines|, because
other commands like |\openup| might have changed them, so
we redefine |\normalbaselines| in this solution. Some of the things
shown here haven't been explained yet, but Chapter~22 will reveal all.
\smallskip
|$$\def\normalbaselines{\baselineskip20pt|\parbreak
|  \lineskip3pt \lineskiplimit3pt }|\parbreak
|\def\mapright#1{\smash{|\parbreak
|    \mathop{\longrightarrow}\limits^{#1}}}|\parbreak
|\def\mapdown#1{\Big\downarrow|\parbreak
|  \rlap{$\vcenter{\hbox{$\scriptstyle#1$}}$}}|\parbreak
|\matrix{&&&&&&0\cr|\parbreak
|  &&&&&&\mapdown{}\cr|\parbreak
|  0&\mapright{}&{\cal O}_C&\mapright\iota&|\parbreak
|    \cal E&\mapright\rho&\cal L&\mapright{}&0\cr|\parbreak
|  &&\Big\Vert&&\mapdown\phi&&\mapdown\psi\cr|\parbreak
|  0&\mapright{}&{\cal O}_C&\mapright\pi&|\parbreak
|    \pi_*{\cal O}_D&\mapright\delta&|\parbreak
|    R^1f_*{\cal O}_V(-D)&\mapright{}&0\cr|\parbreak
|  &&&&&&\mapdown{\theta_i\otimes\gamma^{-1}}\cr|\parbreak
|  &&&&&&\hidewidth R^1f_*\bigl({\cal O}|\parbreak
|    _V(-iM)\bigr)\otimes\gamma^{-1}|^|\hidewidth||\cr|\parbreak
|  &&&&&&\mapdown{}\cr|\parbreak
|  &&&&&&0\cr}$$|

\ansno19.1:
 |$$\sum_{n=0}^\infty a_nz^n\qquad\hbox{converges if}\qquad|\parbreak
|  |\||z|\||<\Bigl(\limsup_{n\to\infty}\root n\!\of{|\||a_n|\|^^|\root|
  |}\,\Bigr)^{-1}.$$|\kern-.33pt\par
\smallskip
|$${f(x+\Delta x)-f(x)\over\Delta x}\to f'(x)|\parbreak
|    \qquad\hbox{as $\Delta x\to0$.}$$|\par
\smallskip
|$$\|\||u_i\|\||=1,\qquad u_i\cdot u_j=0\quad\hbox{if $i\ne j$.}$$|\par
\smallskip
|$$\it\hbox{The confluent image of}\quad\left\{|\parbreak
|    \matrix{\hbox{an arc}\hfill\cr\hbox{a circle}\hfill\cr|\parbreak
|      \hbox{a fan}\hfill\cr}|\parbreak
|    \right\}\quad\hbox{is}\quad\left\{|\parbreak
|    \matrix{\hbox{an arc}\hfill\cr|\parbreak
|      \hbox{an arc or a circle}\hfill\cr|\parbreak
|      \hbox{a fan or an arc}\hfill\cr}\right\}.$$|\par
\smallskip\noindent
The first example includes |\!| and |\,| to give slightly refined spacing;
but the point of the problem was to illustrate the hbox, not to fuss over
such extra details.
The last example can be done much more simply using the ideas of
Chapter~22, if you don't mind descending to the level of \TeX\ primitives;
for example, the first matrix could be replaced by ^^|\halign|
\begintt
\,\vcenter{\halign{#\hfil\cr an arc\cr a circle\cr a fan\cr}}\,
\endtt
and the second is similar.

\ansno19.2:
 |$$\textstyle y={1\over2}x$$|. \ (Switching to text style is
especially common in multiline formulas. For example, you will probably
find occasions to use ^|\textstyle| on both sides of the |&|'s within
an ^|\eqalign|.)

\ansno19.3:
 The latter formula will be in text style, not display style.
And even if you do type `|$$\hbox{$\displaystyle{|\<formula>|}$}$$|', the
results are not quite the same, as we will see later: \TeX\ will compress
the glue in `|$$|\<formula>|$$|' if the formula is too wide to fit on
a line at its natural width, but the glue inside |\hbox{...}| is frozen
at its natural width. The |\hbox| version also invokes |\everymath|.

\ansno19.4:
 One solution is to put the formula in an hbox that occupies a full line:
\begintt
$$\leftline{\indent$\displaystyle
    1-{1\over2}+{1\over3}-{1\over4}+\cdots=\ln2$}$$
\endtt
But this takes a bit of typing. If you make the definitions
\begintt
\def\leftdisplay#1$${\leftline{\indent$\displaystyle{#1}$}$$}
\everydisplay{\leftdisplay}
\endtt
you can type `|$$|\<formula>|$$|' as usual, and the formatting will be
inserted automatically. \ (This doesn't work with equation numbers;
Appendix~D illustrates how to handle them as well.)

\ansno19.5:
 |$$\prod_{k\ge0}{1\over(1-q^kz)}=|\parbreak
        |   \sum_{n\ge0}z^n\bigg/\!\!\prod_{1\le k\le n}(1-q^k).\eqno(16')$$|

\ansno19.6:
 |\eqno\hbox{(3--1)}|.

\ansno19.7:
 When you type an ^{asterisk} in math mode, plain \TeX\ considers
|*| to be a binary operation. In the cases `|(*)|' and `|(**)|', the
binary operations are converted to type~Ord, because they don't appear in
a binary context; but the middle asterisk in `|(***)|' remains of type~Bin.
So the result was `$(***)$'. To avoid the extra medium spaces, you can
type `|\eqno(*{*}*)|'; or you can change ^|\mathcode||`*|, if you never use
|*| as a binary operation.

\ansno19.8:
 Assuming that |\hsize| is less than $10000\pt$, the natural width of
this equation will be too large to fit on a line; also, |\quad| specifies
glue at the left. Therefore `$x=y$' will appear exactly $1\em$ from
the left, and `(5)' will appear flush right. \ (The widths will satisfy
^^{displays, non-centered} $w=z-q$, $d=0$, $k=q-e=18\rm\,mu$.) \
In the case of\/ |\leqno|, `(5)' will appear flush left, followed by
one quad of space in |\textfont2|, followed by one quad of space in the
current text font, followed by `$x=y$'.

\ansno19.9:
 (Note in particular that the final `|.|'\ comes {\sl before\/} the
final `|\cr|'.)
\begintt
$$\eqalign{T(n)\le T(2^{\lceil\lg n\rceil})
    &\le c(3^{\lceil\lg n\rceil}-2^{\lceil\lg n\rceil})\cr
    &<3c\cdot3^{\lg n}\cr
    &=3c\,n^{\lg3}.\cr}$$
\endtt

\ansno19.10:
 |$$\eqalign{P(x)&=a_0+a_1x+a_2x^2+\cdots+a_nx^n,\cr|\parbreak
        |   P(-x)&=a_0-a_1x+a_2x^2-\cdots+(-1)^na_nx^n.\cr}\eqno(30)$$|\par

\ansno19.11:
 Both sides of that equation are considered to be on the left, so
you get results that look like this:
$$\openup-\jot
\left\{\eqalign{\alpha&=f(z)\cr \beta&=f(z^2)\cr \gamma=f(z^3)\cr}
  \right\}.$$

\ansno19.12:
 |$$\leqalignno{\gcd(u,v)&=\gcd(v,u);&(9)\cr|\parbreak
        |    \gcd(u,v)&=\gcd(-u,v).&(10)\cr}$$|

\ansno19.13:
 %
|$$\eqalignno{\biggl(\int_{-\infty}^\infty e^{-x^2}\,dx\biggr)^2|\parbreak
        |  &=\int_{-\infty}^\infty\int_{-\infty}^\infty|\parbreak
        |    e^{-(x^2+y^2)}\,dx\,dy\cr|\parbreak
        |  &=\int_0^{2\pi}\int_0^\infty e^{-r^2}r\,dr\,d\theta\cr|\parbreak
        |  &=\int_0^{2\pi}\biggl(-{e^{-r^2}\over2}|\parbreak
        |    \bigg|\||_{r=0}^{r=\infty}\,\biggr)\,d\theta\cr|\parbreak
        |  &=\pi.&(11)\cr}$$| ^^|\bigg|

\ansno19.14:
 You get the displayed box
$$\eqalign{x&=y+z\cr
  \noalign{\hbox{and}}
  x^2&=y^2+z^2.\cr}$$
Reason: The `and' occurs at the left of the |\eqalign| box, not at the
left of the whole display, and the |\eqalign| box is centered as usual.

\ansno19.15:
 By raising the equation number, he increased the line height,
so \TeX\ put extra space between that line and the previous line
when it calculated the inter-line glue. If he had said
`^|\smash||{\raise...}|', he wouldn't have had that problem.

\ansno19.16:
 |$$\displaylines{\hfill x\equiv x;\hfill\llap{(1)}\cr|\parbreak
        |   \hfill\hbox{if}\quad x\equiv y\quad\hbox{then}\quad|\parbreak
        |      y\equiv x;\hfill\llap{(2)}\cr|\parbreak
        |   \hfill\hbox{if}\quad x\equiv y\quad\hbox{and}\quad|\parbreak
        |      y\equiv z\quad\hbox{then}\quad|\parbreak
        |      x\equiv z.\hfill\llap{(3)}\cr}$$|\par\medskip\noindent
There's also a trickier solution, which begins with
\begintt
$$\displaylines{x\equiv x;\hfil\llap{(1)}\hfilneg\cr
\endtt

\ansno19.17:
 |$$\eqalignno{x_nu_1+\cdots+x_{n+t-1}u_t|\parbreak
        |   &=x_nu_1+(ax_n+c)u_2+\cdots\cr|\parbreak
        |   &\qquad+\bigl(a^{t-1}x_n+c(a^{t-2}+\cdots+1)\bigr)u_t\cr|\parbreak
        |   &=(u_1+au_2+\cdots+a^{t-1}u_t)x_n+h(u_1,\ldots,u_t).|\parbreak
        |     \quad&(47)\cr}$$|\par\noindent
You weren't expected to insert the `|\quad|' on the last line; such
refinements usually can't be anticipated until you see the first proofs.
But without that |\quad| the `(47)' would occur half a quad closer to the
formula.

\ansno19.18:
 |$$\displaylines{\quad\sum_{1\le j\le n}{1\over|\parbreak
        |    (x_j-x_1)\ldots(x_j-x_{j-1})(x-x_j)(x_j-x_{j+1})|\parbreak
        |    \ldots(x_j-x_n)}\hfill\cr|\parbreak
        |  \hfill={1\over(x-x_1)\ldots(x-x_n)}.\quad(27)\cr}$$|

\ansno19.19:
 |$$\def\\#1;{(#1;q^2)_\infty} % to save typing|\parbreak
        |\displaystyle{q^{{1\over2}n(n+1)}\\ea;\\eq/a;\qquad\atop|\parbreak
        |  \hfill\\caq/e;\\cq^2\!/ae;}|\parbreak
        |\over(e;q)_\infty(cq/e;q)_\infty$$|

\ansno20.1:
 |\def\mustnt{I must not talk in class.\par}|\parbreak
        |\def\five{\mustnt\mustnt\mustnt\mustnt\mustnt}|\parbreak
        |\def\twenty{\five\five\five\five}|\parbreak
        |\def\punishment{\twenty\twenty\twenty\twenty\twenty}|\par
\smallskip\noindent Solutions to more complicated problems of this type are
discussed later.

\ansno20.2:
 |ABCAB|. \ (The first |\a| expands into |A\def\a{B...}|; this redefines
|\a|, so the second |\a| expands into |B...|, etc.) \ At least, that's what
happens if\/ |\puzzle| is encountered when \TeX\ is building a list. But if
|\puzzle| is expanded in an ^|\edef| or ^|\message| or something like that,
we will see later that the interior |\def| commands are not performed
while the expansion is taking place, and the control sequences following
|\def| are expanded; so the result is an infinite string
\begintt
A\def A\def A\def A\def A\def A\def A\def A\def A...
\endtt
which causes \TeX\ to abort because the program's input stack is finite.
This example points out that a control sequence (e.g., |\b|) need not be
defined when it appears in the replacement text of a definition. The example
also shows that \TeX\ doesn't expand a macro until it needs to.

\ansno20.3:
 \def\row#1{(#1_1,\ldots,#1_n)}$\row{\bf x}$. Note that the
subscripts are bold here, because the expansion |(\bf x_1,\ldots,\bf x_n)|
doesn't ``turn off'' ^|\bf|. To prevent this, one should write
|\row{{\bf x}}|; or (better), |\row\xbold|, in conjunction with
|\def\xbold{{\bf x}}|.

\ansno20.4:
 The catch is that the parameters have to percolate down to the
|\mustnt| macro, if you extend the previous answer:
\begintt
\def\mustnt#1#2{I must not #1 in #2.\par}
\def\five#1#2{\mustnt{#1}{#2}...\mustnt{#1}{#2}}
\def\twenty#1#2{\five{#1}{#2}...\five{#1}{#2}}
\def\punishment#1#2{\twenty{#1}{#2}...\twenty{#1}{#2}}
\endtt
When you pass parameters from one macro to another in this way, you need to
enclose them in braces as shown. But actually this particular solution
punishes \TeX\ much more than it needs to, because it takes a lot of
time to copy the parameters and read them again and again. There's a
much more efficient way to do the job, by defining control sequences:
\begintt
\def\mustnt{I must not \doit\ in \thatplace.\par}
\def\punishment#1#2{\def\doit{#1}\def\thatplace{#2}%
  \twenty\twenty\twenty\twenty\twenty}
\endtt
and by defining |\five| and |\twenty| without parameters as before.
You can also delve more deeply into \TeX nicalities, constructing solutions
that are more efficient yet; \TeX\ works even faster when macros
communicate with each other via ^{boxes}.
^^{efficient macros} ^^{communication between macros}
For example,
\begintt
\def\mustnt{\copy0 }
\def\punishment#1#2{\setbox0=
  \vbox{\strut I must not #1 in #2.\strut}%
  \twenty\twenty\twenty\twenty\twenty}
\endtt
sets 100 identical paragraphs at high speed, because \TeX\ has to
process the paragraph and break it into lines only once. It's much faster
to ^{copy a box} than to build it up from scratch. \ (The ^{struts} in
this example keep the interbaseline distances correct between boxed
paragraphs, as explained in Chapter~12. Two struts are used, for if the
message takes more than one line there will be a strut at both top
and bottom. If it were known that each sentence will occupy only a single
line, no struts would be needed, because interline glue is added as
usual when a box created by |\copy| is appended to the current vertical list.)

\ansno20.5:
 The |##| feature is indispensable when the replacement text of
a definition contains other definitions. For example, consider
\begintt
\def\a#1{\def\b##1{##1#1}}
\endtt
after which `|\a!|'\ will expand to `|\def\b#1{#1!}|'. We will see later
that |##| is also important for alignments; see, for example, the definition
of\/ |\matrix| in Appendix~B.

\ansno20.6:
 |\def\a#{\b}|.

\ansno20.7:
 Let's go slowly on this one, so that the answer will give enough
background to answer all similar questions. The \<parameter text> of the
definition consists of the three tokens |#1|, |#2|, |[|$_1$; the
\<replacement text> consists of the six tokens |{|$_1$, |#|$_6$, |]|$_2$,
|!|$_6$, |#2|, |[|$_1$. \ (When two tokens of category~6 occur in the
replacement text, the character code of the second one survives; the
character code of a category-6 character is otherwise irrelevant. Thus,
`|\def\!#1!2#[{##]!!#2]|' would produce an essentially identical
definition.) \ When expanding the given token list, argument~|#1| is
|x|$_{11}$, since it is undelimited. Argument~|#2| is delimited by~|[|$_1$,
which is different from~|{|$_1$, so it is set provisionally to |{[y]]|;
but the outer ``braces'' are stripped off, so |#2|~reduces to the
three tokens |[|$_1$, |y|$_{11}$,~|]|$_2$. The result of the expansion
is therefore
\begindisplay
|{|$_1$ |#|$_6$ |]|$_2$ |!|$_6$ |[|$_1$ |y|$_{11}$ |]|$_2$
  |[|$_1$ |z|$_{11}$ |}|$_2$.
\enddisplay
Incidentally, if you display this with ^|\tracingmacros||=1|, \TeX\ says
\begintt
\!!1#2[->{##]!!#2[
#1<-x
#2<-[y]
\endtt
Category codes are not shown, but a character of category~6 always
appears twice in succession. A parameter token in the replacement text
uses the character code of the final parameter in the parameter text.
^^{token lists, as displayed by TeX}

\ansno20.8:
 Yes indeed. In the first case, |\a| receives the meaning of\/~|\b|
that is current at the time of the |\let|. In the second case, |\a|~becomes
a~macro that will expand into the token~|\b| whenever |\a|~is used,
so it has the meaning of\/~|\b| that is current at the time of use.
You need |\let|, if you want to interchange the meanings of\/ |\a| and~|\b|.

\ansno20.9:
 (a) Yes. \ (b) No; any other control sequence can appear
(except those declared as |\outer| macros).

\ansno20.10:
 |\def\overpaid{{\count0=\balance|\parbreak
        |  You have overpaid your tax by \dollaramount.|\parbreak
        |  \ifnum\count0<100 It is our policy to refund|\parbreak
        |    such a small amount only if you ask for it.|\parbreak
        |  \else A check for this amount is being mailed|\parbreak
        |    under separate cover.\fi}}|

\ansno20.11:
 The tricky part is to get the zero in an amount like `|$2.01|'.
\begintt
\def\dollaramount{\count2=\count0 \divide\count2 by100
  \$\number\count2.%
  \multiply\count2 by-100 \advance\count2 by\count0
  \ifnum \count2<10 0\fi
  \number\count2 }
\endtt

\ansno20.12:
 |\def\category#1{\ifcase\catcode`#1|\parbreak
        |  escape\or begingroup\or endgroup\or math\or|\parbreak
        |  align\or endline\or parameter\or superscript\or|\parbreak
        |  subscript\or ignored\or space\or letter\or|\parbreak
        |  otherchar\or active\or comment\or invalid\fi}|\par

\ansno20.13:
 (a,b)~True. (c,d)~False. (e,f)~True. In case~(e), the \<true text>
starts with `|ue|'.  (g)~The |\ifx| is false and the inner |\if| is true;
so the outer |\if| becomes `|\if True...|', which is false. \
(Interestingly, \TeX\ knows that the outer |\if| is false even before it
has looked at the |\fi|'s that close the |\ifx| and the inner |\if|.)

\ansno20.14:
 One idea is to say
\begintt
\let\save=\c \let\c=0 \edef\a{\b\c\d} \let\c=\save
\endtt
because control sequences equivalent to characters are not expandable.
However, this doesn't expand occurrences of~|\c|
that might be present in the expansions of\/ |\b| and~|\d|. Another way,
which is free of this defect, is
\begintt
\edef\next#1#2{\def#1{\b#2\d}} \next\a\c
\endtt
(and it's worth a close look!).

\ansno20.15:
 |\toks0={\c} \toks2=\expandafter{\d}|\parbreak
        |\edef\a{\b\the\toks0 \the\toks2 }|
\smallskip\noindent
(Notice that ^|\expandafter| expands the token after the left brace here.)

\ansno20.16:
 The following shouldn't be taken too seriously, but it does work:
^^|\span|
\begintt
{\setbox0=\vbox{\halign{#{\c\span\d}\cr
      \let\next=0\edef\next#1{\gdef\next{\b#1}}\next\cr}}}
\let\a=\next
\endtt

\ansno20.17:
 Neither one, although |\a| will behave like an unmatched left
brace when it is expanded. The definition of\/ |\b| is {\sl not complete},
because it expands to `|\def\b{{}|'; \TeX\ will continue to read ahead,
looking for another right brace, possibly discovering a runaway
definition! It's impossible to define a macro that has unmatched braces.
But you {\sl can\/} say |\let\a={|; Appendix~D discusses several
other ^{brace tricks}.

\ansno20.18:
 One way is to redefine |\catcode`\^^M=9| (ignored) just before
the |\read|, so that the \<return> will be ignored. Another solution is
to redefine ^|\endlinechar||=-1|, so that no character is put at the
end of the line. Or you could try to be tricky by stripping off the
space with macro expansion as follows:
\begintt
\def\stripspace#1 \next{#1}
\edef\myname{\expandafter\stripspace\myname\next}
\endtt
The latter solution doesn't work if the user types `|%|' at the end of
his~or her name, or if the name contains control sequences.

\ansno20.19:
 Here are two solutions:
\begintt
\def\next#1\endname{\uppercase{\def\MYNAME{#1}}}
\expandafter\next\myname\endname
|smallskip\edef\next{\def\noexpand\MYNAME{\myname}}
\uppercase\expandafter{\next}
\endtt

\ansno20.20:
 (Here's a solution that also numbers the lines, so that the number of
repetitions is easily verifiable.
The only tricky part about this answer is the use of\/ ^|\endgraf|,
which is a substitute for |\par| because |\loop| is not a ^|\long| macro.)
\begintt
\newcount\n
\def\punishment#1#2{\n=0
  \loop\ifnum\n<#2 \advance\n by1
    \item{\number\n.}#1\endgraf\repeat}
\endtt

\ansno21.1:
 The interline skip is added for vboxes, but not for rules; he
forgot to say ^|\nointerlineskip|, before and after the |\moveright|
construction.

\ansno21.2:
 |\vrule height3pt depth-2pt width1in|. Notice that it was necessary
to call it a |\vrule| since it appeared in horizontal mode.

\ansno21.3:
 |\def\boxit#1{\vbox{\hrule\hbox{\vrule\kern3pt|\parbreak
        |      \vbox{\kern3pt#1\kern3pt}\kern3pt\vrule}\hrule}}|\par
\smallskip\noindent
(The resulting box does not have the baseline of the original one;
you have to work a little bit harder to get that.)

\ansno21.4:
 |\leaders|: two boxes starting at $100\pt$, $110\pt$.\par
|\cleaders|: three boxes starting at $95\pt$, $105\pt$, $115\pt$.\par
|\xleaders|: three boxes starting at $93\pt$, $105\pt$, $117\pt$.

\ansno21.5:
 |\def\leaderfill{\kern-0.3em\leaders\hbox to 1em{\hss.\hss}%|\parbreak
        |    \hskip0.6em plus1fill \kern-0.3em }|

\ansno21.6:
 Since no |height| or |depth| specification follows the |\vrule|,
the height and depth are `|*|'; i.e., the rule extends to the smallest
enclosing box. This usually makes a heavy black band, which is too
horrible to demonstrate here. However, it does work if you know that the
enclosing box is sufficiently small; and |\leaders\vrule\vfill| works fine in
vertical mode.

\ansno21.7:
 For example, say
\begintt
\null\nobreak\leaders\hrule\hskip10pt plus1filll\ \par
\endtt
The `|\|\]' provides extra glue that is wiped out by the implied |\unskip|
at the end of every paragraph (see Chapter~14), and the `|\null\nobreak|'
makes sure that the leaders do not disappear at a line break; `^|filll|'
overtakes the ^|\parfillskip| glue.

\ansno21.8:
 |$$\hbox to 2.5in{\cleaders|\parbreak
        |    \vbox to .5in{\cleaders\hbox{\TeX}\vfil}\hfil}$$|

\ansno21.9:
 We assume that a strut is $12\pt$ tall, and that 50 lines
fit on a page:
\begintt
\setbox0=\hbox{\strut I must not talk in class.}
\null\cleaders\copy0\vskip600pt\vfill\eject % 50 times on page 1;
\null\cleaders\box0\vskip600pt\bye % 50 more on page 2.
\endtt
The ^|\null| keeps glue (and leaders) from disappearing at the top of
the page.

\ansno21.10:
 |{\let|\stretch|\the=0\edef|\stretch|\next|\stretch
|{\write|\stretch|\cont|\stretch|{|\<token list>|}}\next|\stretch|}| will
expand everything but |\the| when the |\write| command is given.

\ansno22.1:
 Notice the uses of `|\smallskip|' here to separate the table heading
and footing from the table itself; such refinements are often worthwhile.
\begintt
\settabs\+\indent&10\frac1/2 lbs.\qquad&\it Servings\qquad&\cr
\+&\negthinspace\it Weight&\it Servings&
  {\it Approximate Cooking Time\/}*\cr
\smallskip
\+&8 lbs.&6&1 hour and 50 to 55 minutes\cr
\+&9 lbs.&7 to 8&About 2 hours\cr
\+&9\frac1/2 lbs.&8 to 9&2 hours and 10 to 15 minutes\cr
\+&10\frac1/2 lbs.&9 to 10&2 hours and 15 to 20 minutes\cr
\smallskip
\+&* For a stuffed goose,
  add 20 to 40 minutes to the times given.\cr
\endtt
\def\frac#1/#2{\leavevmode\kern.1em
  \raise.5ex\hbox{\the\scriptfont0 #1}\kern-.1em
  /\kern-.15em\lower.25ex\hbox{\the\scriptfont0 #2}}%
The title line specifies `|\it|' three times, because each entry between
tabs is treated as a group by \TeX; you would get error messages galore
if you tried to say something like \hbox{`|\+&{\it Weight&Servings&...}\cr|'}.
The `^|\negthinspace|' in the title line is a small backspace that
compensates for the slant in an italic {\it W\/}; the author inserted
this somewhat unusual correction after seeing how the table looked
without it, on the first proofs. \ (You weren't supposed to think of this,
but it has to be mentioned.) \ See exercise 11.\fracexno\ for the `|\frac|'
macro; it's better to say `\frac1/2' than `$1\over2$', in a cookbook.\par
Another way to treat this table would be to display it in a vbox, instead
of including a first column whose sole purpose is to specify indentation.

\ansno22.2:
 In such programs it seems best to type |\cleartabs| just before |&|,
whenever it is desirable to reset the old tabs. Multiletter identifiers look
best when set in ^{text italics} with ^|\it|, as explained in Chapter~18.
Thus, the following is recommended:
\begintt
\+\bf while $p>0$ do\cr
  \+\quad\cleartabs&{\bf begin} $q:={\it link}(p)$;
    ${\it free\_node}(p)$; $p:=q$;\cr
  \+&{\bf end};\cr
\endtt

\ansno22.3:
 Here we retain the idea that |&| inserts a new tab, when there
are no tabs to the right of the current position. Only one of the macros
that are used to process |\+|~lines needs to be changed; but
(unfortunately) it's the most complex one:
\begintt
\def\t@bb@x{\if@cr\egroup % now \box0 holds the column
  \else\hss\egroup \dimen@=0\p@
    \dimen@ii=\wd0 \advance\dimen@ii by1sp
    \loop\ifdim \dimen@<\dimen@ii
      \global\setbox\tabsyet=\hbox{\unhbox\tabsyet
        \global\setbox1=\lastbox}%
      \ifvoid1 \advance\dimen@ii by-\dimen@
        \advance\dimen@ii by-1sp \global\setbox1
          =\hbox to\dimen@ii{}\dimen@ii=-1pt\fi
      \advance\dimen@ by\wd1 \global\setbox\tabsdone
        =\hbox{\box1\unhbox\tabsdone}\repeat
    \setbox0=\hbox to\dimen@{\unhbox0}\fi
  \box0}
\endtt

\ansno22.4:
 \par\nobreak\vskip-\baselineskip
\halign{\indent\hfil#&\quad#\hfil\cr
Horizontal lists&Chapter 14\cr\noalign{\nobreak}
Vertical lists&Chapter 15\cr\noalign{\nobreak}
Math lists&Chapter 17\qquad (i.e., the first column would be
  right-justified)\cr}

\ansno22.5:
 |Fowl&Poule de l'Ann\'ee&10 to 12&Over 3&Stew, Fricassee\cr|

\ansno22.6:
 |$$\halign to\hsize{\sl#\hfil\tabskip=.5em plus.5em&|\parbreak
        |   #\hfil\tabskip=0pt plus.5em&|\parbreak
        |   \hfil#\tabskip=1em plus2em&|\parbreak
        |  \sl#\hfil\tabskip=.5em plus.5em&|\parbreak
        |   #\hfil\tabskip=0pt plus.5em&|\parbreak
        |   \hfil#\tabskip=0pt\cr ...}$$|

\ansno22.7:
 The trick is to define a new macro for the preamble:
\begintt
$$\def\welshverb#1={{\bf#1} = }
\halign to\hsize{\welshverb#\hfil\tabskip=1em plus1em&
  \welshverb#\hfil&\welshverb#\hfil\tabskip=0pt\cr ...}$$
\endtt

\ansno22.8:
 |\hfil#: &\vtop{\parindent=0pt\hsize=16em|\parbreak
        |    \hangindent.5em\strut#\strut}\cr|\par\nobreak\medskip\noindent
With such narrow measure and such long words, the ^|\tolerance| should probably
also have been increased to, say, 1000 inside the ^|\vtop|; luckily it turned
^^|\strut| out that a higher tolerance wasn't needed.\par
{\sl Note:\/} The stated preamble solves the problem and demonstrates
that \TeX's line-breaking capability can be used within tables. But this
particular table is not really a good example of the use of\/ |\halign|,
because \TeX\ could typeset it directly, using ^|\everypar| in an
appropriate manner to set up the hanging indentation, and using |\par|
instead of\/ |\cr|. For example, one could say
\begintt
\hsize20em \parindent0pt \clubpenalty10000 \widowpenalty10000
\def\history#1&{\hangindent4.5em
  \hbox to4em{\hss#1: }\ignorespaces}
\everypar={\history} \def\\{\leavevmode{\it c\/}}
\endtt
which spares \TeX\ all the work of\/ |\halign| but yields essentially the
same result. ^^|\leavevmode|

\ansno22.9:
 The equation is divided into separate parts for terms and
plus/minus signs, and tabskip glue is used to center it:
\begintt
$$\openup1\jot \tabskip=0pt plus1fil
\halign to\displaywidth{\tabskip=0pt
  $\hfil#$&$\hfil{}#{}$&
  $\hfil#$&$\hfil{}#{}$&
  $\hfil#$&$\hfil{}#{}$&
  $\hfil#$&${}#\hfil$\tabskip=0pt plus1fil&
  \llap{#}\tabskip=0pt\cr
10w&+&3x&+&3y&+&18z&=1,&(9)\cr
6w&-&17x&&&-&5z&=2.&(10)\cr}$$
\endtt

\ansno22.10:
 |\hfil# &#\hfil&\quad#&\ \hfil#&\ \hfil#\cr|

\ansno22.11:
 |\pmatrix{a_{11}&a_{12}&\ldots&a_{1n}\cr|\parbreak
        |    a_{21}&a_{22}&\ldots&a_{2n}\cr|\parbreak
        |    \multispan4\dotfill\cr|\parbreak
        |    a_{m1}&a_{m2}&\ldots&a_{mn}\cr}|

\ansno22.12:
 `|\cr|' would have omitted the final column, which is a vertical rule.

\ansno22.13:
 One way is to include two lines just before and after the title
line, saying `|\omit&height2pt&\multispan5&\cr|'. Another way is to
put |\bigstrut| into some column of the title line, for some appropriate
invisible box |\bigstrut| of width zero. Either way makes the table
look better.

\ansno22.14:
 The trick is to have ``empty'' columns at the extreme left and right;
then the |\hrulefill|'s are able to span the tabskip glue.
\begintt
$$\vbox{\tabskip=0pt \offinterlineskip
\halign to 36em{\tabskip=0pt plus1em#&
  #\hfil&#&#\hfil&#&#\hfil&#\tabskip=0pt\cr
&&&&&\strut J. H. B\"ohning, 1838&\cr
&&&&\multispan3\hrulefill\cr
&&&\strut M. J. H. B\"ohning, 1882&\vrule\cr
&&\multispan3\hrulefill\cr
&&\vrule&&\vrule&\strut M. D. Blase, 1840&\cr
&&\vrule&&\multispan3\hrulefill\cr
&\strut L. M. Bohning, 1912&\vrule\cr
\multispan3\hrulefill\cr
&&\vrule&&&\strut E. F. Ehlert, 1845&\cr
&&\vrule&&\multispan3\hrulefill\cr
&&\vrule&\strut P. A. M. Ehlert, 1884&\vrule\cr
&&\multispan3\hrulefill\cr
&&&&\vrule&\strut C. L. Wischmeyer, 1850&\cr
&&&&\multispan3\hrulefill\cr}}$$
\endtt

\ansno22.$\infty $:
 (Solution to Dudeney's problem.) \
Let |\one| and |\two| be macros that produce a vertical list
denoting one and two pennies, respectively. The problem can be
solved with ^|\valign| as follows:
\begintt
\valign{\vfil#&\vfil#&\vfil#&\vfil#\cr
  \two&\one&\one&\one\cr
  \one&\one&\two&\one\cr
  \one&\one&\one&\two\cr
  \one&\two&\one&\one\cr}
\endtt
Since |\valign| transposes rows and columns, the result is\quad
\setbox0=\hbox{\vbox{
    \def\pennytop{\hbox to 24pt{\manual\char'130\hfil}}%
      \def\pennyedge{\hbox{\manual\char'133}}%
      \def\one{\pennytop\pennyedge}%
      \def\two{\one\pennyedge}%
      \baselineskip0pt\lineskip0pt\tabskip=14pt
    \hbox{\valign{\vfil#&\vfil#&\vfil#&\vfil#\cr
        \two&\one&\one&\one\cr
        \one&\one&\two&\one\cr
        \one&\one&\one&\two\cr
        \one&\two&\one&\one\cr}}\kern-11pt}}%
\ht0=0pt \dp0=11pt \box0.

\ansno23.1:
 |\footline={\hss\tenrm-- \folio\ --\hss}|

\ansno23.2:
 |\headline={\ifnum\pageno=1 \hss\tenbf R\'ESUM\'E\hss|\parbreak
        |  \else\tenrm R\'esum\'e of A. U. Thor \dotfill\ Page \folio\fi}|
\smallskip\noindent (You should also say |\nopagenumbers| and
|\voffset=2\baselineskip|.)

\ansno23.3:
 |\output={\plainoutput\blankpageoutput}|\parbreak
        |\def\blankpageoutput{\shipout\vbox{\makeheadline|\parbreak
        |    \vbox to\vsize{}\makefootline}\advancepageno}|

\ansno23.4:
 Set |\hsize=2.1in|, allocate `|\newbox\midcolumn|', and use the
following code:
\begintt
\output={\if L\lr
    \global\setbox\leftcolumn=\columnbox \global\let\lr=M
  \else\if M\lr
    \global\setbox\midcolumn=\columnbox \global\let\lr=R
  \else \tripleformat \global\let\lr=L\fi\fi
  \ifnum\outputpenalty>-20000 \else\dosupereject\fi}
\def\tripleformat{\shipout\vbox{\makeheadline
    \fullline{\box\leftcolumn\hfil\box\midcolumn\hfil\columnbox}
    \makefootline}
  \advancepageno}
\endtt
At the end, |\supereject| and say `|\if L\lr \else\null\vfill\eject\fi|'
twice.

\ansno23.5:
 He forgot that ^{interline glue} is inserted automatically before
the |\leftline|; this permits a legal breakpoint between the |\mark| and
the |\leftline| box, according to the rules of page breaking in Chapter~15.
One cure would be to say ^|\nobreak| just after the |\mark|; but it's
usually best to put marks and ^{insertions} just {\sl after\/} boxes.

\ansno23.6:
 Say, for example, |\ifcase2\expandafter\relax\botmark\fi| to
read part $\alpha_2$ of\/ |\botmark|. Another solution puts the five
components into five parameters of a macro, analogous to the method
used by |\inxcheck| later in this chapter; but the |\ifcase| approach
is usually more efficient, because it lets \TeX\ pass over the unselected
components at high speed.

\ansno23.7:
 |\output={\dimen0=\dp255 \normaloutput|\parbreak
        |  \ifodd\pageno\else\if L\lr|\parbreak
        |    \expandafter\inxcheck\botmark\sub\end\fi\fi}|
\smallskip\noindent In this case the |\normaloutput| macro should be the
two-column output routine that was described earlier in this chapter, beginning
with `|\if L\lr|' and ending with `|\let\lr=L\fi|'. \ (There is no need
to test for |\supereject|.)

\ansno23.8:
 False. If the text of the main and/or subsidiary entry is lengthy,
a continuation line may actually become two or more lines. \ (Incidentally,
hanging indentation will then occur, because the |\everypar| command---which
was set up outside the |\output| routine---is effective inside.) \ The
|\vsize| must be large enough to accommodate all continuation lines plus
at least one more line of index material, or else infinite looping will occur.

\ansno24.1:
 If\/ |\cs| has been defined by ^|\chardef| or ^|\mathchardef|, \TeX\
uses ^{hexadecimal notation} when it expands ^|\meaning||\cs|, and it
assigns category~12 to each digit of the expansion. You might have an
application in which you want the last part of the expansion to be treated
as a \<number>. \ (This is admittedly an obscure reason.)

\ansno24.2:
 Yes; any number of spaces can precede any keyword.

\ansno24.3:
 The first two have the same meaning; but the third coerces
|\baselineskip| to a \<dimen> by suppressing the stretchability
and shrinkability that might be present.

\ansno24.4:
 The natural width is $221\rm\,dd$ (which \TeX\ rounds to
$15497423\rm\,sp$ and displays as |236.47191pt|).
The stretchability is $2500\rm\,sp$, since an
internal integer is coerced to a dimension when it appears as an
^^{coerce <number> to <dimen>}
\<internal unit>. The shrinkability is zero. Notice that the final |\space|
is swallowed up as part of the optional ^{spaces} of the \<shrink> part
in the syntax for \<glue>. \ (If |PLUS| had been |MINUS|, the final |\space|
would {\sl not\/} have been part of this \<glue>!)

\ansno24.5:
 If it was non-null when a ^|\dump| operation occurred. Here's
^^|\jobname|
a nontrivial example, which sets up ^|\batchmode| and puts ^|\end| at the
end of the input file:
\begintt
\everyjob={\batchmode\input\jobname\end}
\endtt

\ansno24.6:
 (a) |\def\\#1\\{}\futurelet\cs\\|\]|\\|. (b) |\def\\{\let\cs= }\\|\].
\ (There are many other solutions.)

\ansno24.7:
 \<internal quantity>\is\<internal integer>\alt
\<internal dimen>\parbreak\qquad\alt\<internal glue>\alt\<internal muglue>\alt
\<internal nonnumeric>\parbreak \<internal nonnumeric>\is\<token
variable>\alt \<font>

\ansno26.1:
 Radix 10 notation is used for numeric constants and for the output
of numeric data. The first 10 |\count| registers are displayed at each
|\shipout|, and their values are recorded on the |dvi| file at such times.
% Also, TeX was first implemented on a DEC-10.
% The \catcode for <space> is 10.
% My birthday is January 10.
% Can this all be just a coincidence?
A box whose glue has stretched or shrunk to its stated stretchability
or shrinkability has badness 100; this badness value separates ``loose''
boxes from ``very loose'' or ``underfull'' ones. \TeX\ will scroll up
to 100 errors in a single paragraph before giving up (see Chapter~27).
The normal values of\/ |\spacefactor| and |\mag| are 1000. A~|\prevdepth|
value of $-1000\pt$ suppresses interline glue. The badness rating of a
box is at most 10000, except that the |\badness| of
an overfull box is 1000000.  |INITEX| initializes |\tolerance| to
10000, thereby making all line breaks feasible. Penalties of 10000 or more
prohibit breaks; penalties of $-10000$ or less make breaks mandatory. The
cost of a page break is 100000, if the badness is 10000 and if the
associated penalties are less than 10000 in magnitude (see Chapter~15).

\ansno26.2:
 \TeX\ allows constants to be expressed in radix 8 (octal)
or radix~16 (hexadecimal) notation, and it uses hexadecimal notation to
display |\char| and |\mathchar| codes. There are 16 families for math
fonts, 16 input streams for |\read|, 16 output streams for |\write|.
A |\catcode| value must be less than~16. The notation |^^?|, |^^@|,
|^^A| specifies characters whose codes differ by~64
from the codes of |?|, |@|, |A|; this convention applies only to
characters with ASCII codes less than~128. There are 256 possible characters,
hence 256 entries in each of the |\catcode|, |\mathcode|,
|\lccode|, |\uccode|, |\sfcode|, and |\delcode| tables. All
|\lccode|, |\uccode|, and |\char| values
must be less than~256. A font has at most 256 characters. There are
256~|\box| registers, 256~|\count| registers, 256~|\dimen| registers,
256~|\skip| registers, 256~|\muskip| registers, 256~|\toks| registers,
256~hyphenation tables.
The ``at size'' of a font must be less than~$2048\pt$, i.e.,~$2^{11}\pt$.
Math delimiters are encoded by multiplying the math~code of the ``small
character'' by~$2^{12}$. The magnitude of
a~\<dimen> value must be less than~$16384\pt$, i.e.,~$2^{14}\pt$;
similarly, the \<factor> in a~\<fil dimen> must be less than~$2^{14}$.
A~|\mathchar| or |\spacefactor| or |\sfcode| value must be less than~$2^{15}$;
a~|\mathcode| or |\mag| value must be less than or equal to~$2^{15}$,
and $2^{15}$ denotes an ``active'' math character. There
are $2^{16}\rm\,sp$ per~pt. A~|\delcode| value
must be less than~$2^{24}$; a~|\delimiter|, less than $2^{27}$.
The |\end| command sometimes contributes
a penalty of $-2^{30}$ to the current page. A~\<dimen> must be less than
$2^{30}\rm\,sp$ in absolute value; a~\<number> must be
less than $2^{31}$ in absolute value.

\ansno27.1:
 He forgot to count the space; \TeX\ deleted `|i|', `|m|',
`\]', `|\input|', and four letters. (But all is not lost; he can
type `|1|' or `|2|', then \<return>, and after being prompted by `|*|' he
can enter a new line of input.)

\ansno27.2:
 First delete the unwanted tokens, then insert what you want:
Type `|6|' and then `|I\macro|'. \ (Incidentally, there's a sneaky
way to get at the |\inaccessible| control sequence by typing
\begintt
I\garbage{}\let\accessible=
\endtt
in response to an error message like this. The author ^^{Knuth} designed
\TeX\ in such a way that you can't destroy anything by playing such
nasty tricks.)

\ansno27.3:
 `|I|\]|%|' does the trick, if |%| is a ^{comment character}.

\ansno27.4:
 The `^|minus|' of `|minuscule|' was treated as part of
the |\hskip| command in |\nextnumber|. Quick should put `|\relax|'
at the end of his macro. \ (The ^{keywords} ^|l|, ^|plus|, |minus|,
^|width|, ^|depth|, or ^|height| might just happen to occur in text
when \TeX\ is reading a glue specification or a rule specification;
designers of general-purpose macros should guard against this.
If you get a `^|Missing number|' error and you can't guess
why \TeX\ is looking for a number, plant the instruction
`|\tracingcommands=1|' shortly before the error point; your log file
will show what command \TeX\ is working~on.)

\ansno27.5:
 If this exercise isn't just a joke, the title of this
appendix is a lie.
